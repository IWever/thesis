\section{Reported accidents}
\label{sec:accidents}
As mentioned in the introduction is there a desire from the scientific community to do research into the topic of modeling complex systems to improve safety. But the research becomes more relevant if there is a practical application. The clearest case for non-sufficient safety measures is of an accident. Accidents have often been incentives to adapt rules. For example the leading international treaty for safety of life at sea (\ac{SOLAS}) has been adopted in response to the Titanic disaster. But despite new rules, still occur. Below some accidents are discussed and what caused them to happen.

\subsection{Titanic}
The Titanic disaster on 15th April 1912, is on of the most famous accidents in history, an illustation of the collision is shown in figure \ref{fig:Accident-Titanic}. Many things went wrong, which led to the death of more than 1500 people. It is known that the crew ignored warnings for ice packs and ice bergs in the area, as the Titanic did not reduce her speed. When the accident occurred, the ship was sailing almost at top speed, as the captain did not believe the ice was a serious risk to this ship. Thereby did the last minute manoeuvrer to avoid collision fail, causing more damage than a head-on collision. And after the accidents were the distress signals not received by the nearest ship, as the radio operator was off duty and they did not respond to the flares. New regulations and technologies, mitigate the risk of this happening again. But still the human factor led to many accidents.

\begin{figure}[H]
	\centering
	\includegraphics[width=.7\textwidth]{Titanic-accident.png}
	\caption{Illutstration map of approximate collision location}
	\label{fig:Accident-Titanic}
\end{figure}

\newpage
\subsection{Al Oraiq and MV Flinterstar}
During the night between 5 and 6 October 2015 on the Northsea near Zeebrugge, a collision occurred between the LNG tanker Al Oraiq, sailing under the Marshall Islands flag, and the Flinterstar cargo ship, sailing under the Dutch flag. The Flinterstar sank almost immediately as a result of the collision, an illustration of the accident is shown in Figure \ref{fig:Accident-Flinterstar-Al-Oraiq}. The captain of the Flinterstar was badly injured in the incident but the other ten people on board and the pilot were rescued out of the water unharmed.

\begin{figure}[H]
	\centering
	\includegraphics[width=.7\textwidth]{Flinterstar-Al-Oraiq-accident.png}
	\caption{Illutstration map of approximate collision location}
	\label{fig:Accident-Flinterstar-Al-Oraiq}
\end{figure}

The collision occurred because the bridge team on board of the Al Oraiq wrongly assessed the traffic situation, vessel's speed and distance from the S3 buoy, prior to contacting the nearby vessel Thorco Challenger. After informing the Thorco Challenger, did they pass on starboard side. On board of Al Oraiq were coastal pilots witch did not receive feedback from the watch keeping team, nor was there feedback from other vessels via \ac{VHF} radio. The communication which went via VHF radio was mostly in dutch, the officer on duty at Al Oraiq did not request the Coastal pilots to translate. Also did the bridge watch team not asses the situation properly, leading to very little situation awareness.
On board of the Flinterstar there was insufficient attention for watch keeping duties. As several VHF radio communications between Traffic Centre Zeebrugge and other participants within the area monitored by Traffic Centre Zeebrugge, concerning or involving the presence of an inbound LNG carrier were missed by the Pilot and the bridge watch keeping team on board the Flinterstar.
The pilots on board of Al Oraiq did not attempt to work together. Thereby making decisions without consulting the crew, such as overtaking other vessels. Thus the coastal pilot did not act consistent with international understanding, where a pilot is an advisor to the ship's master. Which means mutual understanding for the functions and duties of each other, based upon effective communication and information exchange. 
The sea pilot on board of the Flinterstar got engaged in a casual conversation with the officer of the watch, drawing his attention away from monitoring the traffic situation. The Sea Pilot was advising the officer of the watch from what appeared to be routine. \cite{Backer2015}

\newpage
\subsection{USS Fitzgerald and ACX Crystal}
A more recent well known collision was between the USS Fitzgerald and ACX Crystal on 17th June 2017. The US destroyer hit the larger Philippines container vessel resulting in the death of 7 US Sailors. An illustration of the accident is shown in figure \ref{fig:Accident-USS-Fitzgerald-Crystal}. According to the accident report did failures occurred on the part of leadership and watch-standers. There were failures in planning for safety, adhere basic navigational practice, execute basic watch standing practice, proper use of available navigation tools and wrong responses.

\begin{figure}[H]
	\centering
	\includegraphics[width=.7\textwidth]{USS-Fitzgerald-Crystal-crash.png}
	\caption{Illutstration map of approximate collision location}
	\label{fig:Accident-USS-Fitzgerald-Crystal}
\end{figure}

In accordance with international rules the USS Fitzgerald was obligated to manoeuvrer to remain clear from the other crossing ships. The officer of the deck responsible for navigation and other crew discussed whether to take action but choose not to, till it was too late. While other crew members also failed to provide more situational awareness and input to the officer of the deck. Did the officer of the deck, exhibit poor seamanship by failing to manoeuvrer as required, failing to sound the danger signal and failing to attempt to contact CRYSTAL on Bridge to Bridge radio. In addition, the Officer of the Deck did not call the Commanding Officer as appropriate and prescribed by Navy procedures to allow him to exercise more senior oversight and judgment of the situation. This was prescribed to an unsatisfactory level of knowledge of the international rulles of the nautical road by USS Fitzgerald officers. Thereby were watch team members not familiar with basic radar fundamentals, impeding effective use. Thereby were key supervisors not aware of existing traffic seperation schemes and the expected flow of traffic, as the approved navigation track did not account, nor follow the Vessel Traffic Separation Scheme. Secondary was the automated identification system not used properly. \cite{USSNavy2017}

\newpage
\subsection{USS John S McCain and Alnic MC}
Even more recent is the collision between the USS John S McCain and Alnic MC on 21st August 2017. The US Destroyer hit the Liberia flagged oil and chemical tanker. Resulting in the death of 10  US Sailors. An illustration of the accident is shown in figure \ref{fig:Accident-USS-John-S-McCain-Alnic}. According to the accident report did the US Navy identify the following causes for the collision: Loss of situational awareness in response to mistakes in the operatoin of the USS John S McCain's steering and propulsion system, while in the presence of a high desnity of maritime traffic. Failure to follow the international nautical rules of the road, which govern the manoeuvring of vessels when risk of collision is present. Watchstanders operating the John S McCain's steering and propulsion systems had insufficient proficiency and knowledge of the system. 

\begin{figure}[H]
	\centering
	\includegraphics[width=.7\textwidth]{USS-John-S-McCain-Alnic-accident.png}
	\caption{Illutstration map of approximate collision location}
	\label{fig:Accident-USS-John-S-McCain-Alnic}
\end{figure}

Leading up to the accident did the commanding officer notice that the helmsman had difficulties maintaining course, while also adjusting the throttles for speed control. In response, he ordered the watch team to divide the duties of steering and throttles, maintaining course control with the Helmsman while shifting speed control to another watchstander. This unplanned shift caused confusion in the watch team, which led to wrong transfers of control, where crew was not aware of. 
Watchstanders failed to recognize this configuration. The steering control transfer caused the rudder to go amidships (centerline). Since the Helmsman had been steering less than 4 degrees of right rudder to maintain course before the transfer, the amidships rudder deviated the ship’s course to the left. Additionally, when the Helmsman reported loss of steering, the Commanding Officer slowed the ship to 10 knots and eventually to 5 knots. Due to the wrong transfer did only one shaft slow down, causing an un-commanded turn to the left (port). The commanding officer and others on the ship's bridge lost situational awareness. They did not understand the forces acting on the ship, nor did the understand the Alnic's course and speed relative to USS John S McCain. Three minutes after the reported loss of steering, was it regained, but already too late to avoid collision. No signals of warning were send by neither ship, which are required by international rules of the nautical road. Nor was there an attempt to make contact trough the \ac{VHF} bridge-to-bridge communication.
Many of the decisions made that led to the accident were the result of poor judgemenet and decision making of the commanding officer. That said, no single person bears full responsibility for this incident. The crew was unprepared for the situation in which they found themselves through a lack of preparation, ineffective command and control. Deficiencies in training and preparations for navigation were at the base of this. \cite{USSNavy2017}