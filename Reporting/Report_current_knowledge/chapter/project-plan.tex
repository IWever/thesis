\section{Project plan}
Based on the insight acquired from the previous chapters a plan of approach can be made for the next steps within my research. This will eventually lead to a desired result and a timeline towards this goal. With a summary how this is positioned within the scientific landscape and the key resources.

\subsection{Desired result}
With the knowledge from the previous chapters a framework can be formed for my research. Hereby taken into account what has already been done and how this can be used.
Although the research consists of two parts. The first part is for the department of Maritime Technology and Transport. The second part of the research will be for the Interactive Intelligence department of computer science. Will the result incorporate the conclusions of both parts, resulting in a single adaption to the bridge system.

This adaption will show the complexity of a situation, which is correlated with te probability of failure. Using a white box approach the hazards. Will there be an overlay for the ECDIS showing areas with increased risk and an advised sailing route, using different colors. Beside this overlay a list of advised actions will be presented, including the desired operations and communication.

\subsection{Timeline}
\begin{table}[h]
	\centering
	\caption{Timeline}
	\label{tab:timeline-project}
	\begin{tabular}{l||l|l}
		Bescrhijving & Start & Deadline \\ \hline \hline
		Project plan met initiele planning &  & 6 okt \\ \hline
		Samenvatting gerelateerde projecten en onderzoeken & 4 okt & 20 okt \\ \hline
		Plan van aanpak met opzet voor rapport & 9 okt & 3 nov \\ \hline
		Inlezen in papers en projecten en vertalen naar current knowledge & 20 okt & 19 nov \\ \hline
		Definieer well-clear op basis van regulations en company policy & 1 nov & 19 nov \\ \hline
		Plan van aanpak definitief maken & 20 nov & 26 nov \\ \hline
		Programma van eisen voor tool opstellen en test voor CS & 26 nov & 17 dec \\ \hline
		Onderzoek naar cost functie en modellen benodigd vanuit MT perspectief & 17 dec & 10 feb \\ \hline
		Opzet voor onderzoek met crew & 5 feb & 10 feb \\ \hline
		Ontwikkelen van GUI en tool op basis van programma van eisen & 12 feb & 5 mar \\ \hline
		Testen met crew & 5 mar & 2 apr \\ \hline
		Tool verbeteren op plekken waar meer detail nodig is & 19 mar & 6 apr \\ \hline
		Opnieuw testen en vergelijken met eerdere tests & 2 apr & 13 apr \\ \hline
		Vergelijking maken tussen theorie en praktijk & 16 apr & 30 apr \\ \hline
		Rapport CS en MT finaliseren & 30 apr & 4 jun \\ \hline
		Artikel schrijven & 4 jun & 29 jun \\ \hline
		Presentatie & 24 jun & 6 jul
	\end{tabular}
\end{table}

\subsection{Research questions}

\subsection{Scope}
\todo{write the scope of the project, already in graduation project plan.docx}

\subsection{Chosen vessels}
\todo{write down which ships will be used, and in which cases}