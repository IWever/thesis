\chapter{Model}
For the maritime technology part of the research a model will be develop incorporating different steps to determine the hazards and probability of failure. Ultimately this should lead to safe decision making by the crew. To get here different steps are taken, as shown in 





\section{Physical models}

\subsection{Manoeuvrability}
How does the inertia of ship work, and movements due to props and rudder.

Abkowitz defined in 1964 a simple model where position (X, Y) and rotation (N) depends on speed, accelation and rudder angles. Including hydrodynamic forces and moments. This is needed to calculate the path. 

\subsection{Environmental forces}
How are we going to model the wind, wave and current forces

\section{Route planning}
What are key issues in optimizing the route
Model predictive control - Tor Arne Johansen

\section{Cost function}

\section{Monitoring}
\subsection{Ship}

\subsection{Environment}
what is the consequence if there is no AIS available, which other methods are there to warn for hazards like that?

- well-clear is also based on good seaman ship, slow down when there is a difficult situation ahead


notes to https://www.youtube.com/watch?v=zsxQUzBinLc
Safe passage of singapore strait
- COLREG 19,20 and 35 - bas visibility
- COLREGs 6 visibility, sea room, traffic density

which rules do apply
effect of trafic seperation scheme (TSS)


user stories:
look well ahead
request information VTIS and what is VTIS
good practice to inform VTIS of entering a zone
navigational advice from VTIS not leading --> important to have system which shows if ship is doing what is discussed
Destination in AIS is not always correct
Determine what would be logical and do they follow this path, if something strange happens. Immediate warning and increase of probability
SMS specifies manning, also for watchkeeping
IALA Buoyage system says what buoy says what
https://www.navcen.uscg.gov/?pageName=AISMessage21