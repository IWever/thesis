\section{Knowledge of the crew}
Seafarers are a group which are protected by international maritime treaties: \ac{SOLAS}, \ac{STCW} and \ac{MARPOL}. Despite the regulations does human behaviour still lead to most accidents at sea. This complex multi-dimensional issue affects maritime safety and marine environmental protection. It involves the entire spectrum of human activities, performed by ships' crews, shore based management, regulatory bodies, recognized organizations, shipyards, legislators, and other relevant parties, all of whom need to cooperate to address human element issues effectively \cite{IMO2017}.
Fortunately a lot of research is preformed around human behaviour, there is thus an opportunity to keep improving the way people are involved. And there capabilities are utilized while mitigating there vulnerabilities.
The ships' crew is leading in this research. Therefore is looked at education they receive and how there knowledge developed. As this determines how they interpret a situation.
Regardless of the entry level, every seafarer has to gain years of experience to earn the job title of ship captain, beside several licences and certificates. \todo{extend which licenses and the amount of experience needed}

Experience is helps to improve situation awareness, as the skill to scan for hazards is more developed \cite{Underwood2013}. Hereby is important to notice that situation awareness is not limited to perceiving, but has multiple levels. This is known as the Endsley model (figure \ref{fig:Endsley-SA-model}), the three levels are \cite{Kalloniatis2017}: 
\begin{itemize}
	\item \emph{Perception}. Data is merely perceived.
	\item \emph{Comprehension}. Interpretation of data, enabling understanding of relevance in relation to tasks performed and goals to be attained. Forming an holistic picture of the operational environment. Identifying the significance of objects and events in that environment.
	\item \emph{Projection}. Making a forecast for likely future states of the situation . This is based on the interpreted data, experience and knowledge.
\end{itemize}

\begin{figure}[hb]
	\centering
	\includegraphics[width=.9\textwidth]{Endsley-SA-model-png.png}
	\caption{Endsley model for Situation awareness}
	\label{fig:Endsley-SA-model}
\end{figure}

To explain why someone predicts a future state that will occur, it is important to get insight in the mental model of the crew.
The mental model is the mechanism which describes elements in the environment within a volume of space and time. Giving explanations of system functioning, observed system states, and predictions of future system states. Done for a specific representation of the real system, for only selected concepts and relationships \cite{Kalloniatis2017}. 
The selected concepts and relationships are based on the background of the person. This is the reason why an economist and an engineer will have completely different mental models when looking at a ship. Where the economist sees it as an investment with related cost and returns. Will the engineer focus more on the way how it sails, propels itself and stays upright. 
The focus of the crew will be on the state of the vessel and the environment. Does the machinery work, what is the operational status of the vessel, what is the speed of the vessel, what is the wind and current speed, will they encounter bad weather, are there other vessels, does the vessel follow the planned route, etc. This means that a well designed bridge and a high quality planning are needed, to be able to understand the risks and know which information is desired when. 
When this does not happen loss of situation awareness occurs. This is according to Sandhaland, based on accidents at the north sea caused by: inadequate design, planning failure, communications failure, distracting elements and insufficient training. 
The consequences have been failure in monitoring the vessels status. For example if the steering was on auto-pilot or manual, detecting obstacles during bad visibility, or not receiving the right thruster status. In some cases it went a level deeper in situation awareness, where the crew received the information. But did not make the right decisions based on this. This was often because the crew was not aware of the risk involved and the effect of operating with the system configured in a specific way. For example when a thruster was deselected, lowering the redundancy \cite{Sandhaland2015}.





http://www.imo.org/en/OurWork/HumanElement/Pages/Default.aspx 
 
Captain/crew learns to sail a ship. 
Do this based on rules from for example IMO, but also based on experience. 
Experience helps to create an image of the world around them. 
This is also known as situation awareness. 
There are different levels of situation awareness (show picture). 
Last level is predicting. This is used to make decision. Can be rational and logical. Under stress this changes, therefore do they train.
Well trained people have predictable mental models. Thus there choices can be transformed into a model.
In case of captain/crew they ask themselves the following questions.

\subsection{Situation Awareness}

\subsection{Decision making and behaviour}

\subsubsection{Normal situation}

\subsubsection{Under stress}

\subsection{Mental models}

\subsubsection{Theories}

\subsubsection{Questions by crew}