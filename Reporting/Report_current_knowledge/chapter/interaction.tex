\section{Knowledge of the crew}
Seafarers are a group which are protected by international maritime treaties: \ac{SOLAS}, \ac{STCW} and \ac{MARPOL}. Despite the regulations does human behaviour still lead to most accidents at sea. On the other hand, people are involved in the design, build, crewing, maintenance, repair and salvage. All with the same set of capabilities and vulnerabilities. The fortunate thing is that a lot is known about human behaviour and thus an opportunity to keep improving the way people are involved.

But lets start at the beginning, how will you become a captain. As this is leading for their knowledge and how they will enter a situation. 

http://www.imo.org/en/OurWork/HumanElement/Pages/Default.aspx

 
Captain/crew learns to sail a ship. 
Do this based on rules from for example IMO, but also based on experience. 
Experience helps to create an image of the world around them. 
This is also known as situation awareness. 
There are different levels of situation awareness (show picture). 
Last level is predicting. This is used to make decision. Can be rational and logical. Under stress this changes, therefore do they train.
Well trained people have predictable mental models. Thus there choices can be transformed into a model.
In case of captain/crew they ask themselves the following questions.

\subsection{Situation Awareness}

\subsection{Decision making and behaviour}

\subsubsection{Normal situation}

\subsubsection{Under stress}

\subsection{Mental models}

\subsubsection{Theories}

\subsubsection{Questions by crew}