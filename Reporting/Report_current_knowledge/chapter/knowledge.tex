\chapter{Knowledge of the crew}
\label{sec:knowledge}
Seafarers are protected by international maritime treaties such as \ac{SOLAS}, \ac{STCW} and \ac{MARPOL}. Despite the regulations does human behavior still lead to most accidents at sea. This complex multi-dimensional issue affects maritime safety and marine environmental protection. It involves the entire spectrum of human activities, performed by ships' crews, shore-based management, regulatory bodies, recognized organizations, shipyards, legislators, and other relevant parties, all of whom need to cooperate to address human element issues effectively \cite{IMO2017}.
Fortunately, a lot of research is performed around human behavior, there is thus an opportunity to keep improving the way people are involved. And their capabilities are utilized while mitigating their vulnerabilities.
This chapter will discuss research on situational awareness, how this impacts the crew, what crew there is and how they can be supported.

\section{Situation awareness}
Experience is key to improve the skill to scan for hazards. This is known as the situation awareness \cite{Underwood2013}. Hereby is important to notice that situation awareness is not limited to perceiving, but has multiple levels. This is known as the Endsley model (figure \ref{fig:Endsley-SA-model}), the three levels are \cite{Kalloniatis2017}: 
\begin{itemize}
	\item \emph{Perception}. Data is merely perceived.
	\item \emph{Comprehension}. Interpretation of data, enabling understanding of relevance in relation to tasks performed and goals to be attained. Forming a holistic picture of the operational environment. Identifying the significance of objects and events in that environment.
	\item \emph{Projection}. Making a forecast for likely future states of the situation. This is based on the interpreted data, experience and knowledge.
\end{itemize}

\begin{figure}[H]
	\centering
	\includegraphics[width=.7\textwidth]{Endsley-SA-model-png.png}
	\caption{Endsley model for Situation awareness}
	\label{fig:Endsley-SA-model}
\end{figure}

\section{Mental model}
To explain why someone predicts a future state that will occur, it is important to get insight to the mental model of the crew.
The mental model is the mechanism which describes elements in the environment within a volume of space and time. Giving explanations of system functioning, observed system states, and predictions of future system states. Done for a specific representation of the real system, for only selected concepts and relationships \cite{Kalloniatis2017}. 
The selected concepts and relationships are based on the background of the person. This is the reason why an economist and an engineer will have completely different mental models when looking at a ship. Where the economist sees it as an investment with related cost and returns. Will the engineer focus more on the way how it sails, propels itself and stays upright. 
The focus of the crew will be on the state of the vessel and the environment. Does the machinery work, what is the operational status of the vessel, what is the speed of the vessel, what is the wind and current speed, will they encounter bad weather, are there other vessels, does the vessel follow the planned route, etc. This means that a well-designed bridge and a high-quality planning are needed, to be able to understand the risks and know which information is desired when. 
When this does not happen loss of situation awareness occurs. This is according to Sandhaland, based on accidents at the north sea caused by: inadequate design, planning failure, communications failure, distracting elements and insufficient training. 
The consequences have been a failure in monitoring the vessels status. For example, if the steering was on auto-pilot or manual, detecting obstacles during bad visibility, or not receiving the right thruster status. In some cases, it went a level deeper in situation awareness, where the crew received the information. But did not make the right decisions based on this. This was often because the crew was not aware of the risk involved and the effect of operating with the system configured in a specific way. For example, when a thruster was deselected, redundancy was lowered, which finally led to the accident. \cite{Sandhaland2015}.

When the crew was not aware of the risk involved, they had a wrong mental model of the system. The best way to improve someone's mental model is by training. This training is mostly focussed on acquiring experience in specific situations. The decision process in these situations is often dynamic, biased by individual perspective and goals, conditioned by previous experiences, including many system components and nonlinear relationships. This is also why it is hard to learn from normal data sources such as books or video. But experience has to be acquired to store situations directly in the mental model of the crew. Consequently, failures and conflicts are needed to improve these processes. As they show boundaries for decision making. This can be done using scenario-based training environments using virtual reality. It has been shown that training in those environments will lower stress and enhance professional performance in real-life situations \cite{Ford1998} \cite{Cohen2016}.

\newpage
\section{Shipping crew}	
Already some roles within the shipping crew were mentioned. But to give more insight here is a summary of the different roles on board of a merchant's vessel. At smaller vessels, roles are combined where possible. The Navy has in some cases even more operational crew members. In figure \ref{fig:crew-structure} the structure is shown for officers. Apart from the licensed officers who manage the vessel, does the crew also consist of ratings who have hands-on skills within their own domain. \cite{Nedcon2013}

\begin{figure}[H]
	\centering
	\begin{tikzpicture}[
		level 1/.style={sibling distance=40mm},
		edge from parent/.style={->,draw},
		>=latex]
		
		% root of the the initial tree, level 1
		\node[root] {Master}
		% The first level, as children of the initial tree
		child {node[level 2] (c1) {Deck crew}}
		child {node[level 2] (c2) {Engine crew}};
		
		% The second level, relatively positioned nodes
		\begin{scope}[every node/.style={level 3}]
			\node [below of = c1, xshift=30pt] (c11) {Chief officer};
			\node [below of = c11] (c12) {Second officer};
			\node [below of = c12] (c13) {Third officer};
			\node [below of = c13] (c14) {Deck cadet};
			
			\node [below of = c2, xshift=30pt] (c21) {Chief engineer};
			\node [below of = c21] (c22) {First engineer};
			\node [below of = c22] (c23) {Second engineer};
			\node [below of = c23] (c24) {Third engineer};
			
		\end{scope}
		
		% lines from each level 1 node to every one of its "children"
		\foreach \value in {1,2,3,4}
		\draw[->] (c1.195) |- (c1\value.west);
		
		\foreach \value in {1,2,3,4}
		\draw[->] (c2.195) |- (c2\value.west);
	\end{tikzpicture}
	\caption{Crew structure basic}
\label{fig:crew-structure}
\end{figure}

\subsection{Deck crew}
\label{sec:deck-crew}
The Deck crew is in charge of the vessel navigation, watch keeping, maintaining the ship’s hull, cargo, gear and accommodation, taking care of the ship’s lifesaving and firefighting appliances. The deck department is also the one in charge of receiving, discharging and caring for cargo. According to the vessel’s hierarchy, the deck officers are as follows: Master, Chief Officer, Second Officer, Third Officer and Deck Cadet (deck officer to be).

The supreme authority on board a merchant's vessel is the Master or Captain. The entire crew is under his command. He is responsible for the safety, use and maintenance of the vessel and makes sure that every crew member carries out his work accordingly. He is also in charge of the following: payroll, ship’s accounting, inventories, custom and immigration regulations, and the ship’s documentation. In order to become Master, a seafarer must first have several years of experience as a deck officer and as Chief Officer.
According to the vessel’s hierarchy, the first deck officer and the head of the deck department after the Master is the Chief Officer. He is in charge of the vessel navigation, watch duties, charging and discharging operations. The Chief Officer also directs all the other officers on deck, creates and posts watch assignments and implements the Master’s orders in order to maintain safe operations and maintenance of the vessel.
Second Officer or Second Mate is the next in rank after the Chief Mate and is the ship’s navigator, focusing on creating the ship’s passage plans and keeping charts and publications up to date. Apart from watchkeeping, the Second Officer may also be designated to train the cadets on board or to fulfill the rank of security, safety, environmental or medical officer.
The Third Officer or Third Mate is the fourth deck officer in command and is usually the Ship’s Safety Officer, responsible for ensuring the good functioning of the fire-fighting equipment and lifesaving appliances. He undertakes bridge watches and learns how to become a Second Officer.
A Cadet on board a merchant's vessel receives structured training and experience on board and learns how to become a deck officer.

Apart from the officers, the deck department crew also consists of ratings, such as AB (Able Body Seaman), OS (Ordinary Seaman) and Boatswain.
The AB is part of the deck crew and has duties such as: taking watches, steering the vessel, assisting the Officer on watch, mooring and un-mooring the vessel, deck maintenance and cleaning. The AB also secures and un-secures the cargo and carries our deck and accommodation patrols.
OS is the crew member whose main duty is to maintain the cleanliness of the whole ship and serves as an assistant for the AB. Being an OS is considered to be an apprenticeship, a period called “sea time” in order to be allowed to take courses and training for AB.
Both AB and OS are usually supervised by a Boatswain, who is also a rating, in charge of examining the cargo-handling gear and lifesaving equipment as well. The Boatswain usually holds an AB certificate as well.
The structure for the deck department on board merchant vessels is mainly the same on all vessel types. \cite{Nedcon2013}

\subsection{Engine crew}
The engine crew is responsible for operating, maintaining and repairing, the propulsion and support system. The engine department is also responsible with the repair and maintenance of other systems, such as: lighting, lubrication, refrigeration, air conditioning, separation, fuel oil, electrical power and so on.
According to the vessel’s hierarchy, the engine officers are as follows: Chief Engineer, Second Engineer, Engine Watch Officer, Electrician Officer and Engine Cadet.

The first engine officer and in charge of the engine department is the Chief Engineer. He takes complete control of the engine room and must make sure that every system and equipment runs by the book and is suitable for inspection at all times. The Chief Engineer also maintains up to date inventory for spare parts, extra fuel and oil and delegates the tasks for the officers under his command. In order to become a Chief Engineer, a seafarer must first be a Second Engineer with at least two years sea time experience.
After the Chief Engineer, in charge with the engine room is the Second Engineer, who also has a management level position. He assists the Chief Engineer in keeping the vessel running efficiently, is responsible for supervising the daily maintenance and operation of the engine room and prepares the engine room for arrival, departure or other operations. He reports directly to the Chief Engineer.
The Engine Watch Officer position is usually held by the Third or Fourth Engineer and it is an operational level job. The Third Engineer is usually responsible for the change of boilers, fuel, the auxiliary engines, condensate and feed systems. The Fourth Engineer is the most inexperienced officer, who has duties assigned by the Second Engineer, and some of his responsibilities are: engine watch, air compressors, purifiers and other auxiliary machinery.
Another officer working in the engine room is the Electrical Engineer, in charge of overseeing and ensuring the maintenance and proper functioning of all the electrical systems and machinery. The Electrical Engineer responds directly to the Second Engineer and to the Chief Officer and has to have the proper training to do this job.

Some merchant vessels also have amongst its crew members an Engine Cadet or Electrical Cadet, who receive structured training and experience on board and learn how to become an engine or electrical officer.
Apart from the officers, the engine department crew also consists of ratings, such as Motorman, Fitter, Electrician, Pumpman and Oiler/wiper.
The Motorman is the engine rating who keeps watch and assists the engine officers when performing maintenance tasks. He also participates in maintaining and repairing the main and auxiliary engines, pumps and boilers.
On board vessels, the Fitter carries out daily maintenance and engine cleaning jobs and is also specialized in fabrication, welding or repairing.
The Electrician on board a merchant's vessel is the rating working on the electrical equipment and systems, wiring and high voltage panels.
Mostly on tanker vessels, we may also find a Pumpman, responsible with the liquid cargo transfer system, pumps, the stripping pumps, filters valves, deck machinery involved in the liquid cargo transfer etc. His main job is to keep the liquid cargo system on a tanker running accordingly.
The Oiler or Wiper on board is the rating in charge with cleaning the engine spaces, machinery, lubricating bearings and other moving parts of the engine and assisting the engine officers in the general maintenance of the machinery in order to ensure the oil temperature is within standards and oil gauges are working properly.
Although the crew structure in the engine room is mainly the same, there are vessels that only have a part of the mentioned crew. This is due either to the size of the vessel or to financial reasons. \cite{Nedcon2013}

\section{Crew support}
All the different crew members have their responsibilities. In the short term, this won't change. Therefore this research will mostly focus on supporting the current crew. The focus is navigation and communication, where acquiring situational awareness is most important. Therefore all different members of the deck crew will be considered. As these crew members depend on developments and would use a decision support system to improve situation awareness.
Thereby will be considered which behavior does currently often lead to incidents and how this can be improved. There is most likely a difference between stressful situations and a controlled environment. This will be taken into account during the tests. Also the impact of regulations on these extra systems will be considered. As some of the regulations may have secondary reasons. For example is the loss of manual control skills due to automation is a serious concern in the highly automated aviation industries. Automation not only results in loss of psychomotor dexterity but also degradation of the cognitive skills required to accomplish the task successfully, especially in emergency situations \cite{Saffarian2012}.

%
%bijkomend voordeel van training is verlagen van stress
%
%
%
%
%http://www.imo.org/en/OurWork/HumanElement/Pages/Default.aspx 
% 
%Captain/crew learns to sail a ship. 
%Do this based on rules from for example IMO, but also based on experience. 
%Experience helps to create an image of the world around them. 
%This is also known as situation awareness. 
%There are different levels of situation awareness (show picture). 
%
%Last level is predicting. This is used to make decision. Can be rational and logical. Under stress this changes, therefore do they train.
%Well trained people have predictable mental models. Thus there choices can be transformed into a model.
%In case of captain/crew they ask themselves the following questions.
