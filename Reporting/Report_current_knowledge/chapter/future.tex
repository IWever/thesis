\section{Future of shipping}
\label{sec:future}
The shipping industry is traditionally driven by regulations. Digitalization and de-carbonization are watch words for the coming decade. Those will be leading so the shipping industry can become safer, more efficient and at the same time reduce its environmental footprint \cite{Eriksen2017}. \\

\subsection{Steps to be taken}
Focussing on the digitalization, ships will become more sophisticated. More data is generated by sensors, improved connectivity and new ways to visualise data. This enables ships to continuously communicate with managers and traffic controllers. At first this can be used to analyse data and give better advice based on expected weather, fuel consumption and arrivals at bottlenecks like ports and bridges.
Later on this can results in unmanned vessels. Either remotely operated from shore, on autopilot or completely autonomous, as shown in figure \ref{fig:From-manned-to-autonomous}. The different projects around the world follow this same path. Below some of these projects are mentioned with their current status.

\begin{figure}[hb]
	\centering
	\includegraphics[width=.75\textwidth]{From-manned-to-autonomous.jpg}
	\caption{From manned to autonomous ships}
	\label{fig:From-manned-to-autonomous}
\end{figure}

The research project MUNIN consists of a constortium of shipbuilders and scientists. The name is an abbreviation for Maritime Unmanned Navigation trough Intelligence in Networks. They did an initial research. Focussing on different elements of an autonomous concept: The development of an IT architecture. Analyse tasks preformed on today's bridge and how this will be on an autonomous bridge. Examine the tasks in relation with a vessel’s technical system and develop a concept for autonomous operation of the engine room. Define the processes in a shore side operation centre required to enable a remote control of the vessel. Thereby taking into account the feasibility of the developed solution, including legal and liability barriers for unmanned vessels.
They concluded that unmanned vessels can contribute to the aim of a more sustainable maritime transport industry. Especially in Europe, shipping companies have to deal with a demographic change within a highly competitive industry, while at the same time the rising ecological awareness exerts additional pressure on them. The autonomous ship represents a long-term, but comprehensive solution to meet these challenges, as it bears the potential to: Reduce operational expenses and
environmental impact.
An concept was developed for a bulker vessel, enabling the consortium to do a financial analysis. Showing the viability, but admitting the limited scope of the project \cite{MUNIN2016}.\\

\subsection{Current industry projects}
Rolls-Royce Marine is involved in different projects which are in some way follow-ups on the MUNIN project. Well-known are the videos of the virtual bridge concept and the Electric Blue. Electric blue is a concept ship, based on a standard 1000 \ac{TEU} feeder. The ship is very adaptable, it can sail for example on both diesel and electricity. The modularity enables it to adapt for specific routes and meet environmental requirements now, and in the future. 
Keeping in mind the way towards autonomous, will it have a virtual bridge, housed below the containers. Utilizing the opportunities of sensors for safe navigation, employing radar, camera, IR camera, lidar and \ac{AIS}. The roadmap for this concept is to have partial autonomy by 2020, remote operation between 2025 and 2030, starting with a reduced passive crew on board. And be fully autonomous in 2035 \cite{Wilson2017}. 
To make these steps they were aware from the start on, that the control room is the nerve centre of remote operations. Using an interactive environment with screen for decision support and improving situation awareness with augmented reality. With these developments does their vision look very promising. However it is still in a concept phase.\\

Just like MUNIN did this project also originate from WATERBORNE, an initiative from the EU and Maritime Industries Forum, supporting cooperation and exchange of knowledge between stakeholders within the deep and short sea shipping industry. Since June 2017 is Rolls-Royce also involved in the unmanned cargo ship development alliance, which is initiated by Asian companies and classification bureaus. Many of the projects where Rolls-Royce is involved, has DNV GL also a role. But beside these projects they are involved in other projects which look very promising.\\

First the projects on Norwegian ferries, which are likely to start sailing automated from 2018, just like an automated shuttle service for offshore installations. 
Already a step further is the Yara Birkeland, and 120 \ac{TEU} container ship. This vessel will initially operate as fully electric manned vessel, but plans are that it will sail autonomously in 2020. Operating between different Yara facilities, transporting fertilizers and raw materials. 
Kongsberg is responsible for the development and delivery of all key enabling technologies. Including the sensors and integration required for remote and autonomous operations, in addition to the electric drive, battery and propulsion control systems \cite{Sames2017}. \\

Where most of the previous projects were focussed around developing a vessel which has to operate in the current environment. Does the smart shipping challenge focus on combining technological developments within different parts of the inland shipping industry. This will help to steer ships remotely, smarter sharing of information and optimisation of waterway maintenance.
A good example are the new vessels from Nedcargo, the Gouwenaar 2 and 3. These vessels will be able to transport more containers, while reducing the fuel consumption. This will not only be acquired by improving the hull shape and machinery, but also by sailing smarter. For example by optimising the speed, based on opening times for bridges and availability of the quay \cite{SMASH2017}. \\ % Boudewijn Baan - Sales Manager - involved vanuit Damen

