\section{The bridge}
In section \ref{sec:knowledge} is discussed, how situation awareness is created by monitoring the vessel. The crew monitors the vessel from the bridge. The development of the bridge design is driven by technological advancement and regulatory demands. Which has increased the amount of equipment on a ship's bridge \cite{Lamme2017}. Leading to a all different components giving more and more data. As described by Speier are nowadays new technologies the primary reason for information overload. Not only because it produces more data, more quickly. But also that this information is disseminated more easily to people who do not need it \cite{Speier1999}. 

\subsection{Bridge elements}
The bridge of a vessel has four elements according to DNV-GL. The human operator, procedures, technical system and the human-machine interface. The safe operation of the vessel can only be assured when these are aligned. In figure \ref{fig:Bridge-system-elements} the different elements and their key factors are shown. Regulations aim to regulate these factors to ensure a safe performance of the bridge system to ensure system reliability in various modes of operation under different operating conditions. \cite{DNVGL2011}

\begin{figure}[hb]
	\centering
	\includegraphics[width=.5\textwidth]{Bridge-system-elements.png}
	\caption{Bridge system elements according to DNV GL}
	\label{fig:Bridge-system-elements}
\end{figure}

\subsection{Instruments and equipment}
Depending on the different station at the bridge certain equipment must be installed and within reach. But at least the following instruments and equipment shall be installed: navigation radar with radar, propulsion control, manual steering device (with take-over), heading control, \ac{ECDIS}, steering mode selector switch, \ac{VHF} unit, whistle and manoeuvring light push buttons, internal communication equipment, central alert management system \ac{UID}s, general alarm control, window wiper and wash controls, control of dimmers for indicators and displays, steering \ac{UID}s, propulsion, emergency stop for propulsion machinery, gyro compass selector switch and steering gear pumps.

These different systems have indicators with information on: propeller revolution, speed, windspeed and direction, rudder angle, rate-of-turn, heading, steering mode, steering position in command, depth indicator, clock, \ac{CAM-HMI}, alarm panel related to unmanned machinery space, alarm panel related to steering control system and steering gear, sound reception display and warning of surveillance period elapsing. 

Depending on the vessel some extra instruments can be for track control, steering control station selection, thruster \ac{UID}s and emergency stop for thrusters. Which give information on the thrust, pitch and when provided a conning information display. \cite{DNVGL2017}

\todo{Add pictures of bridge and different dashboards on screen, seperated for different vessels sizes}

\subsection{History of bridge design}
\todo{add years for steam, steel, technological developments}
A brief history of bridge design starts at the transition from sailing ships to paddle steamers in the 19th century. Where the captain's sight could not be obstructed by the paddle houses, and engineers needed a platform from which they could inspect the paddle wheels. A raised walkway was created between both paddle houses, literally a bridge. The name bridge stuck, even when ships started using screw propellers. From the bridge commands were passed via officers to the different stations. Where physical actions were carried out to control the ship. As technology did not exist to remotely control the ship. The helmsman or coxswain operated the ship's wheel from the enclosed wheel house, and engineers received commands in the engine room via engine order telegraphs. Where the bridge was often open to the elements, a weatherproof pilot house could be provided, from which the navigation officer could issue commands.

With the development of steel ship, came also the requirement of a compass platform. Sited as far away as possible from ferrous interference. Later this was solved with a binnacle. But this was another system which was introduced. In modern vessels most of the stations for physical control have been moved to the bridge. The rudder and throttle can be operated directly from the bridge. Due to previous accidents it is even common to have unmanned machinery spaces during operation in smaller ships. The technological developments have also lead to a variety of sytems as mentioned above. Starting with radar at the start of the 20th century. \ac{ECDIS} was another major step forward, where it was accepted in 1995 by IMO as up-to-date chart as required by \ac{SOLAS}. Later made mandatory in 2009 by \ac{SOLAS}, where \ac{STCW} requires \ac{ECDIS} competence for navigational officers and masters. This added an extra screen. Continuously adding other instruments for meteo, \ac{AIS}, echo sounder, different compasses, etc. First the different instruments had all a separate analogue instrument. Nowadays these are often displayed together at the conning display, while having the separate instruments at less convenient places.

\subsection{Future of bridge design}
This conning display already gave the opportunity to develop a more user oriented bridge environment. But with the continuous development of different sensors. Is a new revolution coming for bridge design using among others: sensor fusion, new ways of visualization and decision support.

Parallel to the research to autonomous vessels, are there many concept for future bridge designs. Both can be traced back to the changes in philosophy on human-computer interaction. Below some of the concepts are explained. Where is mostly is a combination of classification societies, research institutes and commercial companies.

\subsubsection{Concept designs}
Within Damen there is the desire to make the bridge design more standard and create more of a brand identity. An integrated design is desired in this case, where suppliers deliver the back-end. Similar projects which already show a future vision on ship design are the Ulstein Bridge Vision concept and Rolls Royce oX bridge. Where augmented reality in the windows and adapting user interfaces are key. With early warnings, decision support for economic sailing, environmental analysis, and having the ability to use the windows as screen to simulate operations. While it is clear that here lies the future of bridge design, they did not yet come with clear solutions. Although in projects like Waterborne some steps are made when it comes to the user interface.\cite{RollsRoyce2015} \cite{Ulstein2013}

\subsubsection{Research projects}
The CASCADe project has already been more research oriented and towards a practical solution. They have tried to develop a bridge system which adapts displayed content on the user interfaces to the current situation, relevant procedures and the needs of the crew. Using a virtual simulation platform which enables analysis of the cooperative bridge system purely based on models, in particular Cognitive Seafarer Models which mimic decision making and situation awareness processes of real human seafarers. The virtual platform allows a very careful evaluation of the Adaptive Bridge System to research solutions for adaptation which provide benefits (e.g., increase situation awareness) that outweigh their costs (e.g. cognitive disruption).\cite{CASCADe2015} \todo{add information on cognitive seafarer model} \todo{add more researches on bridge design (Myrthe Lamme)}

\subsubsection{Suppliers}
But currently bridges are already build and the companies developing these are also not standing still. But present more realistic there current status. The Kongsberg K-Master work environment integrates already different systems in one user interface. Dynamic positioning, manual propulsion and thruster control, alarm monitoring and remote control of machinery, central bridge alarm system, operation of auxiliary bridge systems and chart, radar, autopilot and conning displays are all combined. Where this system was originally only for the aft bridge, is it now used for a variety of vessels. \cite{Kongsberg2017}
\todo{Check systems Damen is using}
Many of the Damen vessels sail with the Dutch integrated bridge systems, which is the first step towards are more user centred design. Examples are Praxis' Mega-Guard IBS or Alphatron's AlphaBridge. Both modular bridge system use in-house developed instruments. Combining Radar, \ac{ECDIS}, conning, alarms, other ship systems and \ac{AIS}. 
Other major players in the development of integrated bridge designs are: Sperry Marine with the VisionMaster FT, Raytheon Anschütz Synapsis NX and for yachts Admarel.\\

\subsection{Manoeuvring display}
The CASCADe project already showed the advantages of an adaptive bridge system. Key in most of these systems is an modification to the overlay in the \ac{ECDIS} system. This is also chosen as starting point for bridge modifications in this project. As it is already possible using the \ac{ECDIS} system to add alarms and layers. The developed model will be a white box approach, this will help to define alarms and mark forbidden zones. Thereby need to be taken into account, the rules and regulations for the \ac{ECDIS} system as defined by the international hydrographic organisation.

\todo{How is a map build, S-57, .BSB; how to import and modify, preferably with python}

\subsection{Information overload}
The reason not too add another screen is to avoid information overload. While Sandhaland et al. already mentioned how information overload is one of the main reasons for loss of situation awareness. 

\subsection{Bridge users}
------ ------ ------

As mentioned in previous chapter does the crew make decisions based on situation awareness, created by the information given. This information they do receive at the bridge.

From rules is given an bridge is equiped with .... screen, sensors, etc.

This has been so far mostly an evolution, with putting more equipment up. Leading to a possible information overlead. 

There are projects at allewijnse/alphatron/praxis/Damen? who work on new bridge designs. This is the user interface of the ship.

Looking at design choices which can be made for an user interface you can find amount of screens, buttons/touch, offshore the demands are different. Thus should be changed. 

This is higly related to information overlead and decision making under stress

\subsection{Bridge design}

\subsection{User interface}

\subsection{Processable information}
In the cases where the crew was not able to monitor properly, often the right information was somewhere on the displays. But as that 

