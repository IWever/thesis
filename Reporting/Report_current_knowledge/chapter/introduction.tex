\chapter*{Introduction}
\addcontentsline{toc}{chapter}{Introduction}
\label{sec:introduction}
This document will function as a starting point for my research. Summarizing current knowledge, projects and papers, which gives insight where opportunities are to create new knowledge. The main subject of the research will be: Modelling the complexity of maritime situations, to determine the probability of failure, enabling bridge systems to give a clearer advice on potential hazards and safe areas. This is used by the shipping crew to acquire situational awareness faster, compared to traditional bridge systems and raw sensors.

As described in the Dutch National Research Agenda a fundamental question we have to ask ourselves is: How do we get grip on unpredictability of complex networks and chaotic systems? Within the Netherlands Top consortium for Knowledge and Innovation (TKI) exist to support innovation by connecting companies, governmental organizations and research institutes to work together in so-called Joint Industry Projects. TKI Maritime has four main themes for these projects. This research will be related to the smart and safe sailing theme. Which aims at lowering crew and maintenance cost, while maintaining a safe working environment. Developing a system which supports the crew during navigation by adding giving more insight into the complexity of a situation, and where hazards occur. This goal is met.

This research will have two main tracks. The first part is related to the Design track of Maritime Technology. Where I will try to specify a scale for the complexity of a situation, which is correlated with the probability of failure. The second part of the research will be for the Interactive Intelligence department of computer science. Focussing on the interaction between computers and people. In this case more specific, which information does the crew currently get and what information do they need. To improve their situational awareness and make better decisions during operation.

First the relevance of this project will be shown using accidents. Followed by a description of the future of shipping in general.
After that, more attention is given to the current shipping crew. Showing models to explain their behaviour and which functions they have. 
Followed by a detailed look at the history and future of bridge design. Including the most relevant system for this project, the manoeuvring display.
At last the project plan is discussed. What should be in the last chapter of the research, which steps are taken this next year, which questions should be answered and finally what will and will not be considered.
