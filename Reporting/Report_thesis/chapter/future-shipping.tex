\chapter{Steps towards the future}
\label{ch:future}
Research projects funded by the EU should aim at sustainable growth. This means, be competitive while also improving the safety of people working in the industry and reduce the environmental footprint \cite{Eriksen2017} \cite{EU2017}.
This research will mainly focus on improving safety at sea within existing projects developing autonomous ships. Most of these project build upon previously developed technologies which are currently already in use. This chapter will first discuss: Why the step towards autonomous and unmanned is taken? Followed by a description of current projects to develop new technologies. To finally identify which challenges will most likely arise in this process.

\section{Why autonomous and unmanned shipping}
Due to digitalization, ships will become more sophisticated. More data is generated by sensors, improved connectivity and new ways to visualize data. This enables ships to continuously communicate with managers and traffic controllers. At first, this can be used to analyze data and give better advice based on expected weather, fuel consumption and arrivals at bottlenecks like ports and bridges.
But further ahead this might result in unmanned vessels, which might be operated remotely. In parallel there is the transition where people are taken out the chain of commands, which will result in automated or completely autonomous vessels. The main arguments heard for the transition towards autonomous or unmanned ships:
\begin{itemize}
	\item \emph{Improved safety}, as most accidents are caused by human errors. Thereby will there be less crew at the ship, thus less crew is at risk when an accident occurs.
	\item \emph{Lower cost}, as insurance goes down due to improved safety. Thereby is manning a large portion of total cost. With more automation, less crew is needed, although they need to be schooled better.
	\item \emph{Higher productivity}, when there is better usage of data and communication. Thereby comes that computers don't have to work in shifts to go home or take breaks.
	\item \emph{Improve competitiveness}, as tankers which are traded for example, do not have to enter a harbor to get fresh supplies.
	\item \emph{More comfort and attractiveness industry}, as people can have more regular hours to work and do not have to be away for many weeks when working remote.
\end{itemize}
Thereby are maritime trade volumes expected to increase in the future and accordingly the numbers of ships needed to transport the freight will grow, as will the number of seamen required to operate the vessels. At the same time European shipping faces a lack of seafaring personnel already today. An often cited reason for this lies in the unattractiveness of seagoing professions, especially for youngsters. To some extend this is caused by seafaring’s inherent problem of lacking family friendliness and the high degree of isolation from social life that comes along with working on a seagoing ship. The current trend towards slower sailing speeds justified by ecologic and economic considerations increases the length of the ship’s voyage and with that the time seamen spend on sea even further.

Here, the unmanned autonomous vessel represents a way out of the impasse of a shortage in the supply of seafarer due to the job’s perceived unattractiveness and a growing demand for seafarer caused by slow steaming and increasing transport volumes.On the one hand, it could reduce the expected pressure on the labor market for seafarer as it would enable, at least partly, to reduce the labor intensity of ship operation. On the other hand, routine tasks on board would be automated and only the demanding but interesting navigational and technical jobs transferred from ship to a shore side operation center making “seafaring” jobs more attractive and family friendly than today. Furthermore, economic and environmental benefits are also expected when implementing unmanned shipping. \cite{MUNIN2016}

The different projects around the world, working on the transition towards autonomous or unmanned vessel, are mentioned with their current status. Thereby should be considered that the projects are working on different levels of automation. These different levels are shown in figure \ref{fig:automation-levels}.
Beside these levels of automation, are there also different types of automation, each with their own challenges. The types are shown in \ref{fig:manned-remote-autonomous}.

\begin{figure}[hb]
	\begin{subfigure}[b]{0.55\linewidth}
		\centering
		\includegraphics[width=\textwidth]{automation-levels.png}
		\caption{Levels of automation}
		\label{fig:automation-levels}
	\end{subfigure} 
	\begin{subfigure}[b]{0.4\linewidth}
		\centering
		\includegraphics[width=.89\textwidth]{manned-remote-automated-autonomous.png}
		\caption{Types of automation on ships}
		\label{fig:manned-remote-autonomous}
	\end{subfigure}
	\caption{Steps from manned to autonomous ships}
	\label{fig:automation} 
\end{figure}


\section{Projects}
The research project MUNIN has been one of the major projects by a consortium of shipbuilders and scientists. The name is an abbreviation for Maritime Unmanned Navigation through Intelligence in Networks and originated from WATERBORNE, an initiative from the EU and Maritime Industries Forum, supporting cooperation and exchange of knowledge between stakeholders within the deep and short sea shipping industry. They did an initial research between 2013 and 2016. Focussing on different elements of an autonomous concept: 
\begin{itemize}
	\item The development of an IT architecture. 
	\item Analysis tasks performed on today's bridge and how this will be on an autonomous bridge. 
	\item Examining the tasks in relation to a vessel’s technical system and develop a concept for autonomous operation of the engine room. 
	\item Define the processes in a shoreside operation center, required to enable a remote control of the vessel. 
\end{itemize}
Thereby taking into account the feasibility of the developed solution, including legal and liability barriers for unmanned vessels.
They concluded that unmanned vessels can contribute to the aim of a more sustainable maritime transport industry. Especially in Europe, shipping companies have to deal with a demographic change within a highly competitive industry, while at the same time the rising ecological awareness exerts additional pressure on them. The autonomous ship represents a long-term, but comprehensive solution to meet these challenges, as it bears the potential to: Reduce operational expenses and environmental impact.
A concept was developed for a bulker vessel, enabling the consortium to do a financial analysis. Showing the viability, but admitting the limited scope of the project \cite{MUNIN2016}. They have showed the importance of developing a method to determine intentions of other vessels and systems which are needed. But did not yet make the step towards developing such a method, which will be done in this report.

\subsection{Exploratory projects}
Different project worked on the vision about the future of shipping, often these projects have different phases in which the level of automation increases with every iteration. Examples of projects currently running all over the world are:
\begin{itemize}
	\item One Sea – Autonomous Maritime Ecosystem by DIMECC Ltd.
	\item Advanced Autonomous Waterborne Applications
	\item Unmanned Cargo Ship Development Alliance 
	\item ReVolt
\end{itemize}
Rolls-Royce Marine is involved in different projects which are in some way follow-ups on the MUNIN project. The videos of the virtual bridge concept and the Electric Blue vessel have had many views, as this showed clearly their vision of how the shipping industry could look like in the future. Electric Blue is a concept ship, based on a standard 1000 \ac{TEU} feeder. The ship is very adaptable, it can sail for example on both diesel and electricity. The modularity enables Electric Blue to adapt for specific routes and meet environmental requirements now, and in the future. 
Keeping in mind the way towards autonomous, will it start with a virtual bridge, housed below the containers. Utilizing the opportunities of sensors for safe navigation, employing radar, camera, IR camera, lidar and \ac{AIS}. The roadmap for this concept is to have partial autonomy by 2020, remote operation between 2025 and 2030, starting with a reduced passive crew on board. And be fully autonomous in 2035 \cite{Wilson2017}. 
They pinpointed the control room as the nerve center of remote operations. Using an interactive environment with a screen for decision support and improving situation awareness with augmented reality. With these developments does their vision look very promising. However there have not yet been successful prototypes.
Since June 2017 is Rolls-Royce also involved in the unmanned cargo ship development alliance, which is initiated by Asian companies and classification bureaus.

Make inforgraphic: combining different roadmaps, such as DIMECC and Rolls royce
\todo{Make infographic}
\todo{Describe different levels of automation}

\begin{figure}[hbp]
	\centering
	\includegraphics[width=0.9\textwidth]{one-sea-timeline.jpg}
	\caption{EXAMPLE-infographic: Timeline for autonomous ships by DIMECC}
	\label{fig:time-line-autonomous}
\end{figure}

\subsection{Industry projects}
Where the exploratory projects work on the vision and far future of autonomous. Are some companies working towards prototypes, often funded by customers of shipping companies.
The Yara Birkeland is one of the projects ahead of the pack, already building and testing a 120 \ac{TEU} container ship. This vessel will initially operate as fully electric manned vessel, but plans are that it will sail autonomously in 2020. Operating between different Yara facilities in Norway, transporting fertilizers and raw materials. Meaning the path and quay are always the same, which reduces the amount of challenges.
Kongsberg is responsible for the development and delivery of all key enabling technologies. Including the sensors and integration required for remote and autonomous operations, in addition to the electric drive, battery and propulsion control systems \cite{Sames2017}.

Other smaller projects are the development of Norwegian ferries, which are likely to start sailing automated from 2018, just like an automated shuttle service for offshore installations. A partly Dutch project is the Roboat, where a fleet of small pontoons will be used to solve problems on urban waterways. Such as transportation of people and goods or creating temporary dynamic floating structures like bridges and stages. Which is a collaboration between AMS Institute and MIT.

Where most of the previous projects were focussed around developing a vessel which has to operate in the current environment. Does the smart shipping challenge (SMASH) focus on combining technological developments within different parts of the inland shipping industry in the Netherlands. This will help to steer ships remotely, enable intelligent exchange of information and optimization of waterway maintenance.
Good examples are the new vessels from Nedcargo, the Gouwenaar 2 and 3. These vessels will be able to transport more containers, while reducing the fuel consumption. This will not only be acquired by improving the hull shape and machinery, but also by sailing smarter. For example by optimizing the speed, based on opening times for bridges and availability of the quay \cite{SMASH2017}. 

Also did a \acf{JIP} start on autonomous shipping under the name Sovereign. In this project did European companies, research institutes and the technical university of Delft partner to develop a technology to deal with difficult environments and complex transport missions within short sea and port traffic situations. Which is also applicable to other autonomous waterborne operations, such as inland waterways transport and coastal/inter-island short range ferry services. This should result in a ship for Amasus. \todo{more info on JIP and Sovereign} 

Based on the above mentioned projects, can the following be seen as possible use cases:
\begin{itemize}
	\item Local transport between factories and terminals
	\item Short sea shipping
	\item Tugs as extra actuator in dynamic positioning systems
\end{itemize}

\section{Stakeholders}
When these ships will sail, does not only dependent on the rate in which the technology can be developed. But there are also regulatory bodies, such as \ac{IMO} and classification societies which need incorporate autonomous vessels in their frameworks. 
For them the exploratory projects are very important, as this will help to know for which ships codes should be developed. These codes include information on autonomy levels and how to certify unmanned vessels.

The next group are the shipbuilders, system integrators and suppliers for subsystems. These are responsible for the technological development. More and more shipyards try to get involved, to gain knowledge on the development process. 
Thereby are there companies from other industries, which see opportunities for products they already developed for planes or automotive, which could also be used on unmanned vessel. For example to connect ships, computer vision and protocol for testing systems.

The last, but probably most important are the customers, as a technology will only be used if you can make money with it. More and more companies are convinced this is possible. This are not only the chartering companies, but also their customers, such as Heineken, Yara and BHP.

\todo{add model from sovereign}

\section{Challenges when combining unmanned and manned vessel}
\label{sec:challenges-future}
Based on the above mentioned projects it is clear that many project work on different challenges. The challenge of safe operation for unmanned vessels is one of the most important. This is mostly relevant when manned and unmanned vessels meet. As technologies for communication are costly to develop, the aim is to avoid communication when possible. To accomplish this, the first step is to adjust the operational strategies for unmanned ships to avoid complex situations where possible. This means a strategy should be developed on how these ships can avoid complex situations. The most easy way is to take decisions well in advance to make intentions clear. Still some challenges are open, as not all complex situations can be avoided. For these cases there must be a protocol which enables manned and unmanned ships to share the right information.

\emph{In the next chapter} factors are discussed which influence the decision making process. This is based on challenges from above mentioned projects and current research. 
















