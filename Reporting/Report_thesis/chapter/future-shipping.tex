% What possiblities to solve the problem did other come-up with? -> new technology
\chapter{Steps towards the future}
\label{ch:future}
Current projects funded by the EU aim at sustainable growth. This means, be competitive while also reducing the environmental footprint and improve the safety of people working in the industry. \cite{Eriksen2017} \cite{EU2017}
This research will mainly focus on improving safety at sea, being connected and able to message instantly is normal nowadays. Thereby also having the information where and when you want it. Many projects are working on manned and autonomous vessels. They have to build upon the previously developed technologies wich are currently used. These technologies are mentioned in chapter \ref{ch:systems}. This chapter will discuss the steps which will be taken in the near future. First, why the step towards autonomous and unmanned is taken. Followed by a description of current projects to develop new technologies. To finally identify that which challenges will most likely arise in this process.

\section{Why autonomous and unmanned shipping}
Focusing on the digitalization, ships will become more sophisticated. More data is generated by sensors, improved connectivity and new ways to visualize data. This enables ships to continuously communicate with managers and traffic controllers. At first, this can be used to analyze data and give better advice based on expected weather, fuel consumption and arrivals at bottlenecks like ports and bridges.
But further ahead this might result in unmanned vessels, which are operated remotely. In parallel there is the transition where people are taken out the chain of commands, which will result in automated or completely autonomous vessels. The main arguments heard for the transition towards autonomous or unmanned ships:
\begin{itemize}
	\item \emph{Improved safety}, as most accidents are caused by human errors. Thereby will there be less crew at the ship, thus less crew is at risk when an accident occurs.
	\item \emph{Lower cost}, as insurance goes down due to improved safety. Thereby is manning a large portion of total cost. With more automation, less crew is needed, although they need to be schooled better.
	\item \emph{Higher productivity}, when there is better usage of data and communication. Thereby comes that computers don't have to work in shifts to go home or take breaks.
	\item \emph{Improve competitiveness}, as tankers which are traded for example, do not have to enter a harbor to get fresh supplies.
	\item \emph{More comfort and attractiveness industry}, as people can have more regular hours to work and do not have to be away for many weeks when working remote.
\end{itemize}
Thereby are maritime trade volumes expected to increase in the future and accordingly the numbers of ships needed to transport the freight will grow, as will the number of seamen required to operate the vessels. At the same time European shipping faces a lack of seafaring personnel already today. An often cited reason for this lies in the unattractiveness of seagoing professions, especially for youngsters. To some extend this is caused by seafaring’s inherent problem of lacking family friendliness and the high degree of isolation from social life that comes along with working on a seagoing ship. The current trend towards slower sailing speeds justified by ecologic and economic considerations increases the length of the ship’s voyage and with that the time seamen spend on sea even further.

Here, the unmanned autonomous vessel represents a way out of the impasse of a shortage in the supply of seafarer due to the job’s perceived unattractiveness and a growing demand for seafarer caused by slow steaming and increasing transport volumes.On the one hand, it could reduce the expected pressure on the labor market for seafarer as it would enable, at least partly, to reduce the labor intensity of ship operation. On the other hand, routine tasks on board would be automated and only the demanding but interesting navigational and technical jobs transferred from ship to a shore side operation center making “seafaring” jobs more attractive and family friendly than today. Furthermore, economic and environmental benefits are also expected when implementing unmanned shipping. \cite{MUNIN2016}

The different projects around the world, working on the transition towards autonomous or unmanned vessel, are mentioned with their current status.
\todo{Reference to infographic with time-line unmanned/autonomous levels}

\section{Projects}
The research project MUNIN has been one of the major projects by a consortium of shipbuilders and scientists. The name is an abbreviation for Maritime Unmanned Navigation through Intelligence in Networks and originated from WATERBORNE, an initiative from the EU and Maritime Industries Forum, supporting cooperation and exchange of knowledge between stakeholders within the deep and short sea shipping industry. They did an initial research between 2013 and 2016. Focussing on different elements of an autonomous concept: 
\begin{itemize}
	\item The development of an IT architecture. 
	\item Analysis tasks performed on today's bridge and how this will be on an autonomous bridge. 
	\item Examining the tasks in relation to a vessel’s technical system and develop a concept for autonomous operation of the engine room. 
	\item Define the processes in a shoreside operation center, required to enable a remote control of the vessel. 
\end{itemize}
Thereby taking into account the feasibility of the developed solution, including legal and liability barriers for unmanned vessels.
They concluded that unmanned vessels can contribute to the aim of a more sustainable maritime transport industry. Especially in Europe, shipping companies have to deal with a demographic change within a highly competitive industry, while at the same time the rising ecological awareness exerts additional pressure on them. The autonomous ship represents a long-term, but comprehensive solution to meet these challenges, as it bears the potential to: Reduce operational expenses and environmental impact.
A concept was developed for a bulker vessel, enabling the consortium to do a financial analysis. Showing the viability, but admitting the limited scope of the project \cite{MUNIN2016}. They have showed the importance of developing a method to determine intentions of other vessels and systems which are needed. But did not yet make the step towards developing such a method, which will be done in this report.

\subsection{Research projects} \todo{change name so all projects are positive}
Rolls-Royce Marine is involved in different projects which are in some way follow-ups on the MUNIN project. Well-known are the videos of the virtual bridge concept and the Electric Blue vessel, as this was a first clearly drawn vision of how the shipping industry could look like in the future. Electric Blue is a concept ship, based on a standard 1000 \ac{TEU} feeder. The ship is very adaptable, it can sail for example on both diesel and electricity. The modularity enables Electric Blue to adapt for specific routes and meet environmental requirements now, and in the future. 
Keeping in mind the way towards autonomous, will it start with a virtual bridge, housed below the containers. Utilizing the opportunities of sensors for safe navigation, employing radar, camera, IR camera, lidar and \ac{AIS}. The roadmap for this concept is to have partial autonomy by 2020, remote operation between 2025 and 2030, starting with a reduced passive crew on board. And be fully autonomous in 2035 \cite{Wilson2017}. 
To make these steps they were aware from the start on, that the control room is the nerve center of remote operations. Using an interactive environment with a screen for decision support and improving situation awareness with augmented reality. With these developments does their vision look very promising. However there have not yet been successful prototypes.

Other projects which have clearly published there vision for the future are:
- DIMECC with (One Sea)\\
- AAWA\\
- ReVolt\\
\todo{give more info on projects and why my research has added value to them.}

Since June 2017 is Rolls-Royce also involved in the unmanned cargo ship development alliance, which is initiated by Asian companies and classification bureaus.
In the Netherlands a \ac{JIP} has started on autonomous shipping.
\todo{more info on JIP and Soevereign}

Make inforgraphic: combining different roadmaps, such as DIMECC and Rolls royce, thereby also showing the different levels of automation and unmanned
\todo{Make infographic}

\begin{figure}[b]
	\centering
	\includegraphics[width=0.9\textwidth]{one-sea-timeline.jpg}
	\caption{EXAMPLE-infographic: Timeline for autonomous ships by DIMECC}
	\label{fig:time-line-autonomous}
\end{figure}

\begin{figure}[b]
	\centering
	\includegraphics[width=.65\textwidth]{From-manned-to-autonomous.jpg}
	\caption{EXAMPLE-infographic: From manned to autonomous ships}
	\label{fig:From-manned-to-autonomous}
\end{figure}




\subsection{Industry projects}
Beside the visionary projects mentioned before, some companies are also coming with real results. Often funded by customers of shipping companies.
The Yara Birkeland is one of the projects ahead of the pack, already building and testing a 120 \ac{TEU} container ship. This vessel will initially operate as fully electric manned vessel, but plans are that it will sail autonomously in 2020. Operating between different Yara facilities in Norway, transporting fertilizers and raw materials. 
Kongsberg is responsible for the development and delivery of all key enabling technologies. Including the sensors and integration required for remote and autonomous operations, in addition to the electric drive, battery and propulsion control systems \cite{Sames2017}.

Other smaller projects are the development of Norwegian ferries, which are likely to start sailing automated from 2018, just like an automated shuttle service for offshore installations. A partly Dutch project is the Roboat, where a fleet of small pontoons will be used to solve problems on urban waterways. Such as transportation of people and goods or creating temporary dynamic floating structures like bridges and stages. Which is a collaboration between AMS Institute and MIT.

Where most of the previous projects were focussed around developing a vessel which has to operate in the current environment. Does the smart shipping challenge (SMASH) focus on combining technological developments within different parts of the inland shipping industry in the Netherlands. This will help to steer ships remotely, enable intelligent exchange of information and optimization of waterway maintenance.
Good examples are the new vessels from Nedcargo, the Gouwenaar 2 and 3. These vessels will be able to transport more containers, while reducing the fuel consumption. This will not only be acquired by improving the hull shape and machinery, but also by sailing smarter. For example by optimizing the speed, based on opening times for bridges and availability of the quay \cite{SMASH2017}. 

\section{Other stakeholders}
State of lloyds and IMO, developing codes for autonomy level and certification of unmanned vessels.
\todo{update with info from IRYOON}

\section{Possible use cases}
Summarizing the projects, based on this what would be likely use cases:
- Tugs as extra actuator in DP-like system \\
- Local transport between factories, terminals, etc. \\
- Crude oil tankers which are traded while at sea \\
- Short sea shipping (most pilots) \\

\section{Changes in communication}
Describe the impact on the different facets of the bridge (Technical system, man/machine interface, procedures, human operator).
e.g: VDES

\section{Challenges when combining unmanned and manned vessel}
\label{sec:challenges-future}
For unmanned vessels a new system for communication must be developed, as human speech is very hard for computers. The cost of development for this new system depends on the amount of situations it has to cope with. By adjusting the operational strategies for unmanned ships to avoid complex situations as much as possible, these situations can be kept to a minimum. 
The challenges which have to be tackled are therefore how these ships can avoid complex situations. This means that they have to take decisions well in advance so others are aware of their intentions. Still some challenges are open, as not all complex situations can be avoided. For these cases still must be thought about which information should be shared to make the right decision. Which is related to the question, which systems are needed to share this information.
















