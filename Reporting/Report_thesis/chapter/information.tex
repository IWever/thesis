\chapter{Information at the bridge}
The bridge of a vessel can be separated into four elements. The human operator, procedures,
technical system and the human-machine interface. This chapter will focus on the technical system and human-machine interface. Thereby a separation will be made between the instruments available, the information which can be deducted from this and how this can be used. 

\begin{figure}[H]
	\centering
	\includegraphics[width=0.4\textwidth]{Bridge-system-elements.png}
	\caption{Bridge system elements}
	\label{fig:Bridge-system-elements}
\end{figure}

The ship’s navigation bridge shall enable the officer in charge of the navigational watch to perform navigational duties unassisted at all times during normal operating conditions. He shall be able to maintain a proper lookout by sight and hearing as well as by all available means appropriate in the prevailing circumstances and conditions so as to make full appraisal of the situation and the risk of collision, grounding and other hazards to navigation.

\section{Instruments}
At least the following instruments and equipment shall be installed \cite{DNVGL2011}: 
\begin{multicols}{2}
	\begin{itemize}
		\item Navigation radar with radar
		\item Propulsion control
		\item Manual steering device
		\item Heading control
		\item Other related \ac{UID}s
		\item Electronic Chart Display Information System (ECDIS)
		\item Steering mode selector switch
		\item VHF unit
		\item Whistle and manoeuvring light push buttons
		\item Internal communication equipment
		\item Central alert management system
		\item General alarm control
		\item Window wiper and wash controls
		\item Control of dimmers for indicators and displays
		\item Propulsion
		\item Emergency stop machinery
		\item Gyrocompass selector switch
		\item Steering gear pumps
	\end{itemize}
\end{multicols}



What do regulations say about systems which should be on board

\section{Parameters}
Which information really comes from instruments at the bridge

\section{Usage}
Which parameters are relevant for the crew