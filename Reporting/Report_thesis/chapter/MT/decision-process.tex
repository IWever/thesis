\chapter{ Decision-making process}
\label{ch:decision-process}
A rule-based time-domain decision model is used, to acquire more insight into the decision-making process. By applying this model is analysed what the situation is, what problems might occur, and how to act to operate safely. A decision tree is used to analyse this in a structured way. Tags describe the nodes. This generalised model can be optimised in later stages to create a fully functional decision algorithm for autonomous and unmanned ships. The final algorithm is created using advanced modelling techniques and machine learning algorithms, as the decision algorithm becomes too complicated to draw decision trees by hand. First, the decision phases are described, followed by lists of nodes used to identify scenarios and situations. Combining this with the COLREGs will result in a functional decision model. Also, the criteria necessary to go through the steps are discussed. This decision model aims to gain insight into the decision process, and shows which situations are critical when the goal is to avoid communication.

\section{Decision phases}
The decision making process has different phases. As shown in chapter \ref{ch:model}. A simplified version of the decision model is shown in figure \ref{fig:decision-process}. These different phases are described in table \ref{tab:phases-description}.

\begin{figure}[hb]
	\centering
	\includegraphics[width=.8\textwidth]{decision-process.png}
	\caption{Decision process}
	\label{fig:decision-process}
\end{figure}

\begin{table}[H]
	\begin{tabular}{p{0.35\textwidth}|p{0.64\textwidth}}
		\toprule
		Class & Description\\
		\midrule
		Mental model & Acquire knowledge about the situation \\
		Situation identification & Identify the encountered situation, to determine which criteria are relevant, based on waterway lay-out and other ships. \\
		Predict future states & Predict if a problem will occur, as there is only a change in strategy needed when this is the case. \\
		Strategy & If there is a problem, a new strategy should be chosen, this is based on the evaluation of criteria. \\
		Action & From this strategy, different actions will follow. \\
		Result & Finally the result is evaluated, using the same criteria to determine a problem in future states. \\
		\bottomrule
	\end{tabular}
	
	\captionof{table}{Description of phases in decision process}
	\label{tab:phases-description}
\end{table}

\section{Nodes in decision-making tree}
Short keywords are used to describe the nodes within the decision tree. These describe in short what kind of situation, problem, strategy, action or result there is. This section gives definitions for those keywords.

\subsubsection{Identify situations}
Identification of encountered situations is the first step to limit the number of strategies which are relevant to evaluate. The nodes within the identification process are described in table \ref{tab:situations}, more details on how this is determined are described in section \ref{sec:situation-identification}.
\begin{table}[H]
	\begin{tabular}{p{0.35\textwidth}|p{0.64\textwidth}}
		\toprule
		Tag & Description\\
		\midrule
		Passing & The paths of both ships are in the opposite direction, and do not cross. \\
		Crossing & The final direction of both ships differ, but they do cross. \\
		Over-taking & The paths of both ships are the same but at different speeds. \\
		Merge & The starting direction of both ships differ, but the final direction is the same. \\
		\bottomrule
	\end{tabular}
	
	\captionof{table}{Tags for different situations}
	\label{tab:situations}
\end{table}

\subsubsection{Predict future states}
Different criteria are evaluated, to identify whether a problem will occur. These criteria are described in chapter~\ref{ch:criteria-problem}. In table~\ref{tab:identification-criteria} the nodes within the decision tree are discussed to evaluate if there is a problem, and which evaluations are possible for these criteria. 
\begin{table}[H]
	\begin{tabular}{p{0.35\textwidth}|p{0.64\textwidth}}
		\toprule
		Tag & Evaluation \\
		\midrule
		Closest point of approach & Good; Too close \\
		Crossing point & In front; Behind \\
		Crossing distance & Good; Too close \\
		Passing position & Port side; Starboard side \\
		Relative speed & Faster; Same speed; slower \\
		\bottomrule
	\end{tabular}
	
	\captionof{table}{Criteria and result of evaluation to identify a problem}
	\label{tab:identification-criteria}
\end{table}

\subsubsection{Possible strategies}
Using the identification of the situation and the prediction of future states. A limited number of strategies might be possible. The strategies will result in actions but can be categorised into groups. Table~\ref{tab:strategies} describes this and shows the possible actions per strategy. The distance and time that is left to avoid a problem is evaluated to determine if an action is necessary and possible. These criteria are described in table~\ref{tab:manoeuvre-criteria}.
\begin{table}[H]
	\begin{tabular}{p{0.4\textwidth}|p{0.59\textwidth}}
		\toprule
		Tag & Actions \\
		\midrule
		Follow planned path & Continue without change\\
		Increase CPA by yourself & Evasive manoeuvrer; Adjust speed \\
		Increase crossing distance by yourself & Evasive manoeuvrer; Adjust speed \\
		Work together with others & Communicate; Evasive manoeuvrer; Adjust speed \\
		Emergency & Emergency stop; Communicate \\
		\bottomrule
	\end{tabular}
	
	\captionof{table}{Tags for different strategies}
	\label{tab:strategies}
\end{table}

\begin{table}[H]
	\begin{tabular}{p{0.35\textwidth}|p{0.64\textwidth}}
		\toprule
		Tag & Evaluation \\
		\midrule
		Time till problem & ... seconds\\
		Distance till problem & ... meter \\
		\bottomrule
	\end{tabular}
	
	\captionof{table}{Criteria to determine if action is possible}
	\label{tab:manoeuvre-criteria}
\end{table}

\subsubsection{Possible actions}
From the selected strategy, actions will follow. These action are a combination of an action type, and an execution moment for the action. In table \ref{tab:actions} and \ref{tab:time-domain-action} these are described. These actions consist of different smaller sub-actions such as: "rudder 35 degrees to port-side". 
\begin{table}[H]
	\begin{tabular}{p{0.35\textwidth}|p{0.64\textwidth}}
		\toprule
		Tag & Description\\
		\midrule
		Continue without change & Do not change speed or rudder\\
		Evasive manoeuvrer & Steer to starboard or port side first and end at same course\\
		Adjust speed & Reduce speed or speed-up \\
		Emergency stop & Turn ship side-ways and set propulsion in reverse\\
		Communicate & Discuss required actions with other vessel(s)\\
		\bottomrule
	\end{tabular}
	
	\captionof{table}{Types of actions}
	\label{tab:actions}
\end{table}

\begin{table}[H]
	\begin{tabular}{p{0.35\textwidth}|p{0.64\textwidth}}
		\toprule
		Tag & Description\\
		\midrule
		Now & When action can be undertaken as soon as possible \\
		In ... minutes/seconds & Wait to ensure action is necessary \\
		After action ... & Wait with action till you or other has done another action \\
		\bottomrule
	\end{tabular}
	
	\captionof{table}{Time-domain for action}
	\label{tab:time-domain-action}
\end{table}

\subsubsection{Result}
Using the criteria from the action phase, the selected action can be evaluated. Herein is the human factor on board of other vessels taken into account, in the form of perceived risk. This perceived risk is linked to the safety domains \cite{Szlapczynski2017a}. The used safety domain at open sea is described by Coldwell \cite{Coldwell1983}. Figure~\ref{fig:coldwell-safety-domain} shows that this domain is based on an ellipse, that is not centred at the location of the vessel. This safety domain takes into account that ships rather pass on port-side and behind. In this model, the safety domain only depends on the length of the ship.
In busy areas, such as harbours and coastal area's, it is not always possible to use this safety domain. Based on expert reviews for the port of Rotterdam, is a closest point of approach used in these regions of 2 cables, which is equal to 370 meters.

\begin{figure}[p]
	\centering
	\includegraphics[width=.85\textwidth]{coldwell-safety-domain.png}
	\caption{Model for safety domain by Coldwell}
	\label{fig:coldwell-safety-domain}
\end{figure}

Evaluations and criteria are shown in table \ref{tab:criteria-safe-situation}. The evaluation of these criteria determines if problems can be avoided.

\begin{table}[H]
	\begin{tabular}{p{0.35\textwidth}|p{0.64\textwidth}}
		\toprule
		Tag & Evaluation\\
		\midrule
		CPA & Good; Too close\\
		Perceived risk & Safe; Uncomfortable; Close encounter; Too close\\
		Safe situation & Yes; Uncomfortable; No\\
		\bottomrule
	\end{tabular}
	
	\captionof{table}{Tags for safe situation criteria}
	\label{tab:criteria-safe-situation}
\end{table}


\section{Critical paths in decision trees}
When combining the nodes as described in the previous section, huge decision trees are applicable. The decision tree for specific situations and scenarios is much smaller than a tree covering all possible solutions. Based on the previously described nodes, a generalised model of the tree can be drawn. Figure \ref{fig:decision-tree-general} shows this model, where the orange blocks show evaluation criteria and the white blocks are choices. Thereby should be considered that the actions are similarly evaluated as the prediction of future states.

\emph{In the next chapters} will the decision tree be used to identify critical paths. These paths will be evaluated for critical situations. When all possible paths are considered, this will result in a massive decision tree and many critical situations. All these paths do not result in more insight into a solution to avoid communication. Using identification and evaluation criteria several common critical paths can be identified which are used to define manoeuvrability criteria.

\begin{figure}[p]
	\centering
	\includegraphics[width=.9\textwidth]{decision-tree-general.png}
	\caption{Generalized model for decision tree}
	\label{fig:decision-tree-general}
\end{figure}
