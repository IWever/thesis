\chapter{Decision making process}
\label{ch:decision-process}
Using a rule-based time-domain decision model, can a general decision process be defined to make insightful: What the situation is and what problem might occur. Nodes in the decision tree are described with keywords to work in a structured way. This white-box approach will cover as much as possible, however is it likely to use machine learning in the final design of a decision algorithm for autonomous and unmanned ships. First the decision phases are described. Followed by lists of nodes used to identify scenarios and situations. Combining this with the rules, will result in the decision model. Also the criteria necessary to go trough the steps are discussed. The aim of this decision model is gain insight in the decision process to show what critical situations are when you want to try to avoid communication.

\section{Decision phases}
The decision making process has different phases. This can be seen in the process diagram how these phases relate. This is shown in figure \todo{add process diagram}. These different phases are also described in table \ref{tab:phases-description}.
\begin{table}[H]
	\begin{tabular}{p{0.35\textwidth}|p{0.64\textwidth}}
		\toprule
		Class & Description\\
		\midrule
		Mental model & Acquire knowledge about the situation, not part of this research \\
		Situation identification & Identify the encountered situation, to determine which criteria are relevant, based on waterway lay-out and other ships. \\
		Predict future states & The second step is to determine if a problem will occur, as there is only a change in strategy needed when this is the case. \\
		Strategy & If there is a problem, a new strategy should be chosen, this is based on the evaluation of criteria. \\
		Action & From this strategy, different actions will follow. \\
		Result & Finally the result is evaluated, using the same criteria to determine a problem in future states. \\
		\bottomrule
	\end{tabular}
	
	\captionof{table}{Description of phases in decision process}
	\label{tab:phases-description}
\end{table}

\section{Nodes in decision making tree}
To describe the nodes within the decision trees, short keywords are used. These describe in short what kind of situation, problem, strategy, action or result there is. By using the same terms, confusion is avoided. Definitions for those terms are given in this section.

\subsubsection{Encountered situations}
This is the first step to limit the amount of strategies which are relevant to evaluate. The nodes within the identification process are described in table \ref{tab:situations}, more details on how this is determined is described in section \ref{sec:situation-identification}.
\begin{table}[H]
	\begin{tabular}{p{0.35\textwidth}|p{0.64\textwidth}}
		\toprule
		Tag & Description\\
		\midrule
		Passing & The paths of both ships are in opposite direction, and do not cross. \\
		Crossing & The final direction of both ships differs, but they do cross. \\
		Merge & The final direction of both ships is the same. \\
		Over-taking & The paths of both ships are the same but at different speeds. \\
		\bottomrule
	\end{tabular}
	
	\captionof{table}{Tags for different situations}
	\label{tab:situations}
\end{table}

\subsubsection{Predict future states}
To identify if a problem will occur, different criteria are being evaluated. These criteria are described in chapter \ref{ch:criteria-problem}. In table \ref{tab:identification-criteria} the nodes within the decision tree are discussed to evaluate if there is a problem, and what kind of problem there is. 
\begin{table}[H]
	\begin{tabular}{p{0.35\textwidth}|p{0.64\textwidth}}
		\toprule
		Tag & Evaluation \\
		\midrule
		Closest point of approach & Good; Too close \\
		Crossing point & In front; Behind \\
		Crossing distance & Good; Too close \\
		Relative speed & Faster; Same speed; slower \\
		\bottomrule
	\end{tabular}
	
	\captionof{table}{Criteria and result of evaluation to determine problem}
	\label{tab:identification-criteria}
\end{table}

\subsubsection{Possible strategies}
Using the identification of the situation and prediction of future states. A limited number of strategies might be possible. The strategies will result in actions, but can be categorized in the groups as described in table \ref{tab:strategies}.
\begin{table}[H]
	\begin{tabular}{p{0.35\textwidth}|p{0.64\textwidth}}
		\toprule
		Tag & Description\\
		\midrule
		Follow planned path & By doing planned actions \\
		Move away from other path & Change path, in order to increase CPA\\
		Stay parallel for longer & Change path, to sail parallel to other vessel \\
		Adjust speed & Slow-down or speed-up\\
		Abort over-taking & Lower speed and stay behind other vessel \\
		Move away from other position & Check if this is to starboard, otherwise communicate \\
		Communicate & If other (also) has to take action to avoid a problem \\
		\bottomrule
	\end{tabular}
	
	\captionof{table}{Tags for different strategies}
	\label{tab:strategies}
\end{table}

\subsubsection{Possible actions}
From the chosen strategy, different actions will follow. These action are a combination of an action type, and a moment to execute action. In table \ref{tab:actions} and \ref{tab:time-domain-action} these are described. The criteria to determine if an action is possible are described in \ref{tab:manouvre-criteria}.
\begin{table}[H]
	\begin{tabular}{p{0.35\textwidth}|p{0.64\textwidth}}
		\toprule
		Tag & Description\\
		\midrule
		Continue without change & Do not change speed or rudder\\
		Evasive manoeuvrer to starboard & Steer starboard first and end at same course\\
		Evasive manoeuvrer to portside & Steer to portside first and end at same course\\
		Slow down & Reduce speed\\
		Speed-up & Speed-up \\
		Emergency stop & Turn ship side-ways and set propulsion in reverse\\
		Communicate & Discuss required actions with other vessel(s)\\
		\bottomrule
	\end{tabular}
	
	\captionof{table}{Types of actions}
	\label{tab:actions}
\end{table}

\begin{table}[H]
	\begin{tabular}{p{0.35\textwidth}|p{0.64\textwidth}}
		\toprule
		Tag & Description\\
		\midrule
		Now & When action can be undertaken as soon as possible \\
		In ... minutes/seconds & Wait for a period of time to ensure bold movement \\
		After action ... & Wait with action till you or other has done another action \\
		\bottomrule
	\end{tabular}
	
	\captionof{table}{Time-domain for action}
	\label{tab:time-domain-action}
\end{table}

\begin{table}[H]
	\begin{tabular}{p{0.35\textwidth}|p{0.64\textwidth}}
		\toprule
		Tag & Evaluation \\
		\midrule
		Closest point of approach & Good; Too close\\
		Time till problem & ... minutes/seconds\\
		Distance till problem & ... meter \\
		\bottomrule
	\end{tabular}
	
	\captionof{table}{Criteria to determine if action is possible}
	\label{tab:manouvre-criteria}
\end{table}

\subsubsection{Result}
Using the evaluations from the action phase, can the eventual solution be evaluated. Herein is the human factor on board of other vessels taken into account, in the form of perceived risk. Evaluations and criteria are shown in table \ref{tab:criteria-safe-situation}.
\begin{table}[H]
	\begin{tabular}{p{0.35\textwidth}|p{0.64\textwidth}}
		\toprule
		Tag & Evaluation\\
		\midrule
		CPA & Good; Too close\\
		Perceived risk & Safe; Uncomfortable; Close encounter; Too close\\
		Safe situation & Yes; Uncomfortable; No\\
		Communicate & Yes; No\\
		\bottomrule
	\end{tabular}
	
	\captionof{table}{Tags for safe situation criteria}
	\label{tab:criteria-safe-situation}
\end{table}


\section{Critical paths in decision trees}
When combing the nodes as described in the previous section, very large decision trees are obtained. For specific situations and scenarios the eventual decision tree is much smaller than a tree covering all possibilities. 
Therefore the decision trees are split up per phase. Examples of these trees can be found in appendix \ref{app:decision-trees}. Several common critical paths within those trees can be identified. These paths will be used in later chapters to evaluate criteria, manoeuvring requirements and eventually do a case study.






