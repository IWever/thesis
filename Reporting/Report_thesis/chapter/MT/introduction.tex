The previous part described the context and the steps towards autonomous shipping. This part will focus on the limits due to manoeuvrability in the decision making process. However, the steps taken to get to the right decision are similar, as shown in the model from chapter \ref{ch:model}. The insights acquired from the decision process can be used to determine manoeuvrability requirements.

The first phase of observation starts with updating the mental model and is followed by a phase where different chunks of information are connected. This means situations and scenarios can be identified. This is the start of the decision making process in which different trees are used to identify potential hazards and problems which will result in strategies. These strategies are finally narrowed down to the actions. The different nodes and trees are discussed in chapter~\ref{ch:decision-process}. 

More detailed descriptions of the different phases within the decision making process are discussed in the next chapters. First the identification of the situation and scenario is discussed in chapter~\ref{ch:identify-situation}.
Which branch to take in the decision trees, is determined by evaluating criteria. These criteria determine if there are hazards and which manoeuvers are feasible. This works in two ways, while sailing this can be used to determine the right strategy. While the designer can use these criteria to ensure a ship can sail safely in specific situations. The criteria to evaluate what kind of problem there is, are described in chapter~\ref{ch:criteria-problem}. The criteria used to evaluate if strategies are feasible for critical situations are described in chapter~\ref{ch:criteria-manouvre}. In chapter~\ref{ch:criteria-manouvre} is also described how designers can use these manoeuvers to determine the manoeuvring requirements for safe operation. To evaluate if these criteria are useful, manoeuvers are simulated with the tool as described in appendix~\ref{app:tool} is used. These are also used to run several scenarios and see how these criteria affect the decision making process.

The result of this part is a design matrix which can be used by designers to determine the manoeuvring requirements and moment when decisions should be made. This is done for some specific manoeuvers, showing the minimal time and distance needed to make decisions in order to have a safe distance between vessels. This depends for example on the manoeuvring characteristics, speed and type of manoeuvrer.