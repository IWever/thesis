The previous part described the context and the steps needed towards autonomous shipping. This chapter will focus on the decision making process. This is currently done by captains, on autonomous ships this should be automated. To accomplish this, criteria must be known on which these decisions can be based. 

To determine the criteria, a ship will have to identify the situation and specific scenario first. These determine which criteria are relevant. For different possible strategies these criteria are evaluated. Based on this evaluation, the right strategy can be chosen.

To figure out which criteria are relevant, first two scenario for a standard situation are analysed step-by-step. This means gathering information on for example the location, actors and rules which apply. This will result in a decision tree with criteria, based on the distance and speed between vessels these criteria can be evaluated and be determined which are most relevant. 
This will result in a lists for possible locations, actors, scenarios and decisions. Collecting these and repeating the process will result in a database. This database can be used to develop a rule-based time-domain decision model. 

This model is consecutively implemented into a tool to test if ships act as expected. By more varying the scenario's, gaps in the rule-based model can be found. Some might be possible to tackle by extending the database with more strategies or criteria. When this is not possible, ships will have to communicate to come-up with a solution, this will not be part of this research.

This part will also include the design of the tool which is used to evaluate the model and test new strategies to improve navigational safety. The tool will include different modules to make sure a realistic simulation can be done. These modules contain for example a manoeuvring model and communication protocol.

