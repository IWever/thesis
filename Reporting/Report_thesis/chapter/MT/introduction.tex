The previous part described the context and the steps needed towards autonomous shipping. This chapter will focus on the decision making process. This is currently done by captains, on autonomous ships this should be automated. To accomplish this, criteria must be known on which these decisions can be based. 

To determine the criteria, a ship will have to identify the situation and specific scenario first. These determine which criteria are relevant. For different possible strategies these criteria are evaluated. Based on this evaluation, the right strategy can be chosen. A model in which all this is incorporated is John Boyd's OODA loop \cite{Boyd1987}. This model is used in many similar cases, but mostly in military command and control situations \cite{Arciszewski2009} \cite{Kalloniatis2017}.

The first steps within this model are to observe, orient, decide and act. Within this part a description is given how this applies to choosing the right strategy in common maritime encounters. The first step describes what can be observed in order to identify the situation. In the orientation phase the situation is evaluated using different criteria. This will result in a decision on the right strategy. To form the decision model using the identification and evaluation, first there will be case studies, where all steps in the decision making process are discussed. Which results in a rule-based time-domain decision model. 

This model is consecutively implemented into a tool to test if ships act as expected. By more varying the scenario's, gaps in the rule-based model can be found. Some might be possible to tackle by extending the database with more strategies or criteria. When this is not possible, ships will have to communicate to find the right solution. What should be communicated is not part of this research, how this is done will be discussed in part \ref{part:CS}.

This part will also include the design of the tool which is used to evaluate the model and test new strategies to improve navigational safety. The tool will include different modules to make sure a realistic simulation can be done. These modules contain for example a manoeuvring model and traffic controller.

\url{https://www.artofmanliness.com/articles/ooda-loop/} \todo{add picture of OODA loop}