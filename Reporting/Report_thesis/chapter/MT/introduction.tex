\markboth{PART II}{PART II}

The previous part describes the context of this research and showed steps towards autonomous shipping, which are relevant to this research. This part will focus on the effect of manoeuvrability on the decision-making process. The steps taken to get to the right decision are similar to steps shown in the model of chapter \ref{ch:model}. The insights acquired from the effect of the manoeuvrability on the decision process can be used to determine manoeuvrability requirements and answer the following question:

\begin{quotation}
	\emph{How do ship manoeuvrability characteristics influence the domain for decision making, to ensure that the chosen strategy will result in a closest point of approach that does not require communication?} 
\end{quotation}

The first phase of the decision process is the observation phase. This phase starts by updating the mental model and is followed by a phase where different chunks of information are connected. An operator or system uses this to identify situations and scenarios. This identification is the start of the decision making process in which different trees are used to identify potential hazards and problems that will result in strategies. These strategies are finally narrowed down to the actions. The various nodes and trees are discussed in chapter~\ref{ch:decision-process}. 

The next chapter discusses more detailed descriptions of the consecutive phases in the decision-making process. First, the identification of the situation and scenario are addressed in chapter~\ref{ch:identify-situation}.
Evaluating criteria determines which branch to follow in the decision trees. These criteria define if there are hazards and which manoeuvres are feasible, that works in two ways: An operator can use these criteria during operation to determine the right strategy, while a ship designer can use these criteria to ensure a ship can navigate safely in specific situations. The criteria to evaluate what kind of problem there is, are described in chapter~\ref{ch:criteria-problem}. The criteria used to assess if strategies are feasible for critical situations are described in chapter~\ref{ch:criteria-manouvre}. In chapter~\ref{ch:criteria-manouvre} is also described how designers can use these manoeuvres to determine the manoeuvring requirements for safe operation. Manoeuvres are simulated to evaluate if these criteria are useful. With the tool as described in appendix~\ref{app:tool}. These are also used to verify several scenarios and see how these criteria influence the decision-making process.

The result of this part is an overview, that can be used by designers to determine the effect of manoeuvring capabilities at the moment decisions have to be made, which is done for some common critical manoeuvres, showing the minimal time and distance required to make decisions to have a safe distance between vessels. This matrix depends on the manoeuvring characteristics, speed and type of manoeuvre.