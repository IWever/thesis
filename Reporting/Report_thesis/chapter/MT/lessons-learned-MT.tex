\chapter{Evaluation of test results}
In the previous chapters are steps taken to identify common critical situations, and show how ships with different manoeuvring characteristics act in those situations. To avoid communication is it key to have enough clearance, this can be acquired by making the right action well in advance. In critical situation this is however not always possible. When a situation becomes critical, depends on the manoeuvrability of the vessel. In section~\ref{sec:relation-advance-distance} is shown what the relation is between the possible passing distance and distance before initial CPA for different advance distances. By increasing the manoeuvrability of an unmanned vessel is it possible to reduce the number of critical situation and therewith the amount of communication, as a safe passing distance is still acquired. In order to achieve this, does this chapter show how the passing distance can be improved in general by changing manoeuvrability characteristics, and how this should be taken into account in the design or decision process of unmanned ships.

\section{General relation for critical evasive manoeuvrer}
In chapter~\ref{ch:criteria-manouvre} are results shown for the relation between passing distance and distance till CPA for different start speeds and vessels. There is a direct relation between start speed and the advance distance. The manoeuvring characteristics of a vessel are a second input for the advance distance. By both varying the start speed and manoeuvring characteristics, is a general image acquired which shows the relation between passing distance and distance till CPA for different for different advance distances.The general curve for this is shown in figure~\ref{fig:general-advance} and can be extracted from the plot as shown in figure~\ref{fig:passing-distance-advance}.

Going left or right on the curve relates to the maximum course change. The position of the curve depends on two factors, the advance distance and speed of another vessel. Where the speed of another vessel has a relatively small impact, as the bandwith of these curves is small. By comparing different curves, is seen that an increase of 1 knot, will result in a maximum change of 1 percent of the distance till \ac{CPA}, when the passing distance is kept constant. This means for a 400 meter container ship, that for a passing distance of 1000 meter, the distance till initial CPA varies between 960 and 1070, when the speed of the other vessel is between 0 and 18 knots.

The advance distance is however much more relevant for increasing the ability of unmanned vessels to ensure safe operation in critical situations. In case of an evasive manoeuvrer does a reduction of 1 percent of the advance distance, already result in a 1 percent reduction of the distance till initial CPA.
A reduction of 10 percent in advance distance can already be acquired by using different rudder profiles than commonly used profiles, as shown by Liu \cite{Liu2015a}. Similar results are acquired by using a high-lift rudder \cite{Zaky2018}. For a 250 meter tanker at 12 knots, with a desired passing distance of 1000 meter, does this mean a reduction of 100 meters on the distance till initial CPA was reached.

\section{Use case}
\todo{Wanneer is resultaat toepasbaar, gebruik case uit CS deel}


\chapter{Conclusion}
The model developed in this part will start with a sensory information on a situation, it will classify this situation and make a decision based on the evaluation of criteria. The result is an advise to execute a specific strategy, or if the model can't evaluate criteria sufficiently a subset of possible strategies and the remark it needs more information to narrow it down further.

\begin{quotation}
	\emph{How do ship manoeuvrability characteristics influence the domain for decision making, to ensure that the chosen strategy will result in a passing distance which does not require communication?} 
\end{quotation}
\todo{Answer research question}