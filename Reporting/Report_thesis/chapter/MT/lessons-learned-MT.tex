\chapter{Conclusion}
This part describes the steps to show the impact of manoeuvrability on the necessity to communicate. The first was an analysis of the decision making process. This analysis made the selection of critical situations and relevant criteria possible.
A decision tree is defined to determine the favourable strategy in those critical situations. The crossing situation is one of the most critical situations. The favourable strategy in this situation is an evasive manoeuvre. The method to determine the relation between manoeuvring characteristics and criteria is evaluated with the evasive manoeuvre.
This is used to answer the following question:

\begin{quotation}
	\emph{How do ship manoeuvrability characteristics influence the domain for decision making, to ensure that the chosen strategy will result in a closest point of approach that does not require communication?} 
\end{quotation}

For a specific situation which requires an evasive manoeuvre is determined that the key manoeuvring characteristics is the advance distance, which is tested during the turning circle test. A relation is found between this advance distance and decision criteria. The relevant decision criteria during a common critical evasive manoeuvre are the \acf{CPA} and passing distance. When these criteria are met, no communication is needed.

The discovered relations show that in case the required \ac{CPA} is 370 meter, the distance till initial CPA should be at least 835 meter. One of the conclusions that can be drawn from this is that the distance between two traffic lanes should be at least 806 meters to ensure that crossing ships have enough space to react, and can avoid communication.
The maximum advance distance is in this case is 400 meter. Longer advance distances will result in longer distances till initial CPA and thus will need even more space between traffic lanes. The advance distance can be improved by adding rudders or changing the rudder profile. One of these changes could already result in an improvement of 10\% on the advance distance, which results in a reduction of 10\% on the distance required to react. But this distance cannot be smaller than the mentioned 806 meters, when only using the evasive manoeuvre. By sailing parallel this distance can be increased, but this means it is no crossing situation anymore.

The method used to validate if a ship with certain manoeuvring characteristics is able to act safely and avoid communication, can also be used for other situations. More situations are required to define manoeuvring requirements for any ship in any situation. This will ensure that the chosen strategy will result in a closest point of approach which does not require communication.