\chapter{Conclusion}
In this part different steps are taken to show the influence of manoeuvring characteristics on the decision making process, to ensure safe operation. The first step was to gain insight into the decision making process in different situations. Thereby are criteria considered which are used to make these decisions.
Based on this analysis are different manoeuvres described and is determined that the evasive manoeuvre is both common and critical. Thus this manoeuvre is very suitable to test the method which uses a simulation environment to determine what the effect is of different inputs on the decision process. 
This is used to answer the following question:

\begin{quotation}
	\emph{How do ship manoeuvrability characteristics influence the domain for decision making, to ensure that the chosen strategy will result in a closest point of approach which does not require communication?} 
\end{quotation}

The key manoeuvring characteristics is the advance distance, which is tested during the turning circle test. A relation is found between this advance distance and decision criteria. The relevant decision criteria during a common critical evasive manoeuvre are the \acf{CPA} and passing distance. When ships do meet the criteria, no communication is needed.

The found relations shows that in case the \ac{CPA} is 370 meter, the distance till initial CPA should be at least 806 meter. One of the conclusions which can be drawn from this is that the distance between two traffic lanes should be at least 806 meters to ensure that crossing ships have enough space to react, and can avoid communication.
The maximum advance distance in this case is 400 meter. Higher advance distances will result in higher distances till initial CPA and thus will need even more space between traffic lanes. The advance distance can be improved by adding rudders or changing the rudder profile. One of these changes could already result in an improvement of 10\% on the advance distance, which results in a reduction of 10\% on the distance required to react, if it is higher than the mentioned 806 meters. 

The method used to validate if a ship with certain manoeuvring characteristics is able to act safely and avoid communication can also be used for different situations. Creating a set of these situations will make it possible to evaluate for any ship. If they ensure that the chosen strategy will result in an approach distance which does not require communication. Starboard-starboard passing while turning and the moment when two ships merge are examples for these situations. 
More research and mostly number crunching are thereby needed to determine what an acceptable \ac{CPA} and passing distance should be to avoid communication.

\todo{More clear on a level, this goes well, this doesn't, this is doubtful. Translating numbers into general situations in practice}