\chapter{Identification of situation and scenarios}
\label{ch:identify-situation}
The first relevant step for this research is the step to identify the situation and scenario. This identification is used to narrow down possible strategies in the next phases of the decision making process. In this chapter, the first steps of classifying the situation are discussed. Followed by the steps taken in the orientation phase to form strategies.

\section{Situation identification}
\label{sec:situation-identification}
Based on the observations different situations can be classified into four types, the different types are described below: 
\begin{description}
	\item [Passing] Ships do get close, but the paths are not crossing
	\item [Crossing] Paths of ships are crossing
	\item [Merge] Two ships from different directions, heading in the same direction, strategy might lead to an over-taking situation.
	\item [Over-taking] Two ships following the same path with different speeds.
\end{description}
It depends on the type of waterway which situation is likely. To determine this, a classification of paths is used. To do this systematically paths will be based on figure \ref{fig:junction-letters} and can be written as: 
\begin{quote}
	[current position, direction]
\end{quote}
Combining paths of two ships will describe a situation. Using the system to describe paths. The situations can be written down as:
\begin{quote}
	(path own ship $|$ path other ship).
\end{quote}

To classify a situation where two vessels encounter each other, the paths are considered. Key is to determine the angle between those paths. This way it is possible to classify them using table \ref{tab:scenarios-standard}.
The boundaries are based on \ac{COLREGs} and shown in figure \ref{fig:junction-letters}. \todo{use colreg boundaries in junction and terminology (head-on, overtaking, etc.)}.

\begin{minipage}{\textwidth}
	\begin{minipage}[b]{0.69\textwidth}
		\centering
		\begin{tabular}{l|l|l}
			Own ship & Other ships & Situation\\
			\hline
			\big[A,D\big] & [D,C] [D,B] [D,A] [C,B] & Passing \\
						& [C,A] [B,A] [B,C] & \\
			\big[A,C\big] & [C,A] [C,B] [B,A] & Passing \\
			\big[A,B\big] & [D,C] [C,D] [B,A] & Passing \\
			\big[A,C\big] & [D,B] [D,A] [C,D] [B,D] & Crossing \\
			\big[A,B\big] & [D,A] [C,A] [B,D] [B,D] & Crossing \\
			\big[A,D\big] & [C,D] [B,D] & Merge \\
			\big[A,C\big] & [D,C] [B,C] & Merge \\
			\big[A,B\big] & [D,B] [C,B] & Merge \\
			\big[A,D\big] & [A,D] & Over-taking \\
			\big[A,C\big] & [A,C] & Over-taking \\
			\big[A,B\big] & [A,B] & Over-taking \\
		\end{tabular}
		\captionof{table}{Standardized paths for situations}
		\label{tab:scenarios-standard}
	\end{minipage}
	\hfill
	\begin{minipage}[b]{0.3\textwidth}
		\centering
		\includegraphics[width=\textwidth]{junction-letters.png}
		\captionof{figure}{Path description}
		\label{fig:junction-letters}
	\end{minipage}
\end{minipage}

\section{Situations which limit possible strategies}
Beside this first identification of the situation. More details will be taken into account to form the right strategies. Below different factors and their consequences on the strategy are discussed.

\subsection{Type of waterway}
To determine which strategies can be chosen. The type of waterway is the first to consider. As this might restrict the area where can be sailed, which influences the possible strategies. For example is it common to over-take ships in open-water on starboard side. While on restricted waterways ships will sail as far as possible to starboard already. This means the ship which is over-taking will have to pass on the port side of the other vessel, at the center of the waterway.

In the next step other static hazards are considered to check if the chosen strategy does not lead to a collision. Or if there are specific regulation frameworks for this waterway. These are however not part of the first iteration of the decision model, as this will introduce much more complexity, without improving the result in most cases. Examples of static hazards which could be evaluated in future iterations are bridges, buoys, forbidden zones and port mouth. As possibilities for over-taking are limited in those cases for example. This means the strategies are limited.

Another limiting factor related to waterways are the difference in regulations between waterways. Most obvious are signs which forbid to over-take or meet. But others are for example to not create wash or no turning. Or more directive signs on obligated directions or speed limits. This is mostly relevant for coastal and inland waterways.

\todo{add definition of static hazard and more examples, to make link to evaluation tool and tags}
\todo{also describe the rules, which might be based on signs}

\subsection{Actors}
The second major step is the identification of dynamic objects. Those are all relevant moving objects. Most obvious are off course other ships which do come close. But in future developments of the decision model, objects which are not under any control of a human should be considered, such as floating containers.

To predict the path, first the general information about the object should be acquired. Such as manoeuvrability, speed, course, type of object, under control, etc.
Thereby might it be possible in future developments to take into account the human factor to improve the path prediction. This could be based on the experience of the crew, availability of a pilot or if the vessel is completely unmanned.

Examples of such dynamic objects which limit the possible strategies are for example fishery vessels. As they might have long nets behind them while in operation. Ferries in inland waters which have priority over other shipping traffic. Other ships with limited manoeuvrability or forbidden zones around them. 
\todo{add more examples to describe consequence on strategy - Relative speed}

\section{Scenarios}
Using the information on type of waterway, location and actors, the scenario can be identified. Based on the scenarios, can be determined which rules to apply and what their implications are on the possible strategies. The same goes for the estimated path of dynamic objects. This might also narrow down the possible strategies.

Using the above mentioned information in the decision model, the strategies can be narrowed down. This can be used to simplify the decision tree and select the right criteria to evaluate.

\subsection{COLREGs}
How is the path based solely based on COLREGs, without taking into account other ships

Most important rule of \ac{COLREGs}, is that you are allowed to deviate from any rule if it will increase the safe operation. This is only the case when others also expect you to deviate.

use this \url{https://www.myseatime.com/blog/detail/8-colreg-rules-every-navigating-officer-must-understand}

\todo{add more examples to describe consequence on strategy}

