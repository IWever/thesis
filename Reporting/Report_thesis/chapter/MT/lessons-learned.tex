\chapter{Result}
Summarize results of verification and validation and if it can be used in the next step. Thereby also explaining which strategies could be communicated to pilots, seafarers and traffic controllers. To make shipping safer. But on the other hand also show where it lacks support.

\section{Improvement CPA using manoeuvrability characteristics}

\section{Examples of resulting design matrix}
Proefvaarten met schepen om draaicirkel en zigzag te bepalen.\\
Van schepen is bekend hoeveel tijd en afstand kan worden gewonnen met welke manouvre.\\
Dit combineren kan worden gebruikt tijdens scheepsontwerp.\\

\section{Factors not taken into account}
Germans for example do sometimes alter the off-set of AIS to avoid confusing with radar image. As captains should use radar instead of AIS information to navigate. AIS is not reliable enough, but with small errors captains still do it, so giving it a large off-set will force captains to use radar.

other factors equipment, flagstate, origin of crew, etc.

\section{Lessons learned}
The model developed in this part will start with a sensory information on a situation, it will classify this situation and make a decision based on the evaluation of criteria. The result is an advise to execute a specific strategy, or if the model can't evaluate criteria sufficiently a subset of possible strategies and the remark it needs more information to narrow it down further.

\begin{quotation}
	\emph{How do ship manoeuvrability characteristics influence the domain for decision making, to ensure that the chosen strategy will result in a passing distance which does not require communication?} 
\end{quotation}