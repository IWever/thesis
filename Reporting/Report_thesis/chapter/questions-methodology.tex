\chapter{Steps to be taken}
The plans mentioned in section \ref{sec:future} do all have similar challenges which must be solved before autonomous vessels will set sail. One of these issues is related to the current problem already existing within the maritime industry. Situational awareness of the crew. This means getting the right information, understanding the information and making the right decision based on this. In case of navigation this often means understanding the intentions of other ships by sharing the right information.

\section{Specific step on roadmap towards autonomous}
Even before the step where manned and unmanned vessels sail together. It is already important to develop a model which can predict the behaviour of other vessels. As this can be used to improved the warning system within current ECDIS systems. 

\section{Research questions}
The research will be done from two different viewpoints. Where the Maritime part will focus on ship characteristics and how with the right navigational strategy radio communication can be avoided. The part for computer science will focus explicitly on this communication. More specifically which information must be shared to improve situational awareness.

\subsection{Maritime Technology}
Critical situations are moments during a voyage where it is most important that is known what the intentions are of other vessels. But due to the chaotic situation this is not always possible. Supporting the crew in showing possible intentions of other vessels will help to create situational awareness faster. 

But this is not the same for different vessels, as a long heavy ship will mostly go straight ahead at a similar speed, while a small tug boat might go all over the place. Thereby is there the impact of traffic separation schemes and harbor entrances on the likelihood of manoeuvring in a certain direction. 
By adjusting the course in an early stage the intentions can be made clear, without the need for communication. This will lead to an unimpeded voyage. Beside the advantage of less pressure on the crew, is it also more easy to have autonomous vessels sailing between manned vessels.
The moment these intentions have to communicated is highly dependent on the type of vessel or ship characteristics. During an unimpeded voyage it is not needed to communicate as intentions are in all cases clear. This has led to the following research question:

\begin{quotation}
	\Large
	\emph{"How do ship characteristics influence the time-domain for decision making to ensure an unimpeded voyage?"} 
\end{quotation}

\subsection{Computer Science}
Many people are convinced one of the main developments within the maritime industry will be autonomous shipping. An argument is the improved safety for seafarers, as they don't have to be on board. However this does not necessarily go for all other vessels around the autonomous ship. This is also where one of the main arguments against autonomous shipping come from. How do other (manned) vessels know the intentions of autonomous vessels and can be sure that they will not make unexpected movements?

Currently this is secured in two ways. First and foremost are the COLREGs, rules applicable to all vessels, as these rules are concrete these can be programmed and used. Examples are to stay on starboard side of the shipping lane and to not cross other ships with small relative angle. However in critical situations such as the entering of harbors or in busy parts of the world, the VHF radio is used to ensure that intentions are clear.

To make autonomous shipping possible, autonomous vessels should know how to communicate their intentions, without overloading the VHF and AIS channels. An optimization of the communication must be done, where others vessels know enough about the intentions to adapt their path to it, without overloading communication channels.
This leads to the following research question:

\begin{quotation}
	\Large
	\emph{"How to optimize the communication between vessels, to support the decision making by the officer of watch?"}
\end{quotation}

\section{Methodology}
The first step is to get insight into current solutions and projects related to improving situational awareness within the future. Many of these projects are also steps towards autonomous or unmanned sailing. Based on this, a philosophy can be developed on how to predict the behavior and intentions of other vessels. This leads to a high level abstract model which will be the basis to get insight in the information relevant to develop an implementation of a tool. This tool will initially be used to simulate the behavior, thereby validating if predefined strategies are also logical choices to avoid critical situations in specific scenario's. The scenario's and strategies will eventually be validated together with seafarers.