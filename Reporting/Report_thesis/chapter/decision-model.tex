% Which parts should be solved to use the new technology
\chapter{Decision model}
\label{ch:model}
In the previous chapter, several projects are discussed, which gave insight in the challenges towards unmanned and autonomous shipping. Using a model for decision making, this challenge can be tackled in a structured way. This model shows which factors influence the decision making process and how this is supported by relevant research.

The decision process within the model is based on Boyd's OODA loop, Endsley model for situational awareness, and combined with models used in the projects as mentioned in chapter~\ref{ch:future}. 
The OODA loop has different phases: Observe, orient, decide and act. Similar to Endsley's model: Perceive, comprehend, project, decide and act. 
The combined model describes how this applies to choosing the right strategy for safe operation and relates to external factors and relevant theory. In figure \ref{fig:decision-model} a visual of the decision model is shown. The first step describes what can be observed, to form a mental model. The next step is to orient, in which the situation and hazards are identified. This will result in a set of strategies, which will be evaluated using different criteria, resulting in a decision. After this decision, an action is executed. This chapter will discuss these phases in more detail and how they relate to external factors and relevant theory.

\begin{figure}[hbp]
	\centering
	\includegraphics[width=\textwidth]{decision-model.png}
	\caption{Decision model}
	\label{fig:decision-model}
\end{figure}

\section{Decision process}
The decision process is the core of this model. The decision process will be used in part \ref{part:MT} on a less abstract level than described in this chapter. The steps taken in this decision process are similar for manned and unmanned vessels. Although their talents differ for being consistent or handling exceptions.

\subsubsection{Mental model}
The first phase of the decision process is to form a mental model. A mental model is a representation of the surrounding world, including the relationships between its various parts and a person's intuitive perception about his or her own acts and their consequences.
To make this representation sensor data about the environment is used. This raw data must be interpreted, to become information which can be combined into knowledge. 
These steps require still much research, although large steps are taken within the domains of LIDAR \cite{Cameron2017}, computer vision \cite{Marr2017} and sensor fusion \cite{Hoffman2018}. In appendix \ref{app:systems}, the systems which are used currently at manned vessel to form a correct mental model are discussed. For this research only the result of this step is relevant: Is the acquired knowledge sufficient to identify the situation correctly, or is more information desired via communication?
The sensors which in the future will be used is not within the scope of this research, nor how that is combined into information. 

\subsubsection{Identify situation}
The step from information to knowledge, is in the phase where the situation, scenario and hazards are identified. How this would go in practice is discussed in chapter \ref{ch:identify-situation}. This step is taken to identify critical situations which should be evaluated during the design phase of an autonomous ship. 
This research will define a method to evaluate these critical situations. The situation and scenario is defined by the lay-out of the waterway, other nearby vessels, relevant regulations, etc.

\subsubsection{Predict future states and decide on strategy}
Based on the situation different strategies are possible. These strategies have to be evaluated. This done by predicting how the different strategies will influence the path of the different vessels. A trade-off must be made between exact calculations and computation time. For example is the \acf{CPA} currently determined using linearized formulas in common \ac{ARPA} systems. Using non-linearized methods with for example a Bézier curve, will already result in a smaller error. Simulations would improve this even more, however do simulation with correct hydrodynamic models cost much more computational time. In chapter \ref{ch:criteria-problem} the linearized and non-linearized methods are described. The simulation method can be done with the tool as described in appendix \ref{app:tool}, but this tool is not optimized for such calculations, and will therefore not be able to do these calculations in real-time. The hydrodynamic model used in the simulation also described in appendix \ref{apps:hydro-model}. After this phase different manoeuvers are evaluated which corresponds to the different strategies. How this will be done for common critical strategies is discussed in chapter \ref{ch:criteria-manouvre}, this will result in manoeuvring requirements, these requirements can be used by ship designers to ensure the ship can operate safely with minimal need for communication.
After the evaluation of these criteria is known which strategy will result in safe operation of the vessel.

\section{External factors}
How easy it is to go trough the decision process and end-up at the right strategy depends on the situations. This situations is mostly influenced by the environment. In some cases this environment is too complex to make a mental model for, just based on static sensors and should information be retrieved by request. This section will describe in more detail how the environment influences the forming of the mental model and how safe operation within this environment would benefit from communication. 

\subsection{Environment}
As discussed before is the mental model mostly a representation of the environment in which the vessel acts. The sensors will measure this environment. Many critical situations occurred due to weather conditions. The reason is that the sight is limited during heavy rain or snow. Also might the wind and waves limit the manoeuvring capabilities of a vessel. This is also the reason some vessels are not allowed to enter a port when winds are too high.
This is mostly due to the layout of the waterway. Due to currents, operations (e.g. maintenance and fishery) or limited depth, might some area's be restricted. This information is often acquired via communication channels, but communication channels which only allow receiving and not sending, such as \acf{Navtex}.
The same goes for basic information on other ships. They might send their location and speed via \ac{AIS}, but still key is the \ac{ARPA}. Due to weather conditions these systems could have worse performance, as heavy rain creates noise at the radar. In the situations where sensors do not function as expected. Communication is needed, even if whole decision process itself is optimized to avoid communication.
The same goes for communication with shore based stations such as traffic controllers or in the future remote pilots. These often have information which could not be retrieved via sensors or current systems. Such as a place and time to berth or pick-up a pilot.

\subsection{Communication}
As described are there still cases in which communication is necessary. In part \ref{part:CS} the case where communication between manned and unmanned vessel is needed is discussed. Using the \acf{sCE} method a protocol is defined, based on existing systems and protocols. Thus using \ac{AIS} to send written messages, or \ac{VHF} and \ac{SMCP} for verbal messages. 
Other cases such as communication with traffic controllers and pilots could use the same protocol. Although they might need to share more information with unmanned vessels, which could be done with a new system such as \ac{VDES}. This will however not be part of this research.
