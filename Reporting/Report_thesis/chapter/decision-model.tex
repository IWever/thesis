\chapter{Decision model for safe operation}
\label{ch:model}
In the previous chapter, several projects are discussed, which gave insight in the challenges towards unmanned and autonomous shipping. To gain insight into the challenge of communication is a decision model used. This model shows which factors influence the decision-making process and how relevant research support this. The steps in this decision process are the same for manned and unmanned ships.

The decision process within the model is based on Boyd's OODA loop \cite{Boyd1987}, Endsley model for situational awareness \cite{Endsley2013}, and combined with models used in the projects as mentioned in chapter~\ref{ch:future}. 
The OODA loop has different phases: Observe, orient, decide and act. Similar to Endsley's model: Perceive, comprehend, project, decide and act. 
The combined model describes how this applies to, choosing the right strategy for safe operation, and relates to external factors and relevant theory. In figure \ref{fig:decision-model} a visual of the decision model is shown. The first step describes what can be observed, to form a mental model. The next step is to orient, to identify the situation and hazards. This step will result in a set of strategies, which will be evaluated using different criteria, resulting in a decision. After this decision, the operator administers an action. This chapter will discuss these phases in more detail and how they relate to the challenge of communication, by discussing external factors and relevant theory.

\begin{figure}[p]
	\centering
	\includegraphics[width=\textwidth]{decision-model.png}
	\caption{Decision model}
	\label{fig:decision-model}
\end{figure}

\section{Decision process}
The decision process is the core of this model. Part \ref{part:MT} will use the decision process on a less abstract level than described in this chapter. The steps taken in this decision process are similar for manned and unmanned vessels. Their way of thinking differs however when this is related to being consistent or handling exceptions.

\subsubsection{Mental model}
The first phase of the decision process is to form a mental model. A mental model is a representation of the surrounding world, including the relationships between its various parts and a person's intuitive perception about his or her acts and their consequences.
Sensor data about the environment is used to make this representation. Systems interpret raw data, transforming it into information, which can be combined into knowledge. 
These steps require still much research, although large steps are taken within the domains of LIDAR \cite{Cameron2017}, computer vision \cite{Marr2017} and sensor fusion \cite{Hoffman2018}. Appendix \ref{app:systems} discusses the systems which are used at the manned vessel, to form a correct mental model. For this research, only the result of this step is relevant: Is the acquired knowledge sufficient to identify the situation correctly, or is more information needed?
Future technologies and sensors are not within the scope of this research, nor how their outputs result in useful information. 

\subsubsection{Identify situation}
The step from information to knowledge is in the phase where the situation, scenario, and hazards are identified. How this would go in practice is discussed in chapter \ref{ch:identify-situation}. This step identifies critical situations during the design phase of an autonomous ship to be evaluated. 
This research will define a method to evaluate these critical situations. The layout of the waterway, other nearby vessels, relevant regulations, etc. determines the situation and scenario.

\subsubsection{Predict future states and decide on strategy}
Based on the situation, different strategies might be possible. A system or the operator has to evaluate these strategies. This evaluation is done by predicting how different strategies will influence the path of the various vessels. A trade-off must be made between exact calculations and computation time. For example is the \acf{CPA} currently determined using linearised algorithms in common \ac{ARPA} systems. Non-linearized methods using, for example, a Bézier curve will result in smaller errors. Simulations would improve this even more, however, does a simulation with correct hydrodynamic models cost much more computational time. In chapter~\ref{ch:criteria-problem}, the linearised and non-linearized methods are described. Appendix \ref{app:tool} describes the simulation tool. This tool is however not optimised for such calculations. Therefore will it not be able to do these calculations in real-time. The hydrodynamic model used in the simulation also described in appendix \ref{apps:hydro-model}. Different manoeuvers are evaluated after this phase, which corresponds to the different strategies. We will discuss this in chapter \ref{ch:criteria-manouvre} for common critical strategies. This will result in manoeuvring requirements. These requirements can be used by ship designers, to ensure that the ship can operate safely with minimal need for communication.
After the evaluation of these criteria is known which strategy will result in safe operation of the vessel.

\newpage
\section{External factors}
How easy it is to go through the decision process and end up at the right strategy depends on the situation. Environmental factors, such as the traffic situation, will mostly influence the situation. In some cases are the static sensors not sufficient to analyse the environment properly, resulting in an incomplete mental model. This section will describe in more detail how the environment influences the forming of the mental model and how safe operation within this environment would benefit from communication. 

\subsection{Environment}
As discussed before, is the mental model mostly a representation of the environment in which the vessel acts. The sensors will measure this environment. Many critical situations occurred due to weather conditions. The reason is that the sight is limited during heavy rain or snow. The wind and waves might limit the manoeuvring capabilities of a vessel. These are also the reason some vessels are not allowed to enter a port when wind or waves are too high.
If such repercussions are necessary, depend on the layout of the waterway. Due to currents, operations (e.g. maintenance and fishery) or limited depth, might some area's be restricted. Operators acquire this information via communication channels, but communication channels which only allow receiving and not sending, such as \acf{Navtex}.

The same goes for standard information on other ships. They might send their location and speed via \ac{AIS}, but still key is the \ac{ARPA}. Due to weather conditions, these systems could have worse performance, as heavy rain creates noise at the radar. In the situations where sensors do not function as expected. Communication is needed, even if the whole decision process itself is optimised to avoid communication.
The same goes for communication with shore-based stations such as traffic controllers or in the future remote pilots. Sensors or current systems are not able to retrieve this information. Such as a place and time to berth or pick-up a pilot.

Both shore-based stations and ships can only share intentions and their planned path via the radio. Future unmanned will most likely be able to negotiate using other systems. But in the case with manned-manned or manned-unmanned interaction is this only possible via \ac{VHF} radio for now.

\subsection{Communication}
As described are there still cases in which communication is necessary. We discuss in part \ref{part:CS} the case where communication between manned and unmanned vessel is needed. Using the \acf{sCE} method a protocol is defined, based on existing systems and protocols. Thus using \ac{AIS} to send written messages, or \ac{VHF} and \ac{SMCP} for verbal messages. 
Other cases such as communication with traffic controllers and pilots could use the same protocol. Although they might need to share more information with unmanned vessels, which could be done with a new system such as \ac{VDES}. This will however not be part of this research.
