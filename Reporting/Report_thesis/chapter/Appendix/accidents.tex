% What are problems with the current system
\chapter{Accidents}
\label{app:accidents}
In the last centuries much have changed to improve the safety of vessel and decrease the risk for collision. Some were reactions to major accidents which occurred. Such as the disaster with the TITANIC in April 1912, which triggered the development of \ac{SOLAS}. But also new innovations, such as the introduction of \ac{GPS}, \ac{ARPA} and \ac{AIS}. But still accidents occur. To get insight what could result in hazardous situations. Four accidents are discussed, showing the importance of proper communication on different levels. The accidents which will be discussed are:
\begin{itemize}
	\item Collision between MV AL ORAIQ and MV FLINTERSTAR
	\item Collision between MV ARTADI and MV ST-GERMAIN
	\item Collision between USS FITZGERALD and MV ACX CRYSTAL
	\item Collision between USS JOHN S MCCAIN and MV ALNIC MC
	\item Collision between MV CONTI PERIDOT and MV CARLA MAERSK
\end{itemize}
\todo{Add images of resulting damages}
\todo{Unify design of accident picture}

\newpage
\section{MV AL ORAIQ and MV FLINTERSTAR}
\label{sec:al-oraiq-vs-flinterstar}
During the night between 5th and 6th of October 2015 at the Northsea near Zeebrugge, a collision occurred between the LNG tanker AL ORAIQ and Dutch cargo ship FLINTERSTAR. The FLINTERSTAR sank almost immediately as a result of the collision, an illustration of the accident is shown in Figure \ref{fig:Accident-Flinterstar-Al-Oraiq}. The captain of the FLINTERSTAR was badly injured in the incident, the other ten people on board and the pilot were rescued out of the water unharmed.

\begin{figure}[H]
	\centering
	\includegraphics[width=.7\textwidth]{Flinterstar-Al-Oraiq-accident.png}
	\caption{Illutstration map of approximate collision location}
	\label{fig:Accident-Flinterstar-Al-Oraiq}
\end{figure}

The collision occurred because the bridge team on board of the AL ORAIQ wrongly assessed the traffic situation, vessel's speed and distance from the S3 buoy, prior to contacting the nearby vessel Thorco Challenger. After informing the Thorco Challenger, did they pass on the starboard side. On board of AL ORAIQ were coastal pilots which did not receive feedback from the watch keepers, nor was there feedback from other vessels via \ac{VHF} radio. The communication via VHF radio was mostly in dutch, the officer on duty at AL ORAIQ did not request the Coastal pilots to translate. Also did the bridge watch team not assess the situation properly, leading to very little situational awareness.
On board of the FLINTERSTAR there was insufficient attention for watch keeping duties. As several VHF radio communications between Traffic Centre Zeebrugge and other participants within the area monitored by Traffic Centre Zeebrugge, concerning or involving the presence of an inbound LNG carrier were missed by the Pilot and other crew at the bridge on board the FLINTERSTAR.

The pilot on board of AL ORAIQ did not attempt to work together with the crew. Thereby making decisions without consulting the crew, such as overtaking other vessels. 
The sea pilot on board of the FLINTERSTAR got engaged in a casual conversation with the officer of the watch, drawing his attention away from monitoring the traffic situation. The Sea Pilot was advising the officer of the watch from what appeared to be routine. \cite{Backer2015}

\newpage
\section{MV ARTADI and MV ST-GERMAIN}
\label{sec:artadiVSst-germain}
An example where the \ac{COLREGs} were followed but still resulted in two persons killed is the collision between MV ARTADI and MV ST-GERMAIN on February 21st, 1979. The Liberian bulk carrier ARTADI collided with the passenger ferry ST-GERMAIN in the Dover Strait, killing two people and injuring four more. An illustration of the accident is shown in figure \ref{fig:Artadi-Germain}. Both ships followed  \ac{COLREGs} according to the accident report. Due to a lack of communication and wrong presumptions on the intentions, did the accident occur.

\begin{figure}[H]
	\centering
	\includegraphics[width=.7\textwidth]{Artadi-Germain-accident.png}
	\caption{Illutstration map of approximate collision location}
	\label{fig:Artadi-Germain}
\end{figure}

The ferry was spotted in good time on the radar of the ARTADI. Coming from  starboard, ST-GERMAIN was the stand-on ship according to rule 15 of the \ac{COLREGs}. The pilot and master of the ARTADI expected her to keep speed and course and started to make a starboard turn to give way. However, on-board the ST-GERMAIN the intention was not at all to cross the traffic separation scheme diagonally in front of ARTADI, but instead to turn port and follow outside the boarder of the NE going traffic lane until the traffic cleared and she could make the crossing at a right angle (according to rule 10c) \cite{Porathe2013}.

\newpage
\section{USS FITZGERALD and MV ACX CRYSTAL}
A more recent collision was between the USS FITZGERALD and ACX CRYSTAL on June 17, 2017. The US destroyer hit the Philippines container vessel, resulting in the death of 7 US Sailors. An illustration of the accident is shown in figure \ref{fig:Accident-USS-Fitzgerald-Crystal}. According to the accident report did failures occur on the part of leadership and watch-standers. There were failures in planning for safety, adhere basic navigational practice, execute basic watchstanding practice, proper use of available navigation tools and wrong responses.

\begin{figure}[H]
	\centering
	\includegraphics[width=.7\textwidth]{USS-Fitzgerald-Crystal-crash.png}
	\caption{Illutstration map of approximate collision location}
	\label{fig:Accident-USS-Fitzgerald-Crystal}
\end{figure}

In accordance with international rules, the USS FITZGERALD was obligated to manoeuvrer to remain clear from the other crossing ships. The officer of the deck responsible for navigation and other crew discussed whether to take action but choose not to, till it was too late. While other crew members also failed to provide more situational awareness and input to the officer of the deck. Did the officer of the deck, exhibit poor seamanship by failing to manoeuvrer as required, failing to sound the danger signal and failing to attempt to contact CRYSTAL on Bridge to Bridge radio. In addition, the Officer of the Deck did not call the Commanding Officer as appropriate and prescribed by Navy procedures to allow him to exercise more senior oversight and judgment of the situation. This was prescribed to an unsatisfactory level of knowledge of the international rules of the nautical road by USS FITZGERALD officers. Thereby were watch team members not familiar with basic radar fundamentals, impeding effective use. Thereby were key supervisors not aware of existing traffic separation schemes and the expected flow of traffic, as the approved navigation track did not account, nor follow the Vessel Traffic Separation Scheme. Secondary was the automated identification system not used properly. \cite{USNavy2017}

\newpage
\section{USS JOHN S MCCAIN and MV ALNIC MC}
Even more recent is the collision between the USS JOHN S MCCAIN and ALNIC MC on August 21st, 2017. The US Destroyer hit the Liberia flagged oil and chemical tanker. Resulting in the death of 10  US Sailors. An illustration of the accident is shown in figure \ref{fig:Accident-USS-John-S-McCain-Alnic}. According to the accident report did the US Navy identify the following causes for the collision: Loss of situational awareness in response to mistakes in the operation of the USS JOHN S MCCAIN's steering and propulsion system, while in the presence of a high density of maritime traffic. Failure to follow the international nautical rules of the road, which govern the manoeuvring of vessels when risk of collision is present.

\begin{figure}[p]
	\centering
	\includegraphics[width=.8\textwidth]{USS-John-S-McCain-Alnic-accident.png}
	\caption{Illustration map of approximate collision location}
	\label{fig:Accident-USS-John-S-McCain-Alnic}
\end{figure}

Leading up to the accident did the commanding officer notice that the helmsman had difficulties maintaining course, while also maintaining control over speed. In response, he ordered the watch team to divide the duties of steering and speed control. This unplanned shift caused confusion within the watch team, which led to wrong transfers of control, where the crew was not aware of. 
Watchstanders failed to recognize the configuration. The steering control transfer caused the rudder to go amidships (centerline). Since the Helmsman had been steering less than 4 degrees of right rudder to maintain course before the transfer, the amidships rudder deviated the ship’s course to the left. Additionally, when the Helmsman reported a loss of steering, the Commanding Officer slowed the ship to 10 knots and eventually to 5 knots. Due to the wrong transfer did only one shaft slow down, causing an un-commanded turn to the left (port). The commanding officer and others on the ship's bridge lost situational awareness. They did not understand the forces acting on the ship, nor did the understand the ALNIC's course and speed relative to USS JOHN S MCCAIN. Three minutes after the reported loss of steering, was it regained, but already too late to avoid a collision. No signals of warning were send by neither ship, which are required by international rules of the nautical road. Nor was there an attempt to make contact via \ac{VHF} bridge-to-bridge communication.
Many of the decisions made that led to the accident were the result of poor judgment and decision making of the commanding officer. That said, no single person bears full responsibility for this incident. The crew was unprepared for the situation in which they found themselves through a lack of preparation and ineffective command and control. Deficiencies in training and preparations for navigation were at the base of this. \cite{USNavy2017}

\section{MV CONTI PERIDOT and MV CARLA MAERSK}
The last accident which will be discussed is the collision between MV CONTI PERIDOT and MV CARLA MAERSK on 9th March 2015. At 12:30 central daylight time, the the inbound bulk carrier CONTI PERIDOT collided with the outbound tanker CARLA MAERSK in the Houston Ship Channel near Morgan’s Point, Texas. The collision occurred in restricted visibility after the pilot on the CONTI PERIDOT was unable to control the heading fluctuations that the bulk carrier was experiencing during the transit. As a result, the CONTI PERIDOT crossed the channel into the path of the CARLA MAERSK. No one on board either ship was injured in the collision, but an estimated 2,100 barrels (88,200 gallons) of methyl tert-butyl ether spilled from the CARLA MAERSK, and the two vessels sustained about \$8.2 million in total damage. In figure \ref{fig:Conti-Peridot-Carla-Maersk-crash} the fluctuations in heading can be seen of the CONTI PERIDOT. 

\begin{figure}[p]
	\centering
	\includegraphics[width=.8\textwidth]{Conti-Peridot-Carla-Maersk-crash.png}
	\caption{Illutstration map of approximate collision location}
	\label{fig:Conti-Peridot-Carla-Maersk-crash}
\end{figure}

Several safety issues were identified by the National Transportation Safety Board. Inadequate bridge resource management: Despite the pilot’s difficulty controlling the CONTI PERIDOT’s heading leading up to the collision, he and the master did not work together to solve the problem. The pilot did not involve the master because he was unsure whether the master could do anything to help; the master said nothing because he was likely unaware of the vessel’s heading fluctuations and may have been generally reluctant to question the pilot.
Insufficient pilot communications: Although the pilot on the CONTI PERIDOT was having difficulty controlling the vessel and had an earlier near-miss meeting with an oncoming ship, he did not alert the pilots on subsequent oncoming vessels, including the CARLA MAERSK.
Lack of predetermined ship movement strategies during restricted visibility in the Houston Ship Channel: On the day of the accident, local pilot associations  determined that the increasing fog was significant enough to suspend pilot boardings of inbound ships. However, piloted vessels already under way continued the transit in the fog. Investigators found no existing predetermined ship movement strategy for piloted vessels already under way at the onset of hazardous weather conditions.

The National Transportation Safety Board determines that the probable cause of the collision between bulk carrier Conti Peridot and tanker Carla Maersk in the Houston Ship Channel was the inability of the pilot on the Conti Peridot to respond appropriately to hydrodynamic forces after meeting another vessel during restricted visibility, and his lack of communication with other vessels about this handling difficulty. Contributing to the circumstances that resulted in the collision was the inadequate bridge resource management between the master and the pilot on the Conti Peridot. \cite{NTSB2016}\cite{NTSB2016a}

\section{Lessons learned}
In all of the above described accidents, mistakes were made. But in all cases was there not sufficient communication, to warn the other vessel. If this would have happened, they could have taken actions to avoid the collision. In order to achieve effective communication, there should be an universal protocol. This protocol should be used under all circumstances. Thereby is it important that all seafarers understand this protocol, and are able to work with it. Communication with this protocol should be engaged in all cases when there is any doubt if intentions are understood, or if full control over the situation is lost. This is also part of proper bridge resource management. Where a good balance of crew should be at the bridge, to avoid overload. But more importantly also avoid the bystander effect in cases there are too many \cite{Fischer2011}. The crew that is available at the bridge should be aware of forces acting on the vessel, the effect of these forces and notice when the ship does not act as expected. In these situations an emergency protocol or strategy should be executed, which is known by all active crew members. 
