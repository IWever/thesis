\chapter*{Manoeuvring capability}
Ship manoeuvring is the ability to keep course, change course, keep track and change speed. Minimal requirements are given by \ac{IMO} standard. However, shipowners may introduce additional requirements. 
Ship manoeuvrability is described by the following characteristics: 
\begin{itemize}
	\item Initial turning ability (start turning)
	\item Sustained turning ability (keep turning)
	\item Yaw checking ability (stop turning motion)
	\item Stopping ability (in rather short distance and time)
	\item Yaw stability (ability to move straight ahead)
\end{itemize}
During sea-trials these capabilities can be determined. However this project will aim at predicting manoeuvrability while using limited input. Thereby is there a difference between the maximum limits and what a ship is likely to do. This will eventually lead to the possible movements of the vessel.

\section*{IMO standard}
The manoeuvrability of a ship is considered satisfactory is the following criteria are complied:
\begin{enumerate}
	\item \emph{Turning ability}. The advance should not exceed 4.5 ship lengths (L) and the tactical diameter should not exceed 5 ship lengths in the turning circle manoeuvre.
	\item \emph{Initial turning ability}. With the application of 10\degree rudder angle to port or starboard, the ship should not have traveled more than 2.5 ship lengths by the time the heading has changed by 10\degree from the original heading. 
	\item \emph{Yaw-checking and course-keeping abilities}. 
	\begin{enumerate}
		\item The value of the first overshoot angle in the 10\degree/10\degree zig-zag test should not exceed: 
		\begin{enumerate}
			\item 10\degree if L/V is less than 10 seconds
			\item 20\degree if L/V is 30 seconds or more
			\item (5 + 1/2(L/V)) degrees if L/V is between 10 and 30 seconds
		\end{enumerate}
		where L and V are expressed in m and m/s, respectively.
		\item The value of the second overshoot angle in the 10\degree/10\degree zig-zag test should not exceed:
		\begin{enumerate}
			\item 25\degree if L/V is less than 10 seconds
			\item 40\degree if L/V is 30 seconds or more
			\item (117.5 + 0.75(L/V)) degrees if L/V is between 10 and 30 seconds
		\end{enumerate}
		\item The value of the first overshoot angle in the 20\degree/20\degree zig-zag test should not exceed 25\degree. 
	\end{enumerate}
	\item \emph{Stopping ability}. The track reach in the full astern stopping test should not exceed 15 ship lengths. However, this value may be modified by the Administration where ships of large displacement make this criterion impracticable, but should in no case exceed	20 ship lengths. 
\end{enumerate}

\section*{Limits}
These standards give guidance during seatrials, but won't help much 
What are maximum values for manoeuvring capability. Based on trial run are values found for Nomoto (other theories?)

Wat is constant? Versnelling/vertraging of de afgeleide daarvan

Clarke, D., Gedling, P. and Hine, G. (1983). The application of manoeuvring criteria in hull design using linear theory. The Naval Architect, pp. 45–68

\section*{Desired capability}
What are normal movements for a ship of a specific size

\section*{Expected route}
Ship will most likely keep sailing straight and on same speed
\todo{describe formula to determine crossingpoint of line}
\todo{CPA calculation}

\section*{Input}
Nomoto, more detailed is Norrbin equation

\subsection*{Detailed capability}
Key equipment for the manoeuvrability are rudders, fixed fins, jet thrusters, propellers, ducts and waterjets. However it is not practical to determine this for every ship which is nearby. Therefore a more statistical approach is taken using comparable ships.






\subsection*{Prediction with limited data}
Own vessel input comes from sea-trial, other vessels based on received information via AIS.
DWT, L, B, speed, etc.

