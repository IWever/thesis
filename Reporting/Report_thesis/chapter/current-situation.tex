\chapter{Current situation}
Safety at sea has been a relevant topic as long as ships exist. And although the shipping industry exists for a long time, communication is where still major steps are taken to improve this safety. 

Before the invention of radio communication, ships literally lost all connection with the shore and other ships when setting sail. Using flags some communication was possible when ships were close to each-other or to shore, but this only gave limited insight in the intentions of other vessels. 

In the last centuries much have changed, mostly reactive on accidents which occurred. A key event was the disaster with the TITANIC on 15th April 1912, which leaded to the international treaty \ac{SOLAS}. Despite new rules, accidents still occur. Some of them will be discussed below. Followed by a description of the means for communication, equipment used at the bridge and finally whom is using it in what way.

\section{Accidents}
Below three accidents are discussed, showing the importance of proper communication to improve situational awareness. The accidents which will be discussed are:
\begin{itemize}
	\item Collision between MV AL ORAIQ and MV FLINTERSTAR
	\item Collision between USS FITZGERALD and MV ACX CRYSTAL
	\item Collision between USS JOHN S MCCAIN and MV ALNIC MC
\end{itemize}

\newpage
\subsection{MV AL ORAIQ and MV FLINTERSTAR}
During the night between 5 and 6 October 2015 on the Northsea near Zeebrugge, a collision occurred between the LNG tanker AL ORAIQ, sailing under the Marshall Islands flag, and the FLINTERSTAR cargo ship, sailing under the Dutch flag. The FLINTERSTAR sank almost immediately as a result of the collision, an illustration of the accident is shown in Figure \ref{fig:Accident-Flinterstar-Al-Oraiq}. The captain of the FLINTERSTAR was badly injured in the incident but the other ten people on board and the pilot were rescued out of the water unharmed.

\begin{figure}[H]
	\centering
	\includegraphics[width=.7\textwidth]{Flinterstar-Al-Oraiq-accident.png}
	\caption{Illutstration map of approximate collision location}
	\label{fig:Accident-Flinterstar-Al-Oraiq}
\end{figure}

The collision occurred because the bridge team on board of the AL ORAIQ wrongly assessed the traffic situation, vessel's speed and distance from the S3 buoy, prior to contacting the nearby vessel Thorco Challenger. After informing the Thorco Challenger, did they pass on the starboard side. On board of AL ORAIQ were coastal pilots which did not receive feedback from the watch keepers, nor was there feedback from other vessels via \ac{VHF} radio. The communication which went via VHF radio was mostly in dutch, the officer on duty at AL ORAIQ did not request the Coastal pilots to translate. Also did the bridge watch team not assess the situation properly, leading to very little situational awareness.
On board of the FLINTERSTAR there was insufficient attention for watch keeping duties. As several VHF radio communications between Traffic Centre Zeebrugge and other participants within the area monitored by Traffic Centre Zeebrugge, concerning or involving the presence of an inbound LNG carrier were missed by the Pilot and other crew at the bridge on board the FLINTERSTAR.
The pilots on board of AL ORAIQ did not attempt to work together. Thereby making decisions without consulting the crew, such as overtaking other vessels. Thus the coastal pilot did not act consistently with international understanding, where a pilot is an advisor to the ship's master. Which means mutual understanding for the functions and duties of each other, based upon effective communication and information exchange. 
The sea pilot on board of the FLINTERSTAR got engaged in a casual conversation with the officer of the watch, drawing his attention away from monitoring the traffic situation. The Sea Pilot was advising the officer of the watch from what appeared to be routine. \cite{Backer2015}

\newpage
\subsection{USS FITZGERALD and MV ACX CRYSTAL}
A more recent well-known collision was between the USS FITZGERALD and ACX CRYSTAL on 17th June 2017. The US destroyer hit the larger Philippines container vessel resulting in the death of 7 US Sailors. An illustration of the accident is shown in figure \ref{fig:Accident-USS-Fitzgerald-Crystal}. According to the accident report did failures occurred on the part of leadership and watch-standers. There were failures in planning for safety, adhere basic navigational practice, execute basic watchstanding practice, proper use of available navigation tools and wrong responses.

\begin{figure}[H]
	\centering
	\includegraphics[width=.7\textwidth]{USS-Fitzgerald-Crystal-crash.png}
	\caption{Illutstration map of approximate collision location}
	\label{fig:Accident-USS-Fitzgerald-Crystal}
\end{figure}

In accordance with international rules, the USS FITZGERALD was obligated to manoeuvrer to remain clear from the other crossing ships. The officer of the deck responsible for navigation and other crew discussed whether to take action but choose not to, till it was too late. While other crew members also failed to provide more situational awareness and input to the officer of the deck. Did the officer of the deck, exhibit poor seamanship by failing to manoeuvrer as required, failing to sound the danger signal and failing to attempt to contact CRYSTAL on Bridge to Bridge radio. In addition, the Officer of the Deck did not call the Commanding Officer as appropriate and prescribed by Navy procedures to allow him to exercise more senior oversight and judgment of the situation. This was prescribed to an unsatisfactory level of knowledge of the international rules of the nautical road by USS FITZGERALD officers. Thereby were watch team members not familiar with basic radar fundamentals, impeding effective use. Thereby were key supervisors not aware of existing traffic separation schemes and the expected flow of traffic, as the approved navigation track did not account, nor follow the Vessel Traffic Separation Scheme. Secondary was the automated identification system not used properly. \cite{USNavy2017}

\newpage
\subsection{USS JOHN S MCCAIN and MV ALNIC MC}
Even more recent is the collision between the USS JOHN S MCCAIN and ALNIC MC on 21st August 2017. The US Destroyer hit the Liberia flagged oil and chemical tanker. Resulting in the death of 10  US Sailors. An illustration of the accident is shown in figure \ref{fig:Accident-USS-John-S-McCain-Alnic}. According to the accident report did the US Navy identify the following causes for the collision: Loss of situational awareness in response to mistakes in the operation of the USS JOHN S MCCAIN's steering and propulsion system, while in the presence of a high density of maritime traffic. Failure to follow the international nautical rules of the road, which govern the manoeuvring of vessels when risk of collision is present. Watchstanders operating the JOHN S MCCAIN's steering and propulsion systems had insufficient proficiency and knowledge of the system. 

\begin{wrapfigure}{L}{0.6\textwidth}
	\centering
	\includegraphics[width=.6\textwidth]{USS-John-S-McCain-Alnic-accident.png}
	\caption{Illutstration map of approximate collision location}
	\label{fig:Accident-USS-John-S-McCain-Alnic}
\end{wrapfigure}

Leading up to the accident did the commanding officer notice that the helmsman had difficulties maintaining course, while also adjusting the throttles for speed control. In response, he ordered the watch team to divide the duties of steering and throttles, maintaining course control with the Helmsman while shifting speed control to another watchstander. This unplanned shift caused confusion within the watch team, which led to wrong transfers of control, where the crew was not aware of. 
Watchstanders failed to recognize this configuration. The steering control transfer caused the rudder to go amidships (centerline). Since the Helmsman had been steering less than 4 degrees of right rudder to maintain course before the transfer, the amidships rudder deviated the ship’s course to the left. Additionally, when the Helmsman reported a loss of steering, the Commanding Officer slowed the ship to 10 knots and eventually to 5 knots. Due to the wrong transfer did only one shaft slow down, causing an un-commanded turn to the left (port). The commanding officer and others on the ship's bridge lost situational awareness. They did not understand the forces acting on the ship, nor did the understand the ALNIC's course and speed relative to USS JOHN S MCCAIN. Three minutes after the reported loss of steering, was it regained, but already too late to avoid a collision. No signals of warning were send by neither ship, which are required by international rules of the nautical road. Nor was there an attempt to make contact through the \ac{VHF} bridge-to-bridge communication.
Many of the decisions made that led to the accident were the result of poor judgment and decision making of the commanding officer. That said, no single person bears full responsibility for this incident. The crew was unprepared for the situation in which they found themselves through a lack of preparation, ineffective command and control. Deficiencies in training and preparations for navigation were at the base of this. \cite{USNavy2017}

\section{Means of communication}
Communication is a very broad concept and comes in many forms, as it includes everything which enables the exchange of information. The main reason to communicate is improve the safety of life at sea. More specifically to show intentions or ask for aid. The main means of communication are nowadays:

\begin{itemize}
	\item Visible signals
	\begin{itemize}
		\item Change of heading
		\item Light signals
		\item Flags or symbols
	\end{itemize}
	\item Availability on \ac{VHF}
	\item Exchange of information via \ac{AIS}
	\item Horn
\end{itemize}
\todo{it this the complete list?}

\section{The bridge}
The bridge of a vessel can be separated into four elements. The human operator, procedures,
technical system and the human-machine interface. This chapter will focus on the technical system and human-machine interface. Thereby a separation will be made between the instruments available, the information which can be deducted from this and how this can be used. 

\begin{figure}[H]
	\centering
	\includegraphics[width=0.4\textwidth]{Bridge-system-elements.png}
	\caption{Bridge system elements}
	\label{fig:Bridge-system-elements}
\end{figure}

The ship’s navigation bridge shall enable the officer in charge of the navigational watch to perform navigational duties unassisted at all times during normal operating conditions. He shall be able to maintain a proper lookout by sight and hearing as well as by all available means appropriate in the prevailing circumstances and conditions so as to make full appraisal of the situation and the risk of collision, grounding and other hazards to navigation.

\subsection{Technical system}
At least the following instruments and equipment shall be installed \cite{DNVGL2011}: 
\begin{multicols}{2}
	\begin{itemize}
		\item Navigation radar with radar
		\item Propulsion control
		\item Manual steering device
		\item Heading control
		\item Other related \ac{UID}s
		\item Electronic Chart Display Information System (ECDIS)
		\item Steering mode selector switch
		\item VHF unit
		\item Whistle and manoeuvring light push buttons
		\item Internal communication equipment
		\item Central alert management system
		\item General alarm control
		\item Window wiper and wash controls
		\item Control of dimmers for indicators and displays
		\item Propulsion
		\item Emergency stop machinery
		\item Gyrocompass selector switch
		\item Steering gear pumps
	\end{itemize}
\end{multicols}


What do regulations say about systems which should be on board


\subsection{Man/machine interface}
How is information used

\subsection{Procedures}
Training, education and protocols

\subsection{Human operator}
How does it go in practice, experience.

