\chapter{System design specification}
The system design specification describes the solution to the problem in the form of a system design that makes use of the identified relevant human factors knowledge and the envisioned technology. The system specification consists of (a) design scenarios, (b) actors and use cases, (c) requirements, (d) claims, and (e) ontology.

\section{Design scenarios}
The sCE method prescribes the specification of design scenarios. Design scenarios are short stories that provide a clear description of how the user will work with the technology thereby enjoying the solution offered to one of the problem scenarios. Together, the problem and design scenarios provide a contextualized view on:
\begin{enumerate}
	\item The problem the design aims to solve.
	\item The people that are currently affected by this problem.
	\item The way in which the current system design aims to solve this problem.
	\item How people will use the system.
\end{enumerate}

Manned ships are able to understand intentions of unmanned ships, resulting in good situational awareness. In cases they desire more information, they are able to acquire this using current systems. Without the risk for an information overload. Thereby are the additions to existing protocols for those systems easy to understand, as they use the same philosophy as current protocols.

Trust in autonomous ships is formed, as the information is reliable, the interaction is similar to the interaction with other manned ships. This resulted in acceptance of unmanned ships on the general waterways. Where the risk of collisions and perceived risk did not increase.

\section{Use case}
Scenarios are used to create more specific descriptions of step-by-step interactions between the technology and its users (i.e. use cases). Use cases include actors, to specify which stakeholders/agents are interacting with each other in a given action sequence (use case).

NB: Use cases do not specify the way in which the technology enables the described interactions. For example, the interactions may take place through voice commands and audio, but could also be text-based, be instantiated with the use a drop-down menu, or even by a human operator sitting on the other side of the application, listening and responding to the events taking place. No assumptions are made about the level of automation or the current capabilities of the technology in mind. A use case simply describes the behaviour of the system, regardless of the technology required to produce that behaviour. Because the sCE method describes an iterative process of specifying the system’s design, at later stages some behaviours may prove to be infeasible or not viable. This may result in alternative use cases that may be better aligned with the available technology. But it may also be the case that one ends up with a slightly different version of the technology that in fact is able to produce the ideal behaviour described in the original use case. The main goal of iteratively refining the system specification is to gather all the alternative design solutions, compare them in systematic evaluations, and converge to one design solution that is effective, reliable, affordable, etc

The use cases provide a detailed description of the interactions between the technology and its user. Use cases make the design scenario more concrete by describing exactly how the technology makes sure that the elderly is safe and taken care of. Use cases are informed by human factors theories (described in the system’s foundation).

\section{Functional requirements}
In the sCE method, use cases are used to derive functional requirements, i.e. specific functionalities the technology should provide to its user.

\section{Claim}
The sCE method prescribes a strong link between the system’s functional requirements, the system’s objectives, and the hypotheses to be tested during system evaluations. This
is accomplished by annotating all functional requirements with their underlying objectives (called claims).

This explicit linking of requirements to claims enables designers to formulate hypotheses that need to be tested in system evaluations to justify the adoption of the functional requirement. If the claim cannot be proven to be valid through system evaluations, the designers need to refine their system design, for instance, by trying to improve the functionality, replacing the functionality with a different one, or dropping the functionality and the claim altogether (i.e. by deciding that the objective is not reachable at this point). Either way, there is no use of including a functionality that does not achieve its underlying claims.

\section{Ontology}
Lastly, the sCE method prescribes the construction of an ontology, i.e. a vocabulary describing a common language to be used throughout the system specification to avoid miscommunication, misunderstanding, and inconsistencies. Furthermore, the ontology can serve as the basis for the technology’s data structure. By specifying important concepts in the ontology and also choosing to use only one word instead of various ambiguous synonyms, communication becomes clearer and misunderstandings can be reduced to a minimum. The terms specified in the ontology are consistently used throughout the entire project.
Running
