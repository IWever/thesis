\chapter{Foundation}
The foundation segment in the \ac{sCE} methodology describes the design rationale in terms of (a) operational demands, (b) relevant human factors knowledge, and (c) envisioned technologies. Together, these three constituents describe (1) the problem to be solved, (2) the existing knowledge on ways to solve the problem, and (3) the technology needed to implement that solution.

\section{Operational demands}
The operational demands describe the current practice as it is, i.e. without the envisioned technology. For the operational demands, the sCE method prescribes as main components the stakeholders and their characteristics, and the problem description and analysis thereof.

\subsection{Problem scenario}
\acf{COLREGs} have been developed long before bridge-to-bridge voice communications became available. They are supposed to be unambigious. It is the responsibility of all bridge watch keepers to know how to apply them instinctively, on the basis of observation primarily by sight and radar. They work effectively when ships in an interaction obey them; they also specifically address circumstances where one ship does not.

However, as shown in the previous parts, are \ac{COLREGs} not always sufficient to decide on the strategy. For example due to missing information. In those cases the \ac{VHF} radio can be used for verbal bridge-to-bridge communication. Leading here is the \acf{SMCP}. The primary task of this protocol is to diminish misunderstanding in safety related verbal communications. 
Beside this verbal communication, non-verbal communication might be used, such as light signals, sound signals and semaphore signing.
The reason safety related verbal communication is needed, when you will deviate from the rules, deviant behavior is registered, or more information is needed to take the right decision.

The different ways for communication are not developed to be used by unmanned vessels. On the other hand, is it not feasible to require all manned vessel to install new systems for communication, before introducing unmanned vessels.

When using the current systems in a way they are not designed for. This might result in information overload of the crew and communication channels. Resulting in misunderstanding and problems with communication. As the \ac{VHF} can only be used in one way. As it is a receiver or a transmitter, but can't be both at the same time. 

\subsection{Problem analysis}
To avoid misunderstanding which could result in hazardous situations. It is important that manned and unmanned vessels are able to communicate. To solve this problem, a more extensive analysis is made. 
Describing the values of the stakeholders, other actors, and related problems.

\subsubsection{Stakeholders}
The focus of this research will be on bridge-to-bridge communication. This means the most important stakeholders are the bridges of the manned and unmanned vessels.
\begin{itemize}
	\item \emph{Officer of watch}. He is the responsible person. He might work together with a helmsman and a lookout. He has to ensure a proper functioning of all available systems. Thereby does he discuss with other crew members if there are any unusual activities. He is responsible to follow a proper navigation plan, while having his own safe passage plan, to avoid collisions. He will use sight, \ac{ARPA} and \ac{ECDIS}. Thereby is he aware of the ship's speed, turning circle and other handling characteristics to decide on the right strategy. He will monitor the \ac{VHF} radio all the time while underway to assist in emergencies if necessary, to hear Coast Guard alerts for weather and hazards or restrictions to navigation, and to hear another vessel hailing you.
	
	He wants to avoid information overload, while being aware of the situation. This is only possible when he stays concentrated, to acquire this, the tasks must be challenging and he needs to have a form of autonomy.

	\item \emph{Helmsman and lookout}. Both monitor the situation and execute commands from the officer of watch. They also want to avoid information overload.
	
	\item \emph{Crew and pilots on nearby vessels}. They might want to know the intentions of other vessels to base their strategy on. However do not want to receive all discussions. As this will result in an information overload.
\end{itemize}

\subsubsection{Other possible actors}
\begin{itemize}
	\item The agent controlling the unmanned vessel. \todo{is dit een actor?}
	\item Vessel traffic controllers.
	\item Crew on vessels which are not traveling.
	\item Crew which are in distress and require assistance.
	\item Shipowners of unmanned vessels, monitoring vessel from other location.
\end{itemize}

\subsubsection{Information extraction from problem scenario}
\begin{itemize}
	\item Different actors are afraid of information overload.
	\item Officer of watch is afraid to lose situational awareness.
	\item Officer of watch is afraid to lose autonomy.
	\item Current systems not designed to be used by unmanned vessels.
	\item Manned ships want to ask for support or information.
	\item Unmanned ships want to ask for support or information.
\end{itemize}

\subsubsection{Goals}
The main goal is to ensure reliable sharing of information, without the risk for information overload or misunderstanding. Such that manned ships will trust unmanned ships to choose the right strategy. As manned ships can be informed using human language which reasoning is used by unmanned vessels.

\subsubsection{Problem breakdown}
This means that manned vessels should be updated only when requested or in case of an unusual activity which could affect their strategy. And manned vessel should be aware when unmanned vessels desire more information to decide on the right strategy.

\subsubsection{Possible situations when communication is needed}
\begin{itemize}
	\item No visual
	\item Bad weather
	\item Unreliable information from AIS
\end{itemize}

\subsubsection{Problems that might also be adressed}
Communication with traffic controllers. If it is also not feasible for them to have a new system to communicate and direct unmanned ships.

\subsubsection{Infeasible solutions}
The easiest solution for unmanned ships would be to just install a new system. However this is not feasible, as this would mean that all ships which could encounter an unmanned ship will have to install this too. It might be possible to make it obligatory via regulations, this will however cost a lot of time and money. Making the introduction of unmanned ships less likely.
This is also the reason to use a \ac{no-UI}, as a GUI will require new screens or changes to the \ac{ECDIS} which are only possible when regulations are changed.

\section{Human factors}
When designing technology, there are two driving questions that need to be well-thought out: (1) What tasks and/or values is the user trying to accomplish and how can the technology support the user in doing so?, and (2) How can the technology be designed such that the user is able to work with the technology?

The Human Factors segment of the sCE method describes the available relevant knowledge about, for instance, human cognition, performance, task support, learning, human-machine interaction, ergonomics, etc. Note that we emphasize that this knowledge should be relevant for the problem and its design solution: the knowledge described here should lead to a better understanding of either (1) or (2).

The three constituents important to the human factors analysis are: (a) the human factors knowledge, (b) measures, and (c) interaction design patterns.

\subsubsection{Human factor knowledge}
Human factors knowledge describes available knowledge coming from previous research about how to solve the problems that have been identified in the problem analysis.
\begin{itemize}
	\item Which information is needed to have situational awareness? \cite{HFW2002} \cite{Breda1999} \cite{Prison2013} \cite{Hodgetts2015} \cite{Porathe2014}
	\item When does information overload occur? \cite{Arimura2001} \cite{Neerincx2008} \cite{Schutter2016} \cite{Porathe2014}
	\item What is more effective to ensure a feeling of autonomy? \cite{Feys2016}
	\item How is information perceived if it is being pushed or pulled? Thus listen to radio with continious information (repeating), or only send message when asked for
	\item What are the expectations of a human from an unmanned ship?
	\item Do people trust automated systems in emergency situations? \cite{Neerincx2008} \cite{Walliser2011}
	\item Is there an uncanny valley for chatbots?
	\item ...
\end{itemize}
\todo{link papers to questions and answer}

\subsubsection{Human factor measures}
Measures describe how to operationalize the quality of the intended behavior or performance, i.e. how well is a user working with the design able to reach his/her objectives and what is the quality of the collaboration between the human worker and the technology?
\begin{itemize}
	\item Is the system used correctly?
	\item Does the interface give reliable information?
	\item Will it solve the problem of missing information?
	\item ...
\end{itemize}

\subsubsection{Interaction design patterns}
Interaction design patterns (IDPs) focus on the human-computer interaction, such as usable interface design and control options. IDPs offer generic solutions to recurring HCI design problems that have been proven to be effective. IDPs are often made available in repositories.

\begin{itemize}
	\item Auditory alarms
	\item Distress, urgency and safety signals
	\item Usage of message markers
	\item Chatbot
	\item Natural language variations on \ac{SMCP}
	\item Light signals
	\item Sound signals
	\item ...
\end{itemize}

\section{Envisioned technology}
The envisioned technology describes the available options of using existing technology and/or the need to develop novel technology in order to come to a system solution. The sCE method asks designers to specify what devices (hardware) and software could be used in the system design. In addition, for each type of technology, an argument should be provided as to why this technology might be of use and what the possible downsides might be of that particular type of technology.

Using only existing systems to develop a \acf{no-UI}. These existing systems are described in chapter \ref{ch:systems}:

\begin{itemize}
	\item Visible signals
	\begin{itemize}
		\item Positions
		\item Change of heading
		\item Light signals
		\item Flags or symbols
	\end{itemize}
	\item Availability on \ac{VHF}
	\item Exchange of information via \ac{AIS} \todo{is it possible to send direct messages via AIS?}
	\item Horn
\end{itemize}

Thereby is it easier to learn when its based on known protocols, such as \acf{SMCP} and \ac{COLREGs}.

Introducing optical signs to show you are unmanned might be good.