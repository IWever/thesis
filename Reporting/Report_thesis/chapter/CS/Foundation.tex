\chapter{Foundation}
The foundation segment in the \acf{sCE} methodology describes the design rationale in terms of operational demands, relevant human factors knowledge, and envisioned technologies. Together, these three constituents describe the problem to be solved, the existing knowledge on ways to solve the problem, and the technology needed to implement that solution.

\section{Operational demands}
The operational demands describe the current practice as it is, i.e. without the envisioned technology. For the operational demands, the sCE method prescribes as main components the stakeholders with their characteristics and the problem description with an analysis thereof.

\subsection{Problem scenario}
\acf{COLREGs} have been developed long before bridge-to-bridge voice communication became available. These are supposed to be unambiguous. It is the responsibility of all bridge watchkeepers to know how to apply these instinctively, from observation by sight and radar. The regulations work effectively when ships in an interaction obey these; the regulations also specifically address circumstances where one ship does not.

However, as shown in the previous parts, \ac{COLREGs} are not always sufficient to decide on the right strategy, for example, due to missing information. Problems with \ac{COLREGs} happen already more often due to larger ships, more complex manoeuvers and more traffic. In those cases, operators can use the \ac{VHF} radio for verbal bridge-to-bridge communication. Leading here are the \acf{SMCP}. The primary task of these phrases and surrounding protocol is to diminish misunderstanding in safety-related verbal communications. Beside this verbal communication, non-verbal communication might be used, such as light signals, sound signals and text messaging.

Three reasons for safety-related verbal communication are: As the operator you will deviate from the rules, you register deviant behaviour at other ships, or more information is needed to decide on the right strategy. Communication is often necessary due to the lack of visual information. For example, due to bad weather, obstacles like bridges and terminals, or the information received via \ac{AIS} is not reliable. Besides the impact on the information you get by looking out of the window, is also the quality of the \acf{ECDIS} and \acf{ARPA} worse.

The different ways for communication are not developed to be used by unmanned vessels. On the other hand, it is not feasible to require all manned vessels to install new systems for communication, before introducing unmanned vessels. As this will require many more new regulations, development time and time to train seafarers.

Misunderstanding and problems with communication can be the result of changing the usage of systems over time, due to the evolving demands for operators. These changes can result in an information overload of the crew and communication channels, caused by the more frequent use of \ac{VHF}. As it is a receiver or a transmitter, but can't be both at the same time, which means that in case two messages are sent at the same time, both senders will not receive the others message.

\subsection{Problem analysis}
To avoid misunderstanding that could result in hazardous situations, it is important that manned and unmanned vessels are able to communicate. A more extensive analysis is made to solve this problem. 
Describing the values of the different actors and discussing their related problems.

\subsubsection{Primary actors}
The focus of this research will be on bridge-to-bridge communication. For this communication are the most important actors for unmanned and manned vessels:
\begin{itemize}
	\item Manned vessel
	\begin{itemize}
		\item \emph{Officer of watch}. He is the responsible person. He might work together with a helmsman and a lookout. He must ensure proper functioning of all available systems. He does discuss with other crew members if there are any unusual activities. He is responsible for following a proper navigation plan while having his safe passage plan, to avoid collisions. He will use sight, \acf{ARPA} and \acf{ECDIS}. Thereby he is aware of the ship's speed, turning circle and other handling characteristics to decide on the right strategy. He will monitor the \ac{VHF} radio all the time while underway, to assist in emergencies if necessary, to hear Coast Guard alerts for weather and hazards or restrictions to navigation, and to hear another vessel hailing you.
		
		He wants to avoid information overload while being aware of the situation, which is only possible when he stays concentrated. This happens when the tasks are challenging, and he needs to have a form of autonomy \cite{Porathe2014}.
		
		\item \emph{Helmsman and lookout}. Both monitor the situation and execute commands from the officer of watch. A risk for them is information overload or underload \cite{Neerincx2008}.
	\end{itemize}
	
	\item Unmanned vessel
	\begin{itemize}
		\item \emph{Controller agent}. This agent is responsible for situational awareness. Thus getting safely from A to B. It will decide on the navigational strategy, it will do this based on the information acquired via all different means. Including newly developed communication protocols, computer vision and algorithms to transform sensor data into useful information. His duties are similar to the duties of the officer of watch as described for the manned vessel.
	\end{itemize}
	
	\item Other vessels
	\begin{itemize}
		\item \emph{Crew and pilots on nearby vessels}. They might want to know the intentions of other vessels to base their strategy on, without receiving all discussions, as this might result in information overload.
	\end{itemize}
	
\end{itemize}

\subsubsection{Secondary actors}
Beside the first group of actors, the new protocol could also influence others, besides the first groups of actors mentioned above. Although they are not within the scope of this first design cycle, they should be considered to avoid problems such as information overload on current communication channels or confusion. 
\begin{itemize}
	\item Only recipients
	\begin{itemize}
		\item Crew on vessels which are not travelling.
		\item Shipowners of unmanned vessels, monitoring vessel from a remote location.
	\end{itemize}
	\item Not within the scope of the research
	\begin{itemize}
		\item Vessel traffic controllers
		\item Crew which are in distress and require assistance
	\end{itemize}
\end{itemize}

\subsubsection{Goals}
The main goal is to ensure reliable sharing of information, without the risk for information overload or misunderstanding, so that manned ships will trust unmanned ships to choose the right strategy, as manned ships can be informed, using natural language describing the reasoning of unmanned vessels. 
During communication manned vessels should only be updated when requested or in case of an unusual activity that could affect their strategy. Manned vessels should thereby be aware when unmanned vessels desire more information to decide on the right strategy.
It might be possible to develop a protocol for communication with traffic controllers in later iterations, using the same philosophy. This communication becomes more critical when the development of a new system for traffic controllers will take too much time to develop or implement.

\subsubsection{Infeasible solutions}
A logical solution for unmanned ships would be to install a new system on every vessel. To implement this, it would mean that all ships that could encounter an unmanned ship will have to install this. It might be possible to make it obligatory via regulations, which will cost much time and money. This might delay the introduction of unmanned ships significantly.
Time and money are also the reason to use a \ac{no-UI}, as a GUI will require new screens or changes to the \ac{ECDIS} which is only possible when regulations are changed.

\section{Human factors}
\label{sec:human-factors}
When designing technology, two driving questions are crucial to be well-thought out: (1) What tasks and/or values is the user trying to accomplish and how can the technology support the user in doing so?, and (2) How can the technology be designed such that the user can work with the technology?

The Human Factors segment of the sCE method describes the available relevant knowledge about, for instance, human cognition, performance, task support, learning, human-machine interaction, ergonomics, etc. Note that we emphasise that this knowledge should be relevant for the problem and its design solution: the knowledge described here should lead to a better understanding of either (1) or (2). The three elements relevant to the human factors analysis are the human factors knowledge, measures, and interaction design patterns.

\subsubsection{Human factor knowledge}
Human factors knowledge describes available knowledge coming from previous research about solving the problems that have been identified during the problem analysis. The key problems relevant for human factors are information overload, situational awareness, autonomy, and learning a new protocol. The following questions should be answered:
\begin{itemize}
	\item When does information overload occur?
	\begin{itemize}
		\item In case of divided attention, there is a high risk for information overload and distraction by low priority messages. Therefore, the developed system should be context-aware so it can limit this risk by adapting the message to the situation \cite{Arimura2001}.
		\item Overload might appear due to a competition for the operator’s attention that is going on between multiple information items. If automated systems handle many tasks, the operator can deal with high workload circumstances but will suffer from severe underload during quiet periods, probably losing his or her situational awareness \cite{Neerincx2008}.
		\item The information acquired at one specific moment does not necessarily serve for high-level situation awareness, for the user needs to recall the previous related information to understand the situation thoroughly. But constantly providing information might not be the solution because there will be a huge risk for information overload. It is plausible to deliver the desired information for a future task, by detecting what future tasks should be. The user might still fail to keep pace in case of multi-threaded tasks\cite{Porathe2014}.
	\end{itemize}
	
	\item Which information do operators need for situational awareness?
	\begin{itemize}
		\item Understanding the current status of the system is not enough for full situational awareness. Expert decision makers must be able to project their understanding into the future. This projection enables experts to make the decision which results in the best options in the future. Projection requires good mental models of the dynamic relationships between the relevant parts of the environment over time. Experts focus a lot on creating their futures via present decisions. In turn, experts do form these decisions out of their comprehension of the likely interactions of all the elements they deem both relevant and important \cite{Gregory2010}.
		\item Situational awareness can be enhanced by feedback, perceived information from the environment, information from other agents, as well as remote sensors. \cite{Carver2007}
	\end{itemize}
	
	\item How is information perceived when acquiring it passively or actively?
	\begin{itemize}
		\item Attention profoundly modulates the activity of sensory systems, and this can take place at many levels of processing. Especially, imaging studies have revealed the greater activation of auditory areas and areas outside of sensory processing areas when attending to a stimulus \cite{Palmer2007}.
		\item Good teamwork involves anticipating the needs of teammates, and that means pushing information before operators request it. Therefore, if things are going well, there should be little need for pulling information. In this study task, participants were instructed to push information to others, and over time master the specific timing of information sharing to the intended recipient. Findings indicate that pushing information was positively associated with team situation awareness and team performance, and human-autonomy teams had lower levels of both pushing and pulling information than all-human teams \cite{Demir2017}.
	\end{itemize}
	
	\item What is needed for successful teamwork between human and a computer?
	\begin{itemize}
		\item  People need to understand what is happening and why when a teammate tends to respond in a certain way. They need to be able to control the actions of an agent even when it does not always wait for the human’s input before it makes a move, and they need to be able to reliably predict what will happen, even though the agent may alter its responses over time \cite{Bradshaw2003}.
		\item Effective team communication, a fundamental part of team coordination, is crucial for both effective team situation awareness and team performance \cite{Demir2017}.
	\end{itemize}
	
	\item Do people trust automated systems?
	\begin{itemize}
		\item When using automation, the role of the human changes from operator to supervisor. For effective operation, the human must appropriately calibrate trust in the automated system. Improper trust leads to misuse and disuse of the system. \cite{Walliser2011}.
	\end{itemize}
\end{itemize}

\subsubsection{Human factor measures}
Measures describe how to operationalise the quality of the intended behaviour or performance, i.e. how well is a user working with the design able to reach his/her objectives and what is the quality of the collaboration between the human worker and the technology?
\begin{multicols}{2}
\begin{itemize}
	\item Is the system used correctly?
	\item Will the protocol solve the problem of missing information?
	\item Does the protocol act as expected?
	\item What is the impact on attitude towards unmanned ships?
\end{itemize}
\end{multicols}

\subsubsection{Interaction design patterns}
\acf{IDPs} focus on the \acf{HCI}, such as usable interface design and control options. IDPs offer generic solutions to recurring HCI design problems that have been proven to be effective. Relevant IDPs are given in table~\ref{tab:IDPs}.
Keywords are often seen as the new buttons to interact. For the new protocol are message markers the keywords. These make it easier to train the conversational agent and clarify the available options for operators on manned vessels. Whereas the conversational skills of the agent are the core of general communication, the usage of other methods of communication will improve redundancy and effectiveness. 
A multimodal conversational agent also includes visible signals such as masthead light signals, flags and AIS messages. These will enable operators to see immediately if the vessel is unmanned.
The last two methods are for extreme situations: audible and distress signals. The agent for the unmanned vessel should understand what these mean and how to use them before manned vessels trust them.

\begin{table}[htbp]
	\centering
	\begin{tabular}{l|l}
		\toprule
		Radio communication & Usage of message markers and conversational agent \\
		Visible signals & Mast head signals, flags and \ac{AIS} information \\
		Audible signals & Horn and speakers \\
		Distress, urgency and safety signals & Flares and smoke \\
		\bottomrule
	\end{tabular}
	
	\captionof{table}{Interaction design patterns}
	\label{tab:IDPs}
\end{table}

\section{Envisioned technology}
The envisioned technology describes the available options of using existing technology and the need to develop novel technology to come to a system solution. The sCE method asks to specify what devices (hardware) and software the designers could use in the system design. In addition, for each type of technology, an argument should be provided as to why this technology might be of use and what the possible downsides might be of that specific type of technology.

The envisioned technology will use only existing systems to develop a \ac{no-UI}. Different systems that are currently used, are described in appendix~\ref{app:systems}. Below different systems and protocols are mentioned that can be used in the new protocol. Using these already existing systems will shorten the development, learning and implementation time. Table~\ref{tab:envisioned-technology} gives the used systems, equipment and protocols.

Using already existing protocols makes it easier to learn, such as \acf{SMCP} and \ac{COLREGs}. These systems make it also recognisable, which means that users will understand the benefits quicker. Show that it is useful and easy to use, as this is key to the acceptance of technology \cite{Davis1989}.

The type and amount of information presented to users must be tailored to the unique situation in which users use the information. Prior research on trust in automation found that providing human operators with information related to the reliability of an automated tool promoted more optimal reliance strategies on the tool. Further, information related to the limitations of an automated tool aids in trust recovery following errors of the automation. This added information appears to be useful in deciphering the boundary conditions under which the tools are more or less capable. Thus, providing human operators with information related to the performance of an automated tool appears to be beneficial \cite{Lyons2014}. Therefore it is beneficial for the cooperation between manned and unmanned vessels to show if it is an unmanned vessel, which will first be done using visible signals, and also at the start of radio communication. Telling the user that you are unmanned or automated also happens in industry projects, such as Google Duplex \cite{Nieva2018}.

\begin{table}[hbtp]
	\centering
	\begin{tabular}{l|l}
		\toprule
		Radio communication & Conversational agent \\
		& Negotiating agent \\
		& Usage of message markers \\
		& Availability on \ac{VHF} \\
		& Natural language variations on \ac{SMCP} \\
		& NATO phonetic alphabet and numbing \\
		& Addressed \ac{AIS} message to exchange information or interrogate \\
		\midrule
		Visible signals & Light signals \\
		& Mast head signals \\
		& Flags \\
		& Heading, position and movements \\
		\midrule
		Audible signals & Horn \\
		& Speakers\\
		\bottomrule
	\end{tabular}
	
	\captionof{table}{Envisioned technology}
	\label{tab:envisioned-technology}
\end{table}


\vspace{1.5cm}
\emph{In the next chapter} the system design specification will be presented for the envisioned technology. Most important will be the definition of the usage of radio communication and how this should be supported by other ways of communication, such as visible and audible signals.
