\chapter{Design evaluation}
The last part of the sCE method is the design evaluation. The design evaluation aims to test and validate the system’s design, or to discriminate between multiple design options, such that the current design can be improved upon in incremental development cycles. The sCE method describes three parts that are relevant with respect to the system evaluation: (1) the artefact, (2) the evaluation method, and (3) the evaluation results.

The artefact gives a short summary of the developed protocol and its workings, together with the tool in which it can be simulated. The evaluation method will be interviews with experienced seafarers. The aim of the interviews is to answer the following question:
\begin{quotation}
	\emph{Will a protocol based on existing maritime systems and communication protocols be sufficient to ensure safe navigation, while manned and unmanned vessels encounter each other?}
\end{quotation}
Different measures are used for validation and verification. Key variables are: Trust, situation awareness, effectiveness, efficiency and satisfaction. In this chapter these will be described in more detailed and what the results of the evaluation are.

\section{Artifact}
The artifact is an implementation or prototype that incorporates a given set of requirements, interaction design patterns, and technological means. This first iteration will aim at finding flaws based on expert knowledge. To acquire this a simulation environment will be used. This means that there is no implementation using hardware which might be used in latter stages. The simulation environment is created to test situations with experts. Where is verified if experts believe that the protocol is sufficient to get to the right decision, as described in chapter~\ref{ch:system-design}. 

To get the right feedback, several situations will be simulated. These situations are based on the accident reports as described in appendix \ref{app:accidents}, common situations around the port of Rotterdam and cases used in literature. The situations are simulated and visualized using the tool as described in appendix \ref{app:tool}. This will enable the experts to gain situational awareness and give useful feedback on the protocol. The protocol itself is mostly knowledge based and not automated during the evaluation. Thus the interviewer has to know the \acf{SMCP} relevant to the experiment and usage of systems like \acf{AIS} and \acf{ARPA}.

\section{Evaluation method}
The evaluation method can take many forms, such as a human-in-the-loop study, a use-case-based simulation, or an expert review. In this case a so-called Wizard of Oz evaluation is used. This technique enables unimplemented technology to be evaluated by using a human to simulate the response of a system. As the technology itself has not yet been implemented. The "wizard" simulates the system's responses in real-time. Using seafarers and the Wizard of Oz method, an expert evaluation can be acquired on the proposed system without implementing it.

\subsection{Experiment design}
To do the experiment, a participant is needed to have the role of \acf{OoW} and tools to execute. During the experiment there will be different variables, which have to be taken into account to be able to draw the right conclusions. These are described in this section.

\subsubsection{Participants}
The participant is in this case the \acf{OoW}. The experiment will be done with at least 10 different participants. The formal requirements are as follows:
\begin{itemize}
	\item Nationality: Dutch, due to location of experiment.
	\item License: Completed training as a maritime officer.
	\item Experience: At least 3 years of experience as seafarer.
	\item Attitude towards autonomous shipping: Both positive and negative.
	\item Age: 25-60
\end{itemize}
The different participants will receive different scenarios of the same situation in different orders. Such that they are counter-balanced.

\subsubsection{Tools}
Beside the participants, tools are needed to do the experiment. The tools needed are:
\begin{itemize}
	\item Laptop with second screen to show simulation environment.
	\item Questionnaire to be used before, during an after the experiment.
	\item Room without distractions to do the experiment.
	\item Possibility to store and later process actions during the experiment.
\end{itemize}

The simulation environment is discussed in more detail in appendix~\ref{app:tool}. The simulation environment is shown in figure~\ref{fig:printscreen-tool}. Where the map is shown on the left, which is updated regularly. The side-bar can be used to control the selected vessel. In the status-bar at the bottom of the screen, information is given on possible actions, errors and status of simulation.

\begin{figure}[hbp]
	\centering
	\includegraphics[width=\textwidth]{printscreen-maasgeul.png}
	\caption{Simulation environment}
	\label{fig:printscreen-tool}
\end{figure}


\subsubsection{Dependent variables}
During the experiment different variables will be evaluated. These are based on the human factor measures as described in section \ref{sec:human-factors}. This will be a combination of both quantitative and qualitative measurements. Thus combining numerical values with non-numerical arguments. 
Using the variables can be concluded if the system acted as expected and will result in safe navigation when using existing protocols. The first variables relate to experiment:
\begin{description}
	\item [Trust] The system acts not only out of self-interest, but in order to acquire a pareto-optimal solution. This means that the system is honest, approved by well-known institutions and ensure fairness. This is only possible when the system is able to communicate correctly, and others are confident that the system did indeed take the right decisions \cite{Ozawa2013}.
	This is measured using a survey and amount of cooperation with the system.
	\item [Situation awareness] The perception of environmental elements and events with respect to time or space, the comprehension of their meaning, and the projection of their status. This is measured by asking if the participants have noticed some changes in the simulation together with an observer rating system \cite{Naderpour2016}.
	\item [Performance] Simple evaluation if the participant acted correctly. Thus did he safely navigate the vessel by making the right decisions. For more detailed results, the \acf{CPA} can be used.
	\item [Satisfaction] Pleased feeling, as the participant likes the way he acted and how the system works. This will also be measured using a questionnaire.
\end{description}

Based on the variables measured in the experiment, can conclusions be drawn about the used protocol for communication. These are based on the following variables:
\begin{description}
	\item [Effectiveness] The protocol is prompt and thorough.
	\item [Efficiency] The protocol is easy to understand and use. Thereby does it not cost much effort and time to use it during operation.
\end{description}

\subsection{Experiment task}
\label{ssec:experiment-task}
During the experiment will the parcipant act as an \acf{OoW} in different situations. The duties of the \ac{OoW} are to keep watch and navigate the vessel. While keeping a watch on the bridge he is the representative of the ship’s master and has the total responsibility of safe and smooth navigation of the ship. This means he has to follow a proper navigation plan to avoid any kind of collision according to COLREGs. Thereby is he aware of ship’s speed, turning circles, and ship handling characteristics. He also communicates with other vessels when that is necessary.

More specific will he sail a vessel in different scenario's. It depends on the scenario how and if the participant can communicate with other vessels. He will do this by directing a vessel on a 2D-map as shown in the simulation environment. He will have access to the same information as any other bridge, also presented in a similar way, except for the visual 3D-view. But this lack of information is compensated by more reliability for the ECDIS and Radar.
Within the experiment, tests will be done for different situations and scenarios. The resulting strategies are listed below.
\begin{multicols}{2}
\begin{enumerate}
	\item Crossing situation at North-Sea
	\begin{enumerate}[label=(\Alph*)]
		\item Follow COLREGs strictly
		\item Cross in front
		\item Cross at the back
	\end{enumerate}
	\item Entering Maasgeul from Maasvlakte
	\begin{enumerate}[label=(\Alph*)]
		\item Cross in front
		\item Cross at the back
		\item Pass without crossing
	\end{enumerate}
\end{enumerate}
\end{multicols}

\subsubsection{Crossing situation at North-Sea}
The first situation for the experiment is a crossing situation based on the accident between MV ARTADI and MV ST-GERMAIN (appendix~\ref{sec:artadiVSst-germain}). Where both ships followed COLREGs, but due to a lack of communication and wrong presumptions on the intentions, did the accident occur.

The traffic in this simulation consists of three ships: a 250-meter tanker (GULF VALOUR), a 140-meter cargo vessel (ASTRORUNNER) and a 400-meter container vessel (EMMA MAERSK). This situation is shown in figure~\ref{fig:crossing-dover}. The relevant information on these ships is given in table~\ref{tab:info-Dover}.

\begin{table}[hp]
	\centering
	\begin{tabular}{l | r r r l}
		\toprule
		 & GULF VALOUR & ASTRORUNNER & EMMA MAERSK & \\
		\midrule
		Length 	& 249.0	& 141.6	&  397.7 & m \\
		Width 	& 48.0	& 20.6	&  56.4 & m  \\
		Draft 	& 13.2	& 6.5	&  12.6 & m  \\
		Deadweight & 114900 & 9543 & 156907 & m \\
		Type 	& Oil tanker	& General cargo vessel	&  Container vessel & \\
		\midrule
		Position& [-1400, -1400]	& [3250, 0]	&  [400, 2400] & m \\
		Speed 	& 16.0	& 15.2	&  12.0 & knots\\
		Course 	& 45	& 278	&  225 & degrees \\
		Previous port & Singapore & Zeebrugge & Rotterdam \\
		Next port & Rotterdam 	& Dover	& Hongkong & \\
		Direction & North-east	& West	& South-west & \\
		\bottomrule
	\end{tabular}
	
	\captionof{table}{Relevant information crossing at North-Sea}
	\label{tab:info-dover}
\end{table}

\begin{figure}[p]
	\centering
	\includegraphics[width=\textwidth]{situation-dover.png}
	\caption{Situation sketch crossing situation North-Sea}
	\label{fig:crossing-Dover}
\end{figure}

\subsubsection{Entering Maasgeul from Maasvlakte}
The second situation is a common situation at the port of Rotterdam. The situation is based on the description by Pilots from 'Nederlands Loodswezen'. The big challenge here is that ships are accelerating and decelerating. Therefore do traffic controllers notify ships about others intentions. But in the case as presented does this not always happen, or too late.

The traffic in this simulation consists of three ships: a 250-meter tanker (GULF VALOUR), a 140-meter cargo vessel (ASTRORUNNER) and a 140-meter ro-ro cargo vessel (ANGLIA SEAWAYS). This situation is shown in figure~\ref{fig:entering-maasgeul}. The relevant information on these ships is given in table~\ref{tab:info-Rotterdam}.

\begin{table}[hp]
	\centering
	\begin{tabular}{l | r r r l}
		\toprule
		& GULF VALOUR & ASTRORUNNER & ANGLIA SEAWAYS & \\
		\midrule
		Length 	& 249.0	& 141.6	&  142.4 & m \\
		Width 	& 48.0	& 20.6	&  23.0 & m  \\
		Draft 	& 13.2	& 6.5	&  5.0 & m  \\
		Deadweight & 114900 & 9543 & 4650 & m \\
		Type 	& Oil tanker	& General cargo vessel	&  Container vessel & \\
		\midrule
		Position& [-1372, -1377]	& [-3090, 1395]	&  [3000, -550] & m \\
		Speed 	& 7.8	& 13.4	&  10.3 & knots\\
		Course 	& 98	& 114	&  291 & degrees \\
		Origin & Princess Arianehaven & North-Sea & Vulcaanhaven \\
		Destination & North-Sea & Beneluxhaven & North-Sea \\
		Direction & Leaving & Entering & Leaving & \\
		\bottomrule
	\end{tabular}
	
	\captionof{table}{Relevant information entering Maasgeul}
	\label{tab:info-Rotterdam}
\end{table}

\begin{figure}[p]
	\centering
	\includegraphics[width=\textwidth]{situation-maasgeul.png}
	\caption{Situation sketch port of Rotterdam}
	\label{fig:entering-maasgeul}
\end{figure}

\subsection{Experiment procedure}
To execute the experiment. Several steps are taken together with the \acf{OoW}:
\begin{enumerate}
	\item Explain how the OoW can take actions, such as steering, change speed, set way-points or engage in communication.
	\item Ask general questions on attitude and basic information.
	\item Explain situation to OoW in a similar way to common hand-over. Only describe relevant issues for navigational duties.
	\item Start playing simulation.
	\item Depending on the simulation, let autonomous ship take actions or wait for the OoW to engage in communication.
	\item End simulation.
	\item Question OoW why which decision was made.
	\item Question OoW on several "what if"-scenarios and how that would have changed its actions.
	\item Repeat step 3-7 for more situations.
	\item Question OoW about advantages and challenges of protocol.
	\item Question OoW about human factors.
\end{enumerate}

\subsubsection{Explanation (1, 2)}
The participant is not explicitly informed about the exact purpose of the research. It will however get a short introduction on how to use the simulation environment. This includes the commands it can give as an officer of watch. It should be easy to use, as similar action should be executed when operating a vessel. Thereby is some information about the participant acquired:
\begin{itemize}
	\item Which licenses do you have?
	\item What is you experience?
	\item What do you expect from the developments towards autonomous shipping?
	\item What do you see as the biggest challenge for introducing autonomous and unmanned vessels?
\end{itemize}

\subsubsection{Situations and scenarios (3, 4, 5, 6, 9)}
The next steps are repeated several times for the different situations. The situations are described in section \ref{ssec:experiment-task}, thereby does it depend on the experiment order in which situation it is possible to communicate:
\begin{enumerate}
	\item Crossing situation at Nort-Sea
	\item Entering Maasgeul from Maasvlakte
\end{enumerate}

\subsubsection{Relevant questions for situation (7, 8, 9)}
To gain insight into the quality of the experiment and effectiveness of the protocol, different questions will be asked. Here there is also a link with the decision model as described in section~\ref{ch:model}:
\begin{itemize}
	\item What type of situation is this?
	\item Which criteria are relevant?
	\item Which strategy did you choose?
	\item Which actions were taken?
	\item Did other ships behave as expected?
	\item Were you in control over the situation?
	\item Did you miss any information to come-up with the right strategy?
	\item Was it necessary to communicate?
	\item If there was communication, was this as expected?
	\item Can you explain why?
	\item Would you have acted differently is you knew there was a human officer of watch?
\end{itemize}

\subsubsection{General questions on protocol, autonomous shipping and human factors (10, 11, 12)}
After running the different situations an interview is held. This part of the experiment is intended to answer the following questions from the participant perspective and explain the purpose of this research:
\begin{itemize}
	\item Do you expect to trust autonomous ships more, when they would pass the Turing test?
	\item Do you want to know if a ship is unmanned?
	\item Are ship's horns still used to communicate intended manoeuvers?
	\item Do you still use \acf{SMCP}  consciously?
	\item Is the protocol around SMCP easy to use?
	\item Is the protocol around SMCP easy to learn?
	\item Is the protocol around SMCP a good protocol?
\end{itemize}

\section{Evaluation results}
The evaluation results describe the outcomes of the test. Because of the iterative and rapid research cycles, the evaluation does not necessarily include all requirements/claims/use cases available in the system specification. Oftentimes the evaluation investigates a subset of the system specification. Therefore, it is often useful to also specify what claims were tested, with the use of what evaluation method, and what artefact was used during the evaluation (i.e. which requirements, technology, interaction design patterns were included in the artefact).

\subsection{Outcomes}
\todo{Describe outcome of experiment}

\subsection{Conclusions}
\todo{Answer research question for CS}

\begin{quotation}
	\emph{Are existing systems and protocols for communication, sufficient to ensure that manned and unmanned ships can operate side by side safely?}
\end{quotation}