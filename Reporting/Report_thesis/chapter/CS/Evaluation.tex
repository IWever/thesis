\chapter{Design evaluation}
The last part of the sCE method is the design evaluation. The design evaluation aims to test and validate the system’s design, or to discriminate between multiple design options, such that the current design can be improved upon in incremental development cycles. The sCE method describes three parts that are relevant with respect to the system evaluation: (1) the artefact, (2) the evaluation method, and (3) the evaluation results.

\section{Artefact}
The artefact is an implementation or prototype that incorporates a given set of requirements, interaction design patterns, and technological means.

\section{Evaluation method}
The evaluation method can take many forms, such as a human-in-the-loop study, a use- case-based simulation, or an expert review.

Question if it should pass turing test, which Human Factor measures are possible.

\section{Evaluation results}
The evaluation results describe the outcomes of the test. Because of the iterative and rapid research cycles, the evaluation does not necessarily include all requirements/claims/use cases available in the system specification. Oftentimes the evaluation investigates a subset of the system specification. Therefore, it is often useful to also specify what claims were tested, with the use of what evaluation method, and what artefact was used during the evaluation (i.e. which requirements, technology, interaction design patterns were included in the artefact).