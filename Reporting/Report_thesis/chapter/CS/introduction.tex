In the previous part a decision model is described which tries to avoid communication between vessels by taking the right decision at the right moment to ensure safe operation. However this is not always possible. This is often the case due to missing information, or a lack of understanding about the intentions of other vessels. With manned ships, verbal communication via \ac{VHF} is used to acquire the missing information or discuss strategies with other ships. This is likely also necessary when unmanned vessel operate between manned vessel. In this part the development process of such a protocol is discussed and the relevance will be proved using an experiment, answering the following question:

\begin{quotation}
	\emph{Are existing systems and protocols for communication, sufficient to ensure that manned and unmanned ships can operate side by side safely?}
\end{quotation}

A protocol defines the format and the order of messages exchanged between two or more communicating entities, as well as the actions taken on the transmission and/or receipt of a message. Where it is most straight forward to use verbal communication, will it not be limited to this, as it might result in better situational awareness for both the manned and unmanned to use other means, such as visible signals or text messaging.

The communication which currently happens between vessels is most common when \ac{COLREGs} do not result in clear strategies, or when intentions are not clear. Other communication which will not be within the scope of this research, such as the communication with traffic controllers and how conversations are interpreted by other vessels.
Thus this research is a starting point to develop a full protocol needed for the acceptance of unmanned vessels.

Using the \acf{sCE} method, this protocol can be developed using an iterative process, where a  requirement baseline is continuously refined by reviews and prototype evaluations. How to apply \ac{sCE} is described by Neerincx and Linderberg \cite{Neerincx2012}.

The first step is to create a foundation. The current situation of the problem is discussed, which have to be solved. Thereby considering existing knowledge which might be relevant to solve the problem and finally in short the envisioned technology. The next step is to define the system design specification. In which scenarios are described which show how the problem is solved. From this can be extracted what should be designed and why this is done. Using this a design is made which is being evaluated to make improvements in next iterative steps.