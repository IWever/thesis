\chapter{Conclusion}
This part has shown that it is possible to solve the challenge of communication and ensure safety by enabling unmanned vessels to communicate with manned vessels. We solved this challenge by designing a protocol for communication, based on existing maritime systems and protocols, using the \acf{sCE} method. Where the first phase includes a description of operational demands, relevant human factors and envisioned technologies. The second phase consists of design-scenarios and use-cases, which result in a list of requirements. These requirements are used to define an ontology for the system and a description of the capabilities of the system. With experts is evaluated if this is indeed enough to ensure that unmanned vessels can operate safely between manned vessel. All this combined will answer the question:

\begin{quotation}
	\emph{Will a protocol based on existing maritime systems and communication protocols be sufficient to ensure safe navigation, while manned and unmanned vessels encounter each other?}
\end{quotation}

Using \acf{SMCP} as the core for a new protocol, will ensure safe navigation. As the advantage of \ac{SMCP} is that it is already optimized for radio communication. The clear speech acts will make it more easy to automate the communication. Which means than unmanned vessels are able to use the same protocol as is currently used by manned vessels. Thereby did experts confirm that it is an idiot proof system, which is easy to use and learn, certainly when the speech recognition tool will be able to handle natural language variations on \ac{SMCP}.
This will be more easy when officers use the protocol as it is designed. After recent accidents between manned vessels was already decided that pilots and masters should use English speech and \ac{SMCP} in all cases, to avoid misunderstanding. Avoiding misunderstanding will result in better situational awareness, which results in safer navigation.
Thus it is possible to ensure safe navigation when manned and unmanned vessels meet, by developing a protocol. The designed protocol uses existing maritime systems and communication protocols. For this design is the \acf{sCE} method used. Experts do agree that this is indeed the way for designing such a protocol.

The next step is to map different speech acts, to specific situations. This is only possible when situations can be identified by unmanned vessels in a systematic way. Which means that the recognition of situations is one of the key challenges to enable unmanned vessels. 