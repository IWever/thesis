% What is the current state of the system
\chapter{Current systems to improve safety}
\label{ch:systems}
To improve the system and procedures, the first step is to know what is currently available. Thereby should be known which information these systems can give, and how these systems are used. Therefor a description is given of the current means for communication. These can all be related to different elements of the bridge, to make it more tangible.

\section{Means of communication}
Communication is a very broad concept and comes in many forms, as it includes everything which enables the exchange of information. The main reason to communicate is improve the safety of life at sea. More specifically to show intentions or ask for aid. The main means of communication are nowadays:

\begin{itemize}
	\item Visible signals
	\begin{itemize}
		\item Positions
		\item Change of heading
		\item Light signals
		\item Flags or symbols
	\end{itemize}
	\item Availability on \ac{VHF}
	\item Exchange of information via \ac{AIS}
	\item Horn
\end{itemize}
\todo{it this the complete list?}

http://www.bigoceandata.com/news/brief-history-ais/

\section{The bridge}
According to DNV-GL, the bridge of a vessel can be separated into four elements. The human operator, procedures,
technical system and the human-machine interface. As shown in figure \ref{fig:Bridge-system-elements} \cite{DNVGL2011}. This chapter will focus on the technical system and human-machine interface. Thereby a separation will be made between the instruments available, the information which can be deducted from this and how these should be used. 
\todo{make a cleaner infographic showing the different parts of the bridge}

\begin{figure}[H]
	\centering
	\includegraphics[width=0.5\textwidth]{Bridge-system-elements.png}
	\caption{Bridge system elements}
	\label{fig:Bridge-system-elements}
\end{figure}

The ship’s navigation bridge shall enable the officer in charge of the navigational watch to perform navigational duties unassisted at all times during normal operating conditions. He shall be able to maintain a proper lookout by sight and hearing, as well by all available means appropriate in the prevailing circumstances and conditions. To make a complete assessment of the situation, including the risk of collision, grounding and other hazards to navigation.
\todo{Is this relevant, if yes, describe different parts}

\subsection{Technical system}
At least the following instruments and equipment shall be installed \cite{DNVGL2011}: 
\begin{multicols}{2}
	\begin{itemize}
		\item Navigation radar with radar (\ac{ARPA})
		\item Propulsion control
		\item Manual steering device
		\item Heading control
		\item Other related \ac{UID}s
		\item Electronic Chart Display Information System (ECDIS)
		\item Steering mode selector switch
		\item VHF unit
		\item Whistle and manoeuvring light push buttons
		\item Internal communication equipment
		\item Central alert management system
		\item General alarm control
		\item Window wiper and wash controls
		\item Control of dimmers for indicators and displays
		\item Propulsion
		\item Emergency stop machinery
		\item Gyrocompass selector switch
		\item Steering gear pumps
	\end{itemize}
\end{multicols}
\todo{Categorize into communication, situational awareness, etc.}

What do regulations say about systems which should be on board. \\
How much automation is used now, plotter, combining ECDIS and Radar. \\
Collision avoidance and other warning systems, also not related to navigation.


\subsection{Man/machine interface}
How is information used. Mention parts on information overload. Thereby also the difference between ships. Officer walking around on big ship, vs small bridge on a tug for example.

\subsection{Procedures}
Training, education and protocols. How are procedures developed, use maritime support knowledge on writing them.

\subsection{Human operator}
How does it go in practice, experience. Relation with shifts and information overload, or the other side with a boring shift, when out of the blue full attention is desired. Importance of culture to the decision making process.

\section{Relevant systems for autonomous shipping}
\label{sec:relevant-systems}
Not everything is relevant to other ships, in order to determine the possible decisions. This is only: information from radar, information available in AIS (where reliability must be considered), warning systems, etc...

Mention the problems with ECDIS/AIS (old information, not correct as mentioned by loodswezen), as I refer to this in snapshots scenario description.