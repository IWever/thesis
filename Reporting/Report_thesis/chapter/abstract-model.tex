\chapter{Abstract model}
To form the set of possible decisions, different factors should be taken into account. To get insight in these factors, an abstract model is developed. This abstract model will link the creation of a mental model to the physical world, which is key in creating situational awareness. From this model different subjects can be derived, relevant to making the right decision at the right time, which result in the correct strategy to cope with the specific scenarios.
In figure \ref{fig:abstract-model} a visual of the abstract model is shown. In this chapter different parts of the abstract model will be discussed in more detail.

\begin{figure}[hb]
	\centering
	\makebox[\textwidth][c]{
	\includegraphics[width=1.18\textwidth]{abstract-model.png}
	}
	\caption{Abstract model}
	\label{fig:abstract-model}
\end{figure}
\todo{Make model in same style as infographic and situation sketches, using blocks as mentioned below, also use notes from Soevereign meeting}

\section{Ship characteristics}
The main block on manoeuvring capability depends on the ship characteristics. Where there is an influence on both feeling of safety, and the physical limitations. For example a ship crossing another vessel could keep sailing straight as long as possible and than make a hard turn, but it rather changes its course earlier in a way where less loss of speed occurs. With a small ship these hard turns are much easier and will most likely occur more often.

Therefore the capability of the ship is split up in three categories, expected location, likely change of course and physically possible route based on physics laws.
As a full CFD simulation is hard to do for every ship which is encountered, models will be used to make estimations quickly. Thereby is it important to notice that a high accuracy is not necessary, as those extreme movements have a very low probability to occur. Being a bit conservative here will help to allow for all different scenario's to be tested.

\section{Influencing factors}
The same goes for the safe motion parameters. There are regulations on minimal passing distances and how close you should come to others. But good seamanship would be to make intentions clear earlier, so that critical situations can be avoided. Applicability of rules depends on the ship characteristics. Just as with the natural laws, the probability of getting into the extreme situations where regulations are followed but still problems might occur are not likely.
The model from Szlapczynski is used, to set the estimated safety domains for each ship. Leading to courses and speeds which are safe to sail \cite{Szlapczynski2017}. While also incorporating \ac{COLREGs} and arbitrary distances for well-clear, which is the distance where everyone feels safe.
The \ac{COLREGs} are also relevant in areas around ports, where traffic separation schemes are used. These should be followed by different types of vessels. This will influence the most likely position for a vessel to be. Incorporating information from traffic controllers will thereby lead to the most likely expected route. The same goes for areas with multiple objects on the map, where ships should follow lanes marked by buoys, and avoid quays, fishing area's, no-go zones, etc.
Combining this with ship characteristics will lead to a map where for each vessels is shown, where another ship will likely be. When looked at the highest probability this will show the most likely route for each vessel. Which can be used to advise on the route planning of the own vessel. As the probabilities combined with a threshold will result in a set of possible decisions.

\section{Information and communication}
Already mentioned is the impact of other vessels and objects on the probability that a ship will be at a specific location. But to know this impact, information has to be gathered from different parameters. Thereby should be looked if the current parameters are sufficient and can be trusted, or if other equipment or instruments should be installed.

Currently vision and radio communication are very important ways to gather information. These types of information are difficult to gather using computers, currently many researches happen on for example Lidar and other methods. Therefore this research will focus on what information is needed, not how it is gathered. Although developments within technologies to gather this information will be taken into account to develop eventually the strategies. For example related to \ac{AIS}, \ac{VDES} and different protocols.

\section{Situational awareness}
When combining information, communication, characteristics and possible decisions an image of the situation can be formed. This will be interpreted in a way which enables the right decision making for route-planning. This planned route can be used on autonomous vessels, or be used to advise the crew which decision they should make and what information they should share with others. Mental models will be used to validate if indeed the different vessel have the right information and if enough information is shared. The results from these blocks will be verified with seafarers to validate that the theory corresponds with operations.
