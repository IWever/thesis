With the developed application models can be tested to get insight into real-life situations. As mentioned in section \ref{sec:challenges-future}, is one of the challenges for unmanned ships to avoid complex situations. To acquire this decisions must be taken in time. This is only possible when an estimation can be made what other vessels will do. The time-domain model for decision making makes an estimation of the time needed to make a decision by taking into account the available information and own ship characteristics. At first this will result in a set of decisions for each vessel, this is followed by a prediction which decision other ships will take. Based on this a decision can be made for the own vessel, when decisions should be taken to ensure an unimpeded voyage. Thereby should be noted that the current systems for sharing information are used. Variations on this will be made in the next part.

In order to accomplish this, specific modules have to be developed. These are discussed, followed by the scenarios which will be simulated. Simulating these scenarios will result in strategies, these are validated in order to validate the model in general. This will result in lessons which can be applied in real-life situations to avoid collisions in the future.