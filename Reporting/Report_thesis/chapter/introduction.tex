\chapter*{Introduction}
\label{sec:introduction}
\addcontentsline{toc}{chapter}{Introduction}

Considering the major projects within the maritime industry, can be assumed that unmanned and autonomous shipping are getting closer. By taking out seafarers from the chain of command, human errors can be avoided. However taking out humans will lead to more challenges when it comes to communication. Communication is used to share information on the intentions or discuss future actions. Thus to ensure the safety of all vessels; manned, unmanned, remote, automated and autonomous. The question which should be asked is: How do other (manned) vessels know the intentions of other vessels, while there is limited communication possible? This is most relevant when sharing short term intentions, as decisions must be made within minutes or even shorter. On the change course or speed, in order to avoid critical situations.

\subsubsection*{Problem statement}
A computer might like to have as much information as possible. But in case of manned vessels this is not practical, as this will results in an information overload of the crew and communication channels \cite{CCNR2017}. Currently is secured in two ways, how intentions of other vessel can be acquired. First and foremost are the \ac{COLREGs}\cite{IMO1972}, rules applicable to all vessels, as these rules are concrete these can be programmed and used also by autonomous vessels. Examples are to stay on starboard side of the shipping lane and to not cross other ships with small relative angle. However in critical situations such as the entering of harbors or in busy parts of the world, the VHF radio is also used to ensure that intentions are clear.

To make autonomous shipping possible, autonomous vessels should know how to communicate their intentions. Doing this by behaving in a predictable manner and without overloading the different VHF channels (including AIS). Other options are to develop a new system and stay away with autonomous vessels or from other vessels. This will be much harder to implement, but still the way intentions are shared should be optimized. This is done by making sure that the radio is used correctly, clear signs which can be interpreted from the behavior and data exchange in other ways. Which all ensures that other vessels know enough about the intentions to anticipate correctly, without overloading the communication channels.

To do this optimization, there should be taken into account that different vessels have different expectations for other vessels, characteristics which determine the manoeuvrability and strategies for specific situations. As a long and heavy ship will mostly go straight ahead at a similar speed, thus resulting in a predictable path which will be sailed. While a small tug boat might move around much more. And therefore be in situations which would be a collision course for others. While it is necessary to think minutes ahead with a long and heavy ship, this is much shorter for a small tug boat. This time-domain for decision making depends not only on the ship characteristics, but also the waterway characteristics such as depth, traffic separation schemes and harbor entrances.

The first option to optimize the communication of intentions is by taking out radio communication completely. Combining the time-domain for decision making of your own vessel, and estimating when others need to make a decision and which decisions they might make. It is possible to determine when and which decision will lead to an unimpeded voyage for yourself. Where an unimpeded voyage means that no communication via VHF-radio is needed to ensure a safe passage with no (perceived) risk.

By adjusting the course based on this information, the need for radio communication is taken out. This will mean less pressure on the crew and less challenges when introducing autonomous vessels.
However, to form a register of possible decisions, there is also input needed. These still have to be received via other means than radio communication. This is done using for example sensors like radar, communication via AIS or a new system. The raw data from these sources will together form the information needed to create the decision domain. The first step to known which sources are needed, is to know which information is desired. This will also be part of this research, beside looking into the forming a the decision domain.

\subsubsection*{Research questions}
To enable the step towards autonomous shipping, a start is made with a qualitative research. Focused on forming the above mentioned register of possible decisions with the right information. And from here deciding what the decisions should be. By looking into which information is most relevant for the different vessels. This will be done from two perspectives within the domains of Maritime Technology and Computer Science. Resulting in two research questions:

\begin{quotation}
	\emph{How do ship characteristics influence the time-domain for decision making to ensure an unimpeded voyage?} 
\end{quotation}

\begin{quotation}
	\emph{Which information must be shared between vessels, to ensure that possible decisions of other vessels can be predicted correctly?}
\end{quotation}

\subsubsection*{Report structure}
Within the project, several steps will be taken. At first more background is given on the current state of the maritime industry in respect to autonomous sailing and ways of communication. Thereby mentioning how the knowledge developed in this research, will help to tackle challenges in current projects.
Next a more detailed plan of approach for the research is given, including an abstract model which shows how ship characteristics and gathered information influence the possible decisions and ability to estimate intentions.

This model can be used to form strategies to cope with specific situations. To validate if it is possible to use this model, different scenarios are defined which are used as test-cases. These scenarios will be tested within a simulated environment. Requirements for this simulation are first defined, which lead to user stories and an implementation.
This simulation is finally validated together with seafarers on the quality of the communication and if ships act realistically within the different scenarios. This is used to give an qualitative advice on how to optimize communication, or more specifically which information should be shared.
