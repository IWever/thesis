\chapter*{Introduction}
\label{sec:introduction}
\addcontentsline{toc}{chapter}{Introduction}

More people start to believe in the possibilities of unmanned and autonomous shipping, as there are many projects which try to develop the necessary technologies \cite{SMASH2017} \cite{Eriksen2017} \cite{MUNIN2016} \cite{Sames2017} \cite{RollsRoyce2015} \cite{Waterborne2016}. The major reason for the development of autonomous ships are the benefits for operational expenses. But one of the major reasons for acceptance is the mitigation of human errors. Which can be achieved by taking out humans from the chain of command, this will however introduce new challenges \cite{Saarni2018}. In chapter \ref{ch:future} these challenges and corresponding projects are discussed. One of the less explored challenges is the issue of communication. This becomes harder between manned and unmanned vessels. Communication is used to share information on the intentions or discuss future actions. At the moment there is no protocol which can be used by unmanned vessels. However this is necessary to ensure the safety of all vessels: manned, unmanned, remote, automated and autonomous. To solve this challenge, there are two possibilities: Avoid the need for communication, or develop a new communication protocol.

\subsubsection*{Context}
In many situations is the \acf{COLREGs} sufficient to determine the intentions of other vessels \cite{IMO1972}. They can be seen as ship separation rules, which guide all vessels to make early and correct alterations to the course. It will take more time to assess the situation when using \ac{VHF}. Meaning there is less time to act, which limits the possible strategies. These rules do also apply to autonomous ships, and thus can be used when manned and unmanned vessels meet.
Examples are to stay on starboard side of the shipping lane and not to cross other ships with small relative angle. Accidents do still occur, even when both vessels follow the \ac{COLREGs}, such as the accident between Artadi and St-Germain as described in appendix \ref{sec:artadiVSst-germain}. This occurs more often in complex situations, such as a harbor approach.

The risks when \ac{COLREGs} are not sufficient can be mitigated by taking decisions well in advance. What well in advance means, depends on the manoeuvrability of a vessel. A cargo vessel will follow common paths, while a small tug boat might move around much more. Which results in more false positives on potential collisions with other vessels. The manoeuvrability also means that it is necessary to think ahead several minutes with a large ship, while this is much shorter for a small tug boat. This time-domain for decision making depends not only on the ship characteristics, but also the waterway characteristics. In chapter \ref{ch:model} this will be discussed in more detail, examples are the depth, traffic separation schemes and harbor entrances.

\subsubsection*{Research questions}
This research will be a start, to solve the challenge of communication between manned and unmanned vessel without introducing new systems to the bridge of manned vessels. This can be used as a foundation to build a system for decision making which ensures safe operation of both manned and unmanned vessels. The parts relate to different research domains: Maritime Technology and Computer Science. Which discuss two different research questions:

\begin{quotation}
	\emph{How do ship manoeuvrability characteristics influence the domain for decision making, to ensure that the chosen strategy will result in a passing distance which does not require communication?} 
\end{quotation}

\begin{quotation}
	\emph{Are existing systems and protocols for communication, sufficient to ensure that manned and unmanned ships can operate side by side safely?}
\end{quotation}

\subsubsection*{Report structure}
This research will be separated into three parts. In those parts, the challenge of communication between manned and unmanned vessels is considered. The first part gives a more detailed context of autonomous shipping. Using a decision-model and this context, can be shown why communication is a challenge.

Part \ref{part:MT} aims to determine how communication can be avoided. This is possible when ships stay away from each other. To accomplish this, decision should be taken in time. Thus studying the time-domain for decision making. This depends on the moment decision are made in critical situation. Using manoeuvring models these situations are simulated, to determine the time-domain and if this can be improved to ensure that communication is needed in fewer situations.

Part \ref{part:CS} will focus on the critical situations where communication is a must. Currently there is not yet worked on a communication protocol between manned and unmanned vessels. To ensure this will not delay the implementation of unmanned autonomous vessels. A new communication protocol is needed, based on existing systems and protocols. The aim of this part, is to validate if this is sufficient to ensure that the right information is shared. This is done by using the \acf{sCE} method, to quickly define a new protocol which can be tested in a simulation environment.