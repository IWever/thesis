\chapter*{Introduction}
\label{sec:introduction}
\addcontentsline{toc}{chapter}{Introduction}

More people start to believe in the possibilities of unmanned and autonomous shipping, as there are many projects which try to develop the necessary technologies \cite{SMASH2017} \cite{Eriksen2017} \cite{MUNIN2016} \cite{Sames2017} \cite{RollsRoyce2015} \cite{Waterborne2016}. One of the major reasons mentioned is the mitigation of human errors, in autonomous shipping this is done by taking out seafarers from the chain of command, this will however introduce new challenges. One of the less explored challenges is the issue that communication becomes harder between manned and unmanned vessels. Communication is used to share information on the intentions or discuss future actions. At the moment there is no protocol which can be used by unmanned vessels. However this is necessary to ensure the safety of all vessels: manned, unmanned, remote, automated and autonomous. To solve this challenge, there are two possibilities: Avoid the need for communication, or develop a new communication protocol.

\subsubsection*{Context}
A computer works most efficient when it has all information in a structured way. Humans however are better with limited and unstructured data, to avoid information overload of the crew and communication channels \cite{CCNR2017}. This will result in a challenge when manned and unmanned vessels have to communicate.
In many situations are \acf{COLREGs} sufficient to determine the intentions of other vessels \cite{IMO1972}. They can be seen as ship separation rules, which guide all vessels to make early and correct alterations to the course. As it is safer to make early adjustments to course or speed, than to spend too much time using \ac{VHF}, \ac{ARPA} or \ac{ECDIS} to make an assessment. These rules do also apply to autonomous ships, and thus can be used when manned and unmanned vessels meet.
Examples are to stay on starboard side of the shipping lane and not to cross other ships with small relative angle. However in critical situations such as the entering of harbors or in busy parts of the world, are the \ac{COLREGs} not always sufficient.

To ensure the safety of all vessels, must decisions be taken well in advance. This depends on the manoeuvrability of the vessel. A cargo vessel will for example, follow a fairly predictable path. While a small tug boat might move around much more. Which results in more false positives on potential collisions with other vessels. The manoeuvrability also means that it is necessary to think ahead several minutes with a long and heavy ship, while this is much shorter for a small tug boat. This time-domain for decision making depends not only on the ship characteristics, but also the waterway characteristics such as depth, traffic separation schemes and harbor entrances.

\subsubsection*{Problem statement}
This research will be separated into two parts. In those parts, the challenge of communication between manned and unmanned vessels is considered. The first part aims to determine how communication can be avoided, by giving insight into the time-domain for decision making. By looking at the the decision process, critical situation can be selected which are analyzed in more detail. Using manoeuvring models these situations can be simulated to determine the time-domain and determine if this can be improved to ensure that communication is needed in fewer situations.

The second part will focus on the critical situations where communication is a must. Currently there is not yet worked on a communication protocol between manned and unmanned vessels. To ensure this will not delay the implementation of unmanned autonomous vessels. A new communication protocol is needed, based on existing systems and protocols. The aim of this part is to validate if this is sufficient to ensure the right information is shared. This is done by using the \acf{sCE} method to quickly define a new protocol which can be tested in a simulation environment.

\subsubsection*{Research questions}
This research will be a start, to solve the challenge of communication between manned and unmanned vessel. This can be used as a foundation to build a system for decision making which ensure safe operation of both manned and unmanned vessels. The parts relate to different research domains: Maritime Technology and Computer Science. Which resulted in two research questions:

\begin{quotation}
	\emph{How do ship manoeuvrability characteristics influence the domain for decision making, to ensure that the chosen strategy will result in a safe minimal distance between vessels in critical situations?} 
\end{quotation}

\begin{quotation}
	\emph{Are existing systems and protocols for communication, sufficient to ensure that manned and unmanned ships can operate side by side safely?}
\end{quotation}

\subsubsection*{Report structure}
\todo{link structure to chapters}
%Within the project, several steps will be taken. At first more context is given on the current state of the maritime industry in respect to autonomous sailing and ways of communication. Thereby mentioning how the knowledge developed in this research, will help to tackle challenges.
%Next a more detailed plan of approach for the development of an application is given, showing which factors should be taken into account and how this will eventually lead to an application in which models can be verified and validated.
%
%Followed by two different parts. First on the forming of a time-domain for decision making. Within a simulation environment, a model is formed which is based on the current situation for sharing information. This model is used to form strategies to cope with specific situations which occur while sailing. To validate if it is possible to use this model, different scenarios are defined which are used as test-cases. These scenarios will be tested within the simulation environment. Requirements for this simulation are first defined, which lead to user stories and an implementation.
%This simulation is finally validated together with seafarers on the quality of the communication and if ships act realistically within the different scenarios. This is used to give an qualitative advice on how to optimize communication, or more specifically which information should be shared.
%
%The next part will use the previously developed model where variation are made to the information which will be shared between vessels. To determine which information should be shared in more complex situations where current systems for sharing information are not sufficient. Thereby is studied what the impact is of varying the information on the previously mentioned strategies to cope with the specific situations.
