\chapter*{Introduction}
\label{sec:introduction}
\addcontentsline{toc}{chapter}{Introduction}

The Maritime industry is tapping into the world of automation and digitalization. The automation level of vessels is growing rapidly. Vessels are getting connected to the shore by different means. Operational data is becoming available and enriched with other data sets like design data, weather data and maintenance data. These technologies boost the development of autonomous and unmanned vessel designs \cite{Blanke2017}. 
At the same time is the Maritime Industry facing challenges when it comes to crewing and safety \cite{Cappelle2018}. Combined with an increasing competition it is hard to make healthy margins. These trends trigger Maritime Industry to embrace autonomous sailing technologies to secure a healthy future.

These business developments reflect in the significant amount of research and development projects \cite{SMASH2017} \cite{Eriksen2017} \cite{MUNIN2016} \cite{Sames2017} \cite{RollsRoyce2015} \cite{Waterborne2016}. Each one of these projects tackle one or more challenges in the development of autonomous sailing. However, is the challenge of communication between unmanned and manned vessels not been tackled \cite{Saarni2018}. In chapter~\ref{ch:future}, these challenges and corresponding projects are discussed. Currently no solutions are available. However this is necessary to ensure the safety of all vessels: manned, unmanned, remote, automated and autonomous.

This report presents the results of my thesis on communication between unmanned autonomous sailing vessels and manned vessels. It presents a design philosophy which has been translated in a methodology for handling communication which is eventually translated in protocols which are derived based on theory and validated by means of simulations.

% More people start to believe in the possibilities of unmanned and autonomous shipping, as there are many projects which try to develop the necessary technologies \cite{SMASH2017} \cite{Eriksen2017} \cite{MUNIN2016} \cite{Sames2017} \cite{RollsRoyce2015} \cite{Waterborne2016}. The major reason for the development of autonomous ships are the benefits for operational expenses. But one of the major reasons for acceptance is the mitigation of human errors. Which can be achieved by taking out humans from the chain of command, this will however introduce new challenges \cite{Saarni2018}. In chapter \ref{ch:future} these challenges and corresponding projects are discussed. One of the less explored challenges is the issue of communication. This becomes harder between manned and unmanned vessels. Communication is used to share information on the intentions or discuss future actions. At the moment there is no protocol which can be used by unmanned vessels. However this is necessary to ensure the safety of all vessels: manned, unmanned, remote, automated and autonomous.
% To solve this challenge, there are two possibilities: Avoid the need for communication, or develop a new communication protocol.

\subsubsection*{Context}
In many situations is the \acf{COLREGs} sufficient to determine the intentions of other vessels \cite{IMO1972}. They can be seen as ship separation rules, which guide all vessels to make early and correct alterations to their course. It will take more time to assess the situation when using \ac{VHF}. Meaning there is less time to act, which limits the possible strategies. These rules do also apply to autonomous ships, and thus can be used when manned and unmanned vessels meet.
Examples are to stay on starboard side of the shipping lane, and not to cross other ships with small relative angle. When this does not happen, communication is necessary. When this communication doesn't happen correctly, accidents will occur (appendix \ref{app:accidents}). Thereby are there also situations in which regulations are contradictory, such as the accident between Artadi and St-Germain as described in appendix \ref{sec:artadiVSst-germain}. This occurs more often in complex situations, such as a harbor approach.

The risks when \ac{COLREGs} are not sufficient can be mitigated by taking decisions, well in advance. What well in advance means, depends on the manoeuvrability of a vessel. A cargo vessel will follow common paths, while a small tug boat might move around much more. Which results in more false positives on potential collisions with other vessels, when using the same safety domain and evaluation system. The manoeuvrability means that it is necessary to think ahead several minutes with a large ship, while this is much shorter for a small tug boat. This time-domain for decision making depends not only on the ship characteristics, but also on the waterway characteristics. In chapter~\ref{ch:model} this will be discussed in more detail, examples are the depth, traffic separation schemes and harbor entrances.

If it is not possible to take decision well in advance, it is often due to a lack of information. This can be solved by communicating. Communication between manned vessels happens in different ways. Most important is communication via \ac{VHF}, thereby is information from the AIS used to identify vessels. Also can this information help to determine the intentions of the vessel, based on the type of vessel or destination. The used protocols for these systems are discussed in appendix~\ref{app:systems}

\subsubsection*{Design philosophy}
We developed a strategy to cope with the challenge of communication between manned and unmanned vessels, without introducing new systems to the bridge of manned vessels. 
First we will try to avoid the need for communication, by ensuring safe navigation. When this is not possible, there is a need for communication. For this communication a new communication protocol must be developed.
This strategy can be used as a foundation to build a system for decision making which ensures safe operation of both manned and unmanned vessels. To implement this strategy, different steps must be taken. 

First insights have been acquired on the decision model for safe navigation, to determine which factors are taken into account to make a decision. The next step is to determine how these factors influence the decision process. This can also be used to determine the critical situations. Showing when it isn't possible to navigate without communicating. This leaves some specific situations for which a communication protocol must be developed. Using the \acf{sCE} method, the operational demands, relevant human factors and envisioned protocol can be defined. Where the current state of the maritime industry is taken into account.

\subsubsection*{Research questions}
Within the report, different research domains will be touched upon in more detail: Maritime Technology and Computer Science. Where the part on maritime technology focuses on the situation where manoeuvring is just enough. While the computer science part will focus on the development of the communication protocol. The following research questions will be answered for these domains:

\begin{quotation}
	\emph{How do ship manoeuvrability characteristics influence the domain for decision making, to ensure that the chosen strategy will result in a passing distance which does not require communication?} 
\end{quotation}

\begin{quotation}
	\emph{Will a protocol based on existing maritime systems and communication protocols be sufficient to ensure safe navigation, while manned and unmanned vessels encounter each other?}
\end{quotation}

\subsubsection*{Report structure}
This research will be separated into three parts. In those parts, the challenge of communication between manned and unmanned vessels is considered. The part~\ref{part:general} gives a more detailed context of autonomous shipping. Using this context and a decision-model, can be shown why communication is a challenge and how this challenge can be tackled.

Part \ref{part:MT} aims to determine how communication can be avoided. To determine this, it is important to first determine when communication is necessary. This is in case of critical situations. These situations are tested in a simulation environment. This means the time-domain for decision making can be determined to ensure safe operation. Which depends on the moment decisions made in critical situations. These situations are simulated, using a validated manoeuvring model in the simulation environment. Thereby is also evaluated if the time-domain can be improved to ensure that communication is needed in fewer situations.

Part \ref{part:CS} will focus on the critical situations where communication is a must. Currently there is not yet worked on a communication protocol between manned and unmanned vessels. Communication is necessary in cases where there is a lack of trust, missing information or the time-domain for decision making becomes critical. To ensure a fest implementation of the communication protocol between manned and unmanned vessels, will it be based on existing systems and protocols. The aim of this part, is to validate if it is possible to define a protocol using the \acf{sCE} method, which will be accepted by seafarers.
