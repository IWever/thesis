\chapter*{Introduction}
\label{sec:introduction}
\addcontentsline{toc}{chapter}{Introduction}

Considering the major projects within the maritime industry \cite{SMASH2017} \cite{Eriksen2017} \cite{MUNIN2016} \cite{Sames2017} \cite{RollsRoyce2015} \cite{Waterborne2016}. It can be assumed that unmanned and autonomous shipping are getting closer. Mitigating human errors can be done by taking out seafarers from the chain of command, however this will introduce new challenges. For example, the communication with other vessels and decision making in unknown situations. Communication is used to share information on the intentions or discuss future actions. This means the safety of all vessels (manned, unmanned, remote, automated and autonomous) can be ensured. The question which should be asked is: When do vessels have to know each others intentions, certainly when only limited communication is possible? This is most relevant when sharing short term intentions which determine the path. As decisions must be made on changes for the path, within minutes or even shorter.

\subsubsection*{Context}
A computer might like to have as much information as possible. But in case of manned vessels this is not practical, as this will results in an information overload of the crew and communication channels \cite{CCNR2017}. Currently is secured in two ways, how intentions of other vessel can be acquired. First and foremost are the \ac{COLREGs}\cite{IMO1972}, rules applicable to all vessels, as these rules are concrete thus can be programmed and used also by autonomous vessels. Examples are to stay on starboard side of the shipping lane and to not cross other ships with small relative angle. However in critical situations such as the entering of harbors or in busy parts of the world, the VHF radio is also used to ensure that intentions are clear, as the \ac{COLREGs} are not always sufficient anymore in these cases.

To make autonomous shipping possible, autonomous vessels should know how to communicate their intentions. Doing this by behaving in a predictable manner and without overloading the different VHF channels (including AIS). Other options are to develop a new system, or separate autonomous traffic from other vessels. This will be much harder to implement, but even in those cases, the manner of sharing intentions should still be worked out. This is done by making sure that the radio is used correctly, clear signs which can be interpreted from the behavior and data exchange in other ways. Which all ensures that other vessels know enough about the intentions to anticipate correctly, without overloading communication channels.

To solve the problem, different parts should be taken into account, different vessels have different expectations for other vessels, as characteristics which determine the manoeuvrability differ and strategies for specific situations depend on many factors. For example will a cargo ship follow a fairly predictable path. While a small tug boat might move around much more. This means much more false positives on potentional collisions with other vessels. The manoeuvrability also means that it is necessary to think ahead several minutes with a long and heavy ship, while this is much shorter for a small tug boat. This time-domain for decision making depends not only on the ship characteristics, but also the waterway characteristics such as depth, traffic separation schemes and harbor entrances.

\subsubsection*{Problem statement}
This research will separated into two parts. The first will consider the situation where verbal radio communication is taken out completely. In this case decisions must be made clear earlier and more bold to ensure others are aware of your intentions and can anticipate correctly to this. To ensure an unimpeded voyage this must be done in time. This so-called time-domain for decision making can be determined by taking several steps. First the possible decisions must be determined, these are mapped to possible positions using the manoeuvring capabilities of the ship. Combing these results with similar analysis for surrounding ships, the likely decisions can be determined which will lead to an unimpeded voyage. Where an unimpeded voyage means that no verbal communication via VHF-radio is needed, to ensure a passage without perceived risk.

The second part will focus on sharing information to optimize estimation of the time-domain. Much has been done and is still under development on how to get more information to the seafarer. However more information is not the solution. The right information is more important. In the situation as described in the first part, there will be no verbal communication. Therefor other means of communication are needed. Using for example sensors like radar, communication via AIS or a new system. The raw data from these sources will together form the information needed to create the decision domain. The first step to known which sources are needed, is to know which information is desired. Thereby will be determined what the impact of different information is on the decision making process.

Both parts will model the steps from interpreting information and making strategies based on this information, also know as situational awareness of the vessels. To test these models, a tool is developed. In this tool situations at sea can be simulated. The input will be the different models developed in the above described parts, together with situations and sea. The simulation shows which strategies are chosen by the different sets in different scenarios, when different models are implemented.

\todo{Add figure with line with available information, certainty correct decision is being made. Supporting why research is relevant.}

\subsubsection*{Research questions}
The above mentioned models are all steps towards autonomous shipping. This research will be a start, including mostly qualitative research. Eventually it is the aim to use the models to form a register of possible decisions. By looking into which information is most relevant for the different vessels. This will be done from two perspectives within the domains of Maritime Technology and Computer Science. Resulting in two research questions:

\begin{quotation}
	\emph{How do ship characteristics influence the time-domain for decision making, to ensure that the chosen strategy will result in a safe minimal distance between vessels?} 
\end{quotation}

\begin{quotation}
	\emph{Which information must be shared between vessels, to ensure that likely decisions of other vessels can be predicted correctly?}
\end{quotation}

\subsubsection*{Report structure}
Within the project, several steps will be taken. At first more context is given on the current state of the maritime industry in respect to autonomous sailing and ways of communication. Thereby mentioning how the knowledge developed in this research, will help to tackle challenges.
Next a more detailed plan of approach for the development of an application is given, showing which factors should be taken into account and how this will eventually lead to an application in which models can be verified and validated.

Followed by two different parts. First on the forming of a time-domain for decision making. Within a simulation environment, a model is formed which is based on the current situation for sharing information. This model is used to form strategies to cope with specific situations which occur while sailing. To validate if it is possible to use this model, different scenarios are defined which are used as test-cases. These scenarios will be tested within the simulation environment. Requirements for this simulation are first defined, which lead to user stories and an implementation.
This simulation is finally validated together with seafarers on the quality of the communication and if ships act realistically within the different scenarios. This is used to give an qualitative advice on how to optimize communication, or more specifically which information should be shared.

The next part will use the previously developed model where variation are made to the information which will be shared between vessels. To determine which information should be shared in more complex situations where current systems for sharing information are not sufficient. Thereby is studied what the impact is of varying the information on the previously mentioned strategies to cope with the specific situations.
