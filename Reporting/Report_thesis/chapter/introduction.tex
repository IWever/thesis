\chapter*{Introduction}
\label{sec:introduction}
\addcontentsline{toc}{chapter}{Introduction}
\markboth{INTRODUCTION}{INTRODUCTION}

The Maritime industry is tapping into the world of automation and digitalisation. The automation level of vessels is increasing rapidly. Vessels are being connected to the shore by different means. Operational data is becoming available and enriched with data for weather, ship design and maintenance statistics. These technologies boost the development of autonomous and unmanned vessel designs \cite{Blanke2017}. 
At the same time the Maritime Industry faces challenges when it comes to crewing and safety \cite{Cappelle2018}. Also increasing competition makes it hard to realise healthy margins. These trends trigger the Maritime Industry to embrace autonomous sailing technologies to secure a healthy future. 

These business developments are reflected in the significant amount of research and development projects \cite{SMASH2017} \cite{Eriksen2017} \cite{MUNIN2016} \cite{Sames2017} \cite{RollsRoyce2015} \cite{Waterborne2016}. Each of these projects tackles one or more challenges in the development of autonomous sailing. However, the topic of communication between unmanned and manned vessels has not been touched upon \cite{Saarni2018}. In chapter~\ref{ch:future}, the challenges and related projects are discussed. Currently, no solutions are available yet, yet it is necessary to ensure the safety of all vessels: manned, unmanned, remote, automated and autonomous.

This report presents the results of the study on communication between unmanned autonomous sailing vessels and manned vessels. It presents a design philosophy which has been translated into a methodology for handling communication. Both by staying well clear to avoid communication, and the development of a protocol that is derived from current systems and protocols and that is evaluated with experts.

\subsubsection*{Context}
In many situations the \acf{COLREGs} is sufficient to determine the intentions of other vessels \cite{IMO1972}. These can be seen as ship separation rules, which guide all vessels to make early and correct alterations to their course. It will take more time to assess the situation when using \ac{VHF}, as there is less time to act limiting the possible strategies. These rules also apply to autonomous ships, and can thus be used when manned and unmanned vessels meet.
Examples of such regulations are to stay on the starboard side and to make clear manoeuvres. When these rules cannot be applied, communication is necessary. There is a much higher risk of accidents when this communication does not follow protocols. There are also situations in which regulations are contradictory, such as the accident between Artadi and St-Germain (appendix \ref{app:accidents}). These contradictions occur more often in complex situations, such as a harbour approach.

The risks when \ac{COLREGs} are not sufficient can be mitigated by making decisions well in advance. What well in advance means depends on the manoeuvrability and operation of a vessel. A cargo vessel will follow common paths, while a small tugboat might move around much more. These situations result in more false positives on potential collisions with other vessels when using the same safety domain and evaluation system. The impact of manoeuvrability means that it is necessary to think several minutes ahead with a large ship, while this timespan is much shorter for a small tugboat. The time-domain for decision-making depends not only on the ship characteristics but also on the waterway characteristics. Chapter~\ref{ch:model} discusses this in more detail. Examples are the depth, traffic separation schemes and harbour entrances.

If it is not possible to make a decision well in advance, it is often due to a lack of information. Communication solves this problem. This communication between manned vessels happens with different means. Most important is communication via \ac{VHF}, also is information from the AIS used to identify ships. This information can also help to determine the intentions of the vessel based on the type of vessel or destination. The used protocols for these systems are discussed in appendix~\ref{app:systems}.

\subsubsection*{Design philosophy}
We developed a strategy to cope with the challenge of communication between manned and unmanned vessels, without the introduction of new systems to the bridge of manned vessels. 
At first, we will try to avoid the need for communication, by staying well clear from other ships. Due to circumstances this might not be possible and communication is required. For the communication between manned and unmanned vessels, a new communication protocol must be developed.
The strategy for developing this protocol can be used as a foundation to build a system for decision-making that ensures safe operation for both manned and unmanned vessels. Multiple steps are required to develop and implement this strategy and protocol.

First, the decision model for safe navigation has been analysed in order to deduce the relevant factors for decision-making. The next step is to study how these factors influence the decision-process. This is used to specify critical situations. In those critical situations it isn't possible to navigate without communication. Thus identifying situations where a communication protocol is necessary. Using the \acf{sCE} method, the operational demands, relevant human factors and envisioned protocol are defined. The current state of the maritime industry is thereby taken into account.

\subsubsection*{Research questions}
The report covers two research domains: Maritime Technology and Computer Science. The part on maritime technology focuses on the situation where manoeuvring is just enough to avoid the need for communication. The computer science part focuses on the development of a communication protocol. Answering the following research questions:

\begin{quotation}
	\emph{How do ship manoeuvrability characteristics influence the domain for decision making, to ensure that the chosen strategy will result in a closest point of approach that does not require communication?} 
\end{quotation}

\begin{quotation}
	\emph{Will a protocol based on existing maritime systems and communication protocols be sufficient to ensure safe navigation, while manned and unmanned vessels encounter each other?}
\end{quotation}

\subsubsection*{Report structure}
This report contains four parts. Each part discusses the challenge of communication between manned and unmanned vessels. Part~\ref{part:general} describes the context of autonomous shipping. This context and a decision-model show why communication is a challenge, and how this challenge can be tackled.

Part~\ref{part:MT} describes how communication can be avoided. It is important first to determine when communication is necessary. This is especially the case in critical situations, i.e. deviating from the standard safety guidelines. These situations are tested in a simulation environment. The simulations help to define the time-domain for decision-making to ensure safe operation. These situations are simulated, using a validated manoeuvring model. Thereby is evaluated whether ships need communication in fewer cases if the decision-domain is improved.

Part \ref{part:CS} will focus on critical situations where communication is a must. A communication protocol between manned and unmanned vessels has not been part of any known project. Communication is necessary when there is a lack of trust, missing information or the time-domain for decision-making becomes critical. The objective is to validate whether it is possible to design a protocol that is based on existing systems and protocols. The \acf{sCE} method is used.

In part~\ref{part:conclusion} is finally concluded what the results from the previous parts mean for the maritime industry and what the next steps should be to ensure safe operation of autonomous vessels, related to the challenge of communication.
