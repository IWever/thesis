\chapter*{Introduction}
% Describe what this part is about and which question it answers
Critical situations are moments during a voyage where it is most important that is known what the intentions are of other vessels. But due to the chaotic situation this is not always possible. Supporting the crew in showing possible intentions of other vessels will help to create situational awareness faster. 

But this is not the same for different vessels, as a long heavy ship will mostly go straight ahead at a similar speed, while a small tug boat might go all over the place. Thereby is there the impact of traffic separation schemes and harbor entrances on the likelihood of manoeuvring in a certain direction. 
By adjusting the course in an early stage the intentions can be made clear, without the need for communication. This will lead to an unimpeded voyage. Beside the advantage of less pressure on the crew, is it also more easy to have autonomous vessels sailing between manned vessels.
The moment these intentions have to communicated is highly dependent on the type of vessel or ship characteristics. During an unimpeded voyage it is not needed to communicate as intentions are in all cases clear. This has led to the following research question:

\begin{quotation}
	\emph{How do ship characteristics influence the time-domain for decision making to ensure an unimpeded voyage?} 
\end{quotation}

The method used within this research is to create a simulation for different situations, showing a visualization of the possible decisions. Extending this with experience from seafarers to improve the interpretation of the tool. To eventually succeed in predicting when the crew has to act to secure an unimpeded voyage.




Question: How do ship characteristics influence the time-domain for decision making to ensure an unimpeded voyage? 

ship characteristics:?

Unimpeded voyage: a voyage where is it possible to correctly predict the intentions of other vessels and adapt to this in a timely manner in such a way that the COLREGs are sufficient for route planning.



Hypothesis: