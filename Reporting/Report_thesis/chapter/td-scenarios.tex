\chapter{Scenario's}
\label{ch:scenarios}
To gain more insight into the decision process, and when information is desired so the right decision can be made in time simulations are done. These will be done for several standard situations and real scenarios which occur at the North Sea.

\section{Standard situations}
To verify if the tool is working as expected, different standard situations and scenarios are tested. Which are also used to determine the thresholds to execute different strategies. The situations, scenarios and related strategies are shown in table \ref{tab:scenarios-standard}.

\begin{table}[h]
	\makebox[\textwidth][c]{
		\begin{tabular}{lll}
			Situation & Scenario & Strategy \\
			\hline
			Simple crossing & other ship from port side & passing at the front \\
			Simple crossing & other ship from port side & passing at the back \\
			Simple crossing & other ship from starboard & passing at the front \\
			Simple crossing & other ship from starboard & passing at the back \\
			Crossing & with a small angle & passing at the front, with larger angle \\
			Crossing & with a small angle & passing at the back, with larger angle\\
			Take-over & ship in front & wait till there is enough space \\
			Enter traffic lane & to port side with ship from starboard & get in front\\
			Enter traffic lane & to port side with ship from port & get in front\\
			Enter traffic lane & to starboard side with ship from port & get in front\\
			Enter traffic lane & to starboard side with ship from starboard & leading to no interaction\\
		\end{tabular}
	}
	\caption{Standard scenario's}
	\label{tab:scenarios-standard}
\end{table}

During the simulation an analyses is made what the ships perceive as hazards. This also depends on the acceptable distance for a well-clear situation. By varying this distance, can be seen if strategies will change. 
Thereby will be verified if \ac{COLREGs} are followed.
Varying the shared information will result in a better understanding of the mental model in those situations and can be seen what the effect is on recognizing hazards in the correct way.
\todo{Make clear visualizations of scenarios}

\section{Real scenarios}
Two scenarios will be simulated, first a situation such as the crash between Al Oraiq and Flinterstar as described in section \ref{sec:al-oraiq-vs-flinterstar}. Where there will be different strategies, if the Al Oraiq took other decisions, if the Flinterstart took other decisions and if both used different strategies. 

The second situation is the entering of the Maasgeul from the Beerkanaal at the end of the Maasvlakte, specifically in the situation when there are multiple vessels already coming from the Maasgeul, Nieuwe Waterweg and possibly the Callandkanaal.

\todo{Add maps of situations}