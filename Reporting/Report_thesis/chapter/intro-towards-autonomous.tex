Safety at sea has been a relevant topic as long as ships exist. Nowadays communication has become very important to ensure the safety of all ships. Where communication can be all forms of sharing information. Before the invention of radio communication, ships literally lost all connection with the shore and other ships when setting sail. Flags were used when ships were close to others ships or to the shore. This form of communication was not complete as it only gave limited insight in the intentions of other vessels for example.

New technologies led to new ways for communication, which subsequently led to safer operation. But new technologies also lead to complexer situations. As ships get bigger and perform more complex operations, resulting in less manoeuvrability. In those situations it is very important to share the right information at the right time. The current situation of the maritime is assessed together with a summary of likely new technologies related to navigational safety. 

First the technologies which are currently used to ensure safe navigation of vessels is discussed. This is followed by a description of accidents which occurred, giving insight in the mistakes made and how this can be avoided with improved systems. Which projects on these new systems and unmanned ships are discussed in chapter \ref{ch:future}. The development of autonomous and unmanned ships are discussed, as these have advantages when it comes to safety. The last chapter of this part will describe a model to discusses the knowledge which will autonomous and unmanned ships will be safer.