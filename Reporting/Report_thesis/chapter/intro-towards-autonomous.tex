\markboth{PART I}{PART I}

Safety at sea has been a relevant topic for as long as ships exist. Nowadays communication has become very important to ensure the safety of all vessels. Communication can be all forms of sharing information. Before the invention of radio communication, ships lost all connection with the shore and other ships when setting sail. Ships used flags when they were close to other ships or the coast. This form of communication was not complete as it only gave limited insight into the intentions of other vessels for example. To ensure the safety of ships, they agreed on how to act in specific situations. These agreements became the foundation of the regulations as written down in the \acf{COLREGs} \cite{IMO1972}.

New technologies led to new ways for communication, such as radio communication. This led subsequently to safer operations. Communication works very well between manned vessels, as a human can work well with limited unstructured information compared to computers. Communication is a challenge for unmanned ships. But autonomous ships are getting closer. Thus a solution must be found. Thereby comes that new technologies have led to more complex situations, as ships get bigger and perform more complex operations. Due to the limitations of unmanned vessels when it comes to communication, it does become even more critical to make the right decisions well in advance. To avoid critical situations and enable those ships to share the correct information at the right moment in time.

Many projects are working on unmanned autonomous ships. Chapter \ref{ch:future} starts to describe why steps are taken towards autonomous and unmanned shipping, including the economic and social incentives. Followed by a description of projects, showing the technological push factor. This description will give more insight into the challenges as seen relevant by others, for the introduction of unmanned vessels and the communication between manned and unmanned ships. A distinction is made between the exploratory projects, aimed to develop a vision of the future, and the applied projects intended to develop prototypes in the shorter term.
Chapter \ref{ch:model} relates these challenges to the decision model for ships. This decision model illustrates that the need for communication depends on the steps taken in the decision process, i.e. that it might be possible to avoid communication and ensure safe operation.