Safety at sea has been a relevant topic as long as ships exist. There are several factors which have a big influence on safety. First and foremost is communication in all the ways where information is shared. Before the invention of radio communication, ships literally lost all connection with the shore and other ships when setting sail. Flags were used when ships were close to others ships or to the shore. This form of communication was not complete as it only gave limited insight in the intentions of other vessels for example.

New technologies lead to new ways for communication, which subsequently lead to a safer industry. But new technologies also lead to complexer situations. As ships get bigger and perform more complex operations, resulting in less manoeuvrability. In those situations it is very important to share the right information. The current situation of the maritime is assessed together with a summary of likely new technologies related to navigational safety. 

To give insight in the current situation of the shipping industry, first some accidents are discussed and which lessons can be learned from this. This is followed by a description of the current means for communication. This eventually all comes back to the different elements of the bridge. Giving insight in how accidents occur and what is currently available to prevent them. This is a basis for technologies which will be developed at the moment, and are discussed in the last chapter of this part. The above mentioned knowledge is used as a starting point for the model to gain insight when decisions should be made in current operations to improve safety.