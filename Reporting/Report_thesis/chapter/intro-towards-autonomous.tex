Safety at sea has been a relevant topic as long as ships exist. Nowadays communication has become very important to ensure the safety of all ships. Where communication can be all forms of sharing information. Before the invention of radio communication, ships literally lost all connection with the shore and other ships when setting sail. Flags were used when ships were close to others ships or to the shore. This form of communication was not complete as it only gave limited insight in the intentions of other vessels for example. To ensure the safety of ships, they agreed how to act in specific situations. These agreements became the foundation of the regulations as written down in the \acf{COLREGs} \cite{IMO1972}.

New technologies led to new ways for communication, such as radio communication. Which subsequently led to safer operation. This works very well between manned vessels, as human can work well with limited unstructured information compared to computers. For unmanned ships is this however a challenge. With autonomous ships getting closer, a solution must be found. Also because new technologies have led to more complex situations. As ships get bigger and perform more complex operations. Due to the limitations of unmanned vessels when it comes to communication, does it become even more important to make the right decisions well in advance. To avoid critical situations, or enable those ships to share the right information at the right time.

There are many projects working on unmanned autonomous ships. Chapter \ref{ch:future} starts to describe why steps are taken towards autonomous and unmanned shipping, including the economical and social incentives. Followed by a description of projects, showing the technological push factor. This will give more insight into the challenges as seen by others, for the introduction of unmanned vessels. Which shows the relevance of research into the communication between manned and unmanned vessels. A separation is made between the more exploratory projects aimed to form a vision of the future, and the applied projects aimed develop prototypes on the short term.
Chapter \ref{ch:model} relates these challenges to the decision model for ships. This shows how the need for communication depends on the steps taken in the decision process, thus that it might be possible to avoid communication and ensure safe operation.